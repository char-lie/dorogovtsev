\chapter{Основы}
\section{Методы оценок характеристик распределения
    наблюдаемых случайных величин}
$x_1, \dots, x_n$ --- независимые одинаково распределённые случайные величины
с неизвестной функцией распределения $F$.

Цель --- найти $F$ или сказать что-то о её свойствах.

\begin{definition} Эмпирической (выборочной) функцией распределения,
    построенной по выборке $x_1, \dots, x_n$ называется функция
    $$\cdfn{x}=\frac{1}{n}\cdot \sum_{k=1}^n
    \indicator{x_k\le x}$$
\end{definition}

$$\cdf{x}=\Probability{x_1 \le x}
    =\mean{\indicator{x_1\le x}}$$

\begin{theorem}
    $$\probability{\cdfn{x}\dCovergence \cdf{x}}=1$$
\end{theorem}

Как можно попытаться отследить плотность распределения?
Постараемся найти функцию распределения, а потом и плотность.

Допустим, $F$ имеет хорошую (непрерывную) плотность.
Тогда как из $F$ получить $p$?

Мы знаем, что $F'=p$, но это никому не нужно, так как $F_n'$ --- производная
ступенчатой функции, которая почти везде будет равна нулю.

Но также мы помним, что
$$\cdf{b}-\cdf{a}=\int\limits_a^b \pdf{x} dx$$

Тогда, положив $a=x$, взяв некую $\Delta$, и постановив $b=x+\Delta$,
получаем следующее
$$\cdf{x+\Delta}-\cdf{x}=\int\limits_x^{x+\Delta} \pdf{y} dy$$

Делим обе части на $\Delta$ и при достаточно малых его значениях получаем
$$\frac{1}{\Delta}\cdot\int\limits_x^{x+\Delta} \pdf{y} dy
=\frac{\cdf{x+\Delta}-\cdf{x}}{\Delta}
\approx\frac{d\cdf{x}}{dx}=\pdf{x}$$

Значит, можем заменить $\pdf{x}$ не производной, а такой разностью.
$$\pdf{x}\approx\frac{\cdf{x+\Delta}-\cdf{x}}{\Delta}$$

Возьмём выборку из $m$ случайных величин в порядке возрастания
$a_1, \dots, a_m$, обозначим отрезки $I_j=[a_{j-1},a_j]$
и введём функцию $q\left(y\right)$
$$q\left(y\right)
=\sum_{j=1}^m \frac{\cdf{a_j}-\cdf{a_{j-1}}}{a_j-a_{j-1}}
    \cdot\indicator{y\in I_j}$$

Теперь введём последовательность функций $q_n\left(y\right)$ и видим,
что она сходится к $q_n\left(y\right)$ почти наверное
согласно закону больших чисел,
а та в свою очередь имеет сходимость порядка $\frac{1}{n}$
к плотности распределения $\pdf{y}$
\begin{equation}\label{eq:histogram_start}
q_n\left(y\right)
=\sum_{j=1}^m \frac{\cdfn{a_j}-\cdfn{a_{j-1}}}{a_j-a_{j-1}}
    \cdot\indicator{y\in I_j}
\end{equation}

Отметим, что $q_n$ сходится к $q$ почти наверное, а $q$ в свою очередь
сходится к $p$
$$q_n\left(y\right)\acovergence q\left(y\right)\covergencen{m}{}\pdf{y}$$

Функция $q_n$ называется \textbf{гистограммой}.

Избавимся от $a_j$ в формуле, а для этого вспомним, чему равно $\cdfn{x}$
$$\cdfn{x}=\frac{1}{n}\cdot \sum_{k=1}^n
\indicator{x_k\le x}$$

Теперь посмотрим, чему равна разность $\cdfn{a_j}-\cdfn{a_{j-1}}$,
которая, как мы видим, является вероятностью того,
что $x$ попало в отрезок $I_j$
$$\cdfn{a_j}-\cdfn{a_{j-1}}
    =\frac{1}{n}\cdot \sum_{k=1}^n
    \indicator{x_k\le a_j}-\frac{1}{n}\cdot \sum_{k=1}^n
    \indicator{x_k\le a_{j-1}}$$

Сгруппируем слагаемые и получим чуть более компактную запись разности
\begin{equation}\label{eq:cdfn_difference}
    \cdfn{a_j}-\cdfn{a_{j-1}}=\frac{1}{n}\cdot \sum_{k=1}^n
    \left[\indicator{x_k\le a_j}-\indicator{x_k\le a_{j-1}}\right]
\end{equation}

Рассмотрим возможные значения индикаторов

Если оба индикатора равны единице,
это значит, что $x_k$ не больше $a_j$ и не больше $a_{j-1}$.
Поскольку $a_{j-1}\le a_j$, то можно обойтись тем, что $x\le a_{j-1}$
\begin{align*}
    \begin{cases}
        \indicator{x_k\le a_j}=1\\
        \indicator{x_k\le a_{j-1}}=1\\
        a_{j-1}\le a_j
    \end{cases}
    \Rightarrow
    \begin{cases}
        x_k\le a_j\\
        x_k\le a_{j-1}\\
        a_{j-1}\le a_j
    \end{cases}
    \Rightarrow
        x_k\le a_{j-1}\le a_j
    \Rightarrow
        x_k\le a_{j-1}
\end{align*}

Такая ситуация,
что $x$ больше, чем $a_j$, но не больше, чем $a_{j-1}$, невозможна,
так как $a_{j-1}$ не больше, чем $a_j$,
а признать возможной такое положение дел ($a_j<x_k\le a_{j-1}$)
означало бы то, что $a_j<a_{j-1}$
\begin{align*}
    \begin{cases}
        \indicator{x_k\le a_j}=0\\
        \indicator{x_k\le a_{j-1}}=1\\
        a_{j-1}\le a_j
    \end{cases}
    \Rightarrow
    \begin{cases}
        x_k>a_j\\
        x_k\le a_{j-1}\\
        a_{j-1}\le a_j
    \end{cases}
    \Rightarrow
    \begin{cases}
        a_j<x_k\le a_{j-1}\\
        a_{j-1}\le a_j
    \end{cases}
\end{align*}


Если оба индикатора равны нулю,
то это значит, что $x$ строго больше как $a_j$, так и $a_{j-1}$.
Опять же, поскольку $a_{j-1}\le a_j$, то достаточно сказать, что $x>a_j$.
\begin{align*}
    \begin{cases}
        \indicator{x_k\le a_j}=0\\
        \indicator{x_k\le a_{j-1}}=0\\
        a_{j-1}\le a_j
    \end{cases}
    \Rightarrow
    \begin{cases}
        x_k>a_j\\
        x_k>a_{j-1}\\
        a_j\ge a_{j-1}
    \end{cases}
    \Rightarrow
        x_k>a_j\ge a_{j-1}
    \Rightarrow
        x_k>a_j
\end{align*}

Если же $x$ больше, чем $a_{j-1}$, но не больше, чем $a_j$,
то $x$ попадает в полуинтервал $\left(a_{j-1},a_j\right]$
\begin{align*}
    \begin{cases}
        \indicator{x_k\le a_j}=1\\
        \indicator{x_k\le a_{j-1}}=0\\
        a_{j-1}\le a_j
    \end{cases}
    \Rightarrow
    \begin{cases}
        x_k\le a_j\\
        x_k>a_{j-1}\\
        a_j\ge a_{j-1}
    \end{cases}
    \Rightarrow
        a_{j-1}<x_k\le a_j
\end{align*}

Вспомним формулу \eqref{eq:cdfn_difference}
$$\cdfn{a_j}-\cdfn{a_{j-1}}
=\frac{1}{n}\cdot \sum_{k=1}^n
    \left[\indicator{x_k\le a_j}-\indicator{x_k\le a_{j-1}}\right]$$

Очевидно, что нас интересуют те пары, разность которых не равна нулю.
Это значит, что те случаи, когда $x>a_j$ или $x\le a_{j-1}$, нас не интересуют.
Поскольку такой случай, что $a_j<x\le a_{j-1}$ невозможен, то его тоже отбросим.
Значит, остался только тот вариант,
когда $x$ попадает в полуинтервал $\left(a_{j-1},a_j\right]$
$$\frac{1}{n}\cdot \sum_{k=1}^n
        \left[\indicator{x_k\le a_j}-\indicator{x_k\le a_{j-1}}\right]
    =\frac{1}{n}\cdot \sum_{k=1}^n \indicator{x_k\in \left(a_{j-1},a_j\right]}
$$

Пренебрегаем тем, что у нас полуинтервал, и будем считать,
что вероятность попадения $x$ чётко на границу интервала пренебрежимо мала
и заменим индикатор на более удобный.
$$\frac{1}{n}\cdot \sum_{k=1}^n \indicator{x_k\in \left(a_{j-1},a_j\right]}
=\frac{1}{n}\cdot \sum_{k=1}^n \indicator{x_k\in I_j}$$

Получаем компактную запись для разности функций распределения
\begin{equation}\label{eq:cdfn_difference_final}
\cdfn{a_j}-\cdfn{a_{j-1}}
=\frac{1}{n}\cdot \sum_{k=1}^n \indicator{x_k\in I_j}
\end{equation}


Вернёмся к уравнению \eqref{eq:histogram_start}
$$
q_n\left(y\right)
=\sum_{j=1}^m \frac{\cdfn{a_j}-\cdfn{a_{j-1}}}{a_j-a_{j-1}}
    \cdot\indicator{y\in I_j}
    $$

Воспользовавшись тем, что $\left(a_j-a_{j-1}\right)$ --- длина отрезка $I_j$,
а разность $\cdfn{a_j}-\cdfn{a_{j-1}}$ была только что компактизирована,
получаем такую формулу
$$q_n\left(y\right)=\sum_{j=1}^m\frac{1}{n}\sum_{k=1}^n
\indicator{x_k\in I_j}\cdot\frac{1}{\left|I_j\right|}
\cdot\indicator{y \in I_j}$$

Упростим формулу.
Введём функцию, которая считает количество элементов выборки $x_1, \dots, x_n$,
попавших в интервал $I_j$.
Это будет сумма индикаторов того, что элемент $x_k$ попал в интервал $I_j$

$$\nu_j\left(X\right)
=\sum_{x\in X} \indicator{x\in I_j}
=\sum_{k=1}^n \indicator{x_k\in I_j}$$

Поскольку $\indicator{y\in I_j}$ зависит от $j$ и не зависит от $k$,
то его можно перенести во внешнюю сумму. Получаем следующую формулу
$$q_n\left(y\right)
=\sum_{j=1}^m \frac{\indicator{y\in I_j}}{n\cdot\left|I_j\right|}
    \cdot\nu_j\left(X\right)$$

У этой суммы только один ненулевой элемент,
так как $y$ может попасть только в один отрезок
(пренебрегаем возможностью его попадения на границу между двумя отрезками).
Тогда обозначим номер отрезка, в который попал $y$, как $k$,
а функцию $q_n\left(y\right)$ как $q_n^k$
$$q_n^k=
    \frac{\nu_k\left(X\right)}{n\cdot\left|I_k\right|}$$

Что мы тут видим? Теперь $k$ --- номер ``столбика'' гистограммы
(номер интересующего нас отрезка --- номер отрезка, в который попал $y$).

``Высота'' столбика (значение функции на определённом отрезке)
пропорциональна количеству элементов, попавших в этот отрезок (что логично).
Кроме того, происходит деление на общее количество элементов,
которое возникло, чтобы $q\left(y\right)$ сходилось к $\pdf{y}$.

Делителю же $\left|I_k\right|$ отведена особая роль --- он предотвращает
искажение гистограммы при различных длинах отрезков.
Получается, что, чем длиннее отрезок, тем ниже столбик,
так как элементы более ``размазаны'' по отрезку --- тоже логично.

Если рассматривать значение функции как высоту прямоугольника,
а длину отрезка как его ширину (графически это так и изображается),
то оказывается, что отношение количества элементов, попавших в отрезок,
к количеству всех элементов выборки, является площадью прямоугольника
$$S_k=\frac{\nu_k\left(X\right)}{n}$$

\section{Оценка неизвестных параметров}
Снова у нас есть $x_1, \dots, x_n$ --- выборка из распределения $F_\theta$,
где $\theta$ --- неизвестный параметр из множества $\Theta$

\begin{example}Нормальное распределение с известным СКО $\sigma=1$
    и неизвестным математическим ожиданием,
    тогда $\theta$ --- математическое ожидание
\end{example}
\begin{example}Нормальное распределение, в котором неизвестны оба параметра.
    Тогда $\theta$ будет парой $(a,\sigma)$
\end{example}

Главный вопрос --- определение основных параметров.

\begin{definition}Функцию от выборки,
    значение которой заменяет неизвестный параметр,
    назвают оценкой
\end{definition}
\begin{example}Предположим, что выборка сделана из распределения Бернулли,
    то есть $\left\{x_i\right\}$ --- набор одинаково распределённых
    случайных величин, причём
    \begin{align*}
    x_i=
    \begin{cases}
        1,&p\\
        0,&1-p
    \end{cases}
    \end{align*}

    Тогда неизвестный параметр --- величина $p$
    (вероятность удачного эксперимента)
    $$\theta=p\in\left[0;1\right]=\Theta$$

    Введём разные оценки $\hat{p}$
    \begin{align*}
        \hat{p}_1&=\frac{1}{n}\sum_{k=1}^n x_k\\
        \hat{p}_2&=x_1\\
        \hat{p}_3&=
            \frac{2}{n}\sum_{k=1}^{\left\lfloor \frac{n}{2} \right\rfloor} x_k
    \end{align*}
\end{example}
