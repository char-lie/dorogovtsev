 \chapter{Основы}
\section{Методы оценок характеристик наблюдаемых случайных величин}
$x_1, \dots, x_n$ --- независимые одинаково распределённые случайные величины
с неизвестной функцией распределения $F$.
\index{функция распределения!неизвестная}
Логично, что вероятность выпадения каждого $x_k$
(вероятность того, что наугад взятый из выборки $x$ будет равен $x_k$)
одинакова
$$\probability{x=x_k} = \frac{1}{n}$$

Цель --- найти $F$ или сказать что-то о её свойствах.

\subsection{Эмпирическая функция распределения}
\index{функция распределения!эмпирическая}
\index{функция распределения!выборочная}
\begin{definition}[Эмпирическая функция распределения]
    Эмпирической (выборочной) функцией распределения,
    построенной по выборке $x_1, \dots, x_n$, называется функция
    $$\cdfn{x}=\frac{1}{n}\cdot \sum_{k=1}^n
    \indicator{x_k\le x}$$
\end{definition}

\begin{theorem}\label{theorem:distributionFunction:empiricalToCumulative}
    \index{теорема!о восстановлении!неизвестной функции распределения}
    Неизвестная функция распределения $\cdf{x}$ может быть сколь угодно
    точно восстановлена по выборке достаточно большого объёма
    \cite[стр.~25]{BorovkovMS}.
    $$\cdfn{x} \acovergence \cdf{x}$$
\end{theorem}
\begin{proof}[Идея доказательства]
Вспомним, чему равна эмпирическая функция распределения
$$\cdfn{x}=\frac{1}{n}\cdot \sum_{k=1}^n
\indicator{x_k\le x}$$

Заметим, что индикаторы $\indicator{x_k\le x}$
являются независимыми одинаково распределёнными случайными величинами,
а функцию распределения $\cdf{x}$ можно записать следующим образом
$$\cdf{x}=\Probability{x_1\le x}=\mean{\indicator{x_1\le x}}$$

Так как эмпирическая функция распределения является
средним арифметическим индикаторов, то по усиленному закону больших чисел
она сходится к неизвестной функции распределения почти наверное
при устремлении длины выборки к бесконечности
$$\cdfn{x}=\frac{1}{n}\cdot\sum_{k=1}^n\indicator{x_k\le x}
\acovergence\mean{\indicator{x_1}}=\cdf{x}$$

Теорема доказана
$$\cdfn{x} \acovergence \cdf{x}$$
\end{proof}

\subsection{Гистограмма}
\label{subsection:histogram}
Как можно попытаться отследить плотность распределения?
Постараемся найти функцию распределения, а потом и плотность.

Допустим, $F$ имеет хорошую (непрерывную) плотность.
Как тогда из $F$ получить $p$?

Мы знаем, что $F'=p$, но это никому не нужно, так как $F_n'$ --- производная
ступенчатой функции, которая почти везде будет равна нулю.

Но также мы помним, что
$$\cdf{b}-\cdf{a}=\int\limits_a^b \pdf{x} dx$$

Положим $a=x$ и введём $\Delta_x=b-x$
$$\cdf{x+\Delta_x}-\cdf{x}=\int\limits_x^{x+\Delta_x} \pdf{y} dy$$

Делим обе части на $\Delta_x$.
$$\frac{1}{\Delta_x}\cdot\int\limits_x^{x+\Delta_x} \pdf{y} dy
=\frac{\cdf{x+\Delta_x}-\cdf{x}}{\Delta_x}$$

Несложно заметить,
что при достаточно малых значениях $\Delta_x$
получаем плотность распределения $\pdf{x}$
$$\frac{\Delta\cdf{x}}{\Delta_x}
\xrightarrow[]{\Delta_x\to 0}
\frac{d\cdf{x}}{dx}=\pdf{x}$$

Значит, можем заменить $\pdf{x}$ не производной, а такой разностью.
$$\pdf{x}\approx\frac{\cdf{x+\Delta_x}-\cdf{x}}{\Delta_x}$$

Возьмём $m$ полуинтервалов на числовой прямой
$I_j=\left(a_{j-1},a_j\right], i=\overline{1,m}$
таких, каждое значение выборки попадает в свой интервал.
Для этого определим пару свойств точек, ограничивающих эти интервалы:
\begin{enumerate}
    \item Каждая следующая точка строго правее (больше) предыдущей
        (так как зачем нам одинаковые точки?)
        $$a_0<a_1<\dots<a_m$$
    \item Каждое значение выборки должно попадать ровно в один полуинтервал.
        Очевидно, что данные полуинтервалы $I_j$ не пересекаются между собой.
        Значит, осталось потребовать, чтобы
        крайнее левое значение было меньше минимального значения из выборки,
        а крайнее правое --- не меньше максимального
        $$a_0<min\left(X\right)\le max\left(X\right)\le a_m$$
\end{enumerate}

Введём функцию $q\left(y\right)$
$$q\left(y\right)
=\sum_{j=1}^m \frac{\cdf{a_j}-\cdf{a_{j-1}}}{a_j-a_{j-1}}
    \cdot\indicator{y\in I_j}$$

Определим последовательность функций $q_n\left(y\right)$,
заменив $\cdf{x}$ на $\cdfn{x}$ в предыдущем определении
\begin{equation}\label{eq:histogram_start}
q_n\left(y\right)
=\sum_{j=1}^m \frac{\cdfn{a_j}-\cdfn{a_{j-1}}}{a_j-a_{j-1}}
    \cdot\indicator{y\in I_j}
\end{equation}

Отметим, что $q_n$ сходится к $q$ почти наверное, так как сходятся почти
наверное функции распределения согласно теореме
\ref{theorem:distributionFunction:empiricalToCumulative}
$$q_n\left(y\right)\acovergence q\left(y\right)$$

Функция $q$ в свою очередь сходится к $p$, так как приблизительно равна
производной, а при бесконечно большом количестве интервалов их длины
становятся бесконечно малыми
$$q\left(y\right)\covergencen{m}{}\pdf{y}$$

Функция $q_n$ называется гистограммой.
\index{гистограмма}

Избавимся от $a_j$ в формуле, а для этого вспомним, чему равно $\cdfn{x}$
$$\cdfn{x}=\frac{1}{n}\cdot \sum_{k=1}^n
\indicator{x_k\le x}$$

Теперь посмотрим, чему равна разность $\cdfn{a_j}-\cdfn{a_{j-1}}$,
которая, как мы видим, является вероятностью того,
что $x$ попало в отрезок $I_j$
\begin{align*}
    \cdfn{a_j}-\cdfn{a_{j-1}}=\\
    =\frac{1}{n}\cdot \sum_{k=1}^n
        \indicator{x_k\le a_j}-\frac{1}{n}\cdot \sum_{k=1}^n
        \indicator{x_k\le a_{j-1}}
\end{align*}

Сгруппируем слагаемые и получим чуть более компактную запись разности
\begin{eqnarray}\label{eq:cdfn_difference}
    \cdfn{a_j}-\cdfn{a_{j-1}}=\nonumber\\
    =\frac{1}{n}\cdot \sum_{k=1}^n
        \left[\indicator{x_k\le a_j}-\indicator{x_k\le a_{j-1}}\right]
\end{eqnarray}

Рассмотрим возможные значения индикаторов

Если оба индикатора равны единице,
это значит, что $x_k$ не больше $a_j$ и не больше $a_{j-1}$.
Поскольку $a_{j-1}\le a_j$, то можно обойтись тем, что $x\le a_{j-1}$
\begin{align*}
    \begin{cases}
        \indicator{x_k\le a_j}=1\\
        \indicator{x_k\le a_{j-1}}=1\\
        a_{j-1} < a_j
    \end{cases}
    \Rightarrow
    \begin{cases}
        x_k\le a_j\\
        x_k\le a_{j-1}\\
        a_{j-1} < a_j
    \end{cases}
    \\\Rightarrow
        x_k\le a_{j-1} < a_j
    \Rightarrow
        x_k\le a_{j-1}
\end{align*}

Такая ситуация,
что $x$ больше, чем $a_j$, но не больше, чем $a_{j-1}$, невозможна,
так как $a_{j-1}$ не больше, чем $a_j$,
а признать возможной такое положение дел ($a_j<x_k\le a_{j-1}$)
означало бы то, что $a_j<a_{j-1}$
\begin{align*}
    \begin{cases}
        \indicator{x_k\le a_j}=0\\
        \indicator{x_k\le a_{j-1}}=1\\
        a_{j-1} < a_j
    \end{cases}
    &\Rightarrow
    \begin{cases}
        x_k>a_j\\
        x_k\le a_{j-1}\\
        a_{j-1} < a_j
    \end{cases}
    \\&\Rightarrow
    \begin{cases}
        a_j<x_k\le a_{j-1}\\
        a_{j-1} < a_j
    \end{cases}
\end{align*}


Если оба индикатора равны нулю,
то это значит, что $x$ строго больше как $a_j$, так и $a_{j-1}$.
Опять же, поскольку $a_{j-1}\le a_j$, то достаточно сказать, что $x>a_j$.
\begin{align*}
    \begin{cases}
        \indicator{x_k\le a_j}=0\\
        \indicator{x_k\le a_{j-1}}=0\\
        a_j > a_{j-1}
    \end{cases}
    \Rightarrow
    \begin{cases}
        x_k>a_j\\
        x_k>a_{j-1}\\
        a_j > a_{j-1}
    \end{cases}
    \\\Rightarrow
        x_k>a_j > a_{j-1}
    \Rightarrow
        x_k>a_j
\end{align*}

Если же $x$ больше, чем $a_{j-1}$, но не больше, чем $a_j$,
то $x$ попадает в полуинтервал $\left(a_{j-1},a_j\right]$
\begin{align*}
    \begin{cases}
        \indicator{x_k\le a_j}=1\\
        \indicator{x_k\le a_{j-1}}=0\\
        a_j > a_{j-1}
    \end{cases}
    \Rightarrow
    \begin{cases}
        x_k\le a_j\\
        x_k>a_{j-1}\\
        a_j > a_{j-1}
    \end{cases}
    \\\Rightarrow
        a_{j-1}<x_k\le a_j
\end{align*}

Вспомним формулу \eqref{eq:cdfn_difference}
\begin{align*}
    \cdfn{a_j}&-\cdfn{a_{j-1}}=\\
    &=\frac{1}{n}\cdot \sum_{k=1}^n
    \left[\indicator{x_k\le a_j}-\indicator{x_k\le a_{j-1}}\right]
\end{align*}

Очевидно, что нас интересуют те пары, разность которых не равна нулю.
Это значит, что те случаи, когда $x>a_j$ или $x\le a_{j-1}$, нас не интересуют.
Поскольку такой случай, что $a_j<x\le a_{j-1}$ невозможен, то его тоже отбросим.
Значит, остался только тот вариант,
когда $x$ попадает в полуинтервал $\left(a_{j-1},a_j\right]$
$$\frac{1}{n}\cdot \sum_{k=1}^n
        \left[\indicator{x_k\le a_j}-\indicator{x_k\le a_{j-1}}\right]
    =\frac{1}{n}\cdot \sum_{k=1}^n \indicator{x_k\in \left(a_{j-1},a_j\right]}
$$

Видим знакомые полуинтервалы $\left(a_{j-1},a_j\right]=I_j$. Воспользуемся этим
$$\frac{1}{n}\cdot \sum_{k=1}^n \indicator{x_k\in \left(a_{j-1},a_j\right]}
=\frac{1}{n}\cdot \sum_{k=1}^n \indicator{x_k\in I_j}$$

Получаем компактную запись для разности функций распределения
\begin{equation}\label{eq:cdfn_difference_final}
\cdfn{a_j}-\cdfn{a_{j-1}}
=\frac{1}{n}\cdot \sum_{k=1}^n \indicator{x_k\in I_j}
\end{equation}


Вернёмся к уравнению \eqref{eq:histogram_start}
$$
q_n\left(y\right)
=\sum_{j=1}^m \frac{\cdfn{a_j}-\cdfn{a_{j-1}}}{a_j-a_{j-1}}
    \cdot\indicator{y\in I_j}
    $$

Воспользовавшись тем,
что $\left(a_j-a_{j-1}\right)$ --- длина полуинтервала $I_j$,
а разность $\cdfn{a_j}-\cdfn{a_{j-1}}$ была только что переписана
через индикаторы, получаем такую формулу
\begin{equation}\label{eq:histogramPreFinal}
    q_n\left(y\right)
        =\sum_{j=1}^m\frac{1}{n}\sum_{k=1}^n
            \indicator{x_k\in I_j}\cdot\frac{1}{\left|I_j\right|}
            \cdot \indicator{y \in I_j}
\end{equation}

\begin{definition}[Гистограмма]\index{гистограмма}
    Гистограммой выборки $x_1, \dots, x_n$ называется функция
    $$q_n\left( y \right)
    = \frac{1}{n} \cdot \sum_{j=1}^{m} \left\{
        \frac{\indicator{y \in I_j}}{\left| I_j \right|}
        \cdot \sum_{k=1}^{n} \indicator{x_k\in I_j} \right\}$$

    Где $I_j$ --- полуинтервалы ненулевой длины, покрывающие весь промежуток
    числовой оси, на который попадают числа из выборки
    $$a_0 < \min\left( X \right) \le \max\left( X \right) \le a_m,\;
        I_j = \left( a_{j-1}, a_j \right], j = \overline{1, m}$$
\end{definition}

Упростим формулу \eqref{eq:histogramPreFinal}, введя функцию
$\nu_j\left(X\right)$ \cite[стр.~68]{BorovkovMS},
которая считает количество элементов выборки $X=x_1, \dots, x_n$,
попавших в интервал $I_j$.
Это будет сумма индикаторов того, что элемент $x_k$ попал в $I_j$

$$\nu_j\left(X\right)
=\sum_{x\in X} \indicator{x\in I_j}
=\sum_{k=1}^n \indicator{x_k\in I_j}$$

Поскольку $\indicator{y\in I_j}$ зависит от $j$ и не зависит от $k$,
то его можно перенести во внешнюю сумму. Получаем следующую формулу
$$q_n\left(y\right)
=\sum_{j=1}^m \frac{\nu_j\left(X\right)}{n\cdot\left|I_j\right|}
    \cdot \indicator{y\in I_j}$$

У этой суммы только один ненулевой элемент,
так как $y$ может попасть только в один полуинтервал.
Тогда обозначим номер отрезка, в который попал $y$, через $k$ ($y\in I_k$),
а функцию $q_n\left(y\right)$ запишем как $q_n^k$
\begin{equation}\label{eq:histogram_borovkov}
    q_n^k = \frac{\nu_k\left(X\right)}{n\cdot\left|I_k\right|}
\end{equation}


Что мы тут видим?

Теперь $k$ --- номер столбика гистограммы. В математическом смысле это
номер интересующего нас полуинтервала --- того, в который попал $y$.

Высота столбика --- значение функции на определённом полуинтервале ---
пропорциональна количеству элементов, попавших в этот отрезок, что логично.

Делителю $\left|I_k\right|$ отведена особая роль --- он предотвращает
искажение гистограммы, когда длины отрезков разные; когда они одинаковые,
можно вынести длину как нормирующий множитель. То есть, чем длиннее отрезок,
тем ниже столбик, так как элементы более размазаны по отрезку, что тоже логично.

Представим, что значение функции --- это высота прямоугольника,
а длина отрезка --- его ширина (графически это изображается именно так).
Тогда отношение количества элементов, попавших в полуинтервал,
к количеству всех элементов выборки (вероятность того, что случайно взятый
элемент из выборки попадёт в $k$-ый отрезок \cite[стр.~24]{BorovkovMS}),
является площадью прямоугольника. Воспользовавшись формулой
\eqref{eq:histogram_borovkov}, получаем равенство
$$S_k
    = q_n^k \cdot \left| I_k \right|
    = \frac{\nu_k\left(X\right)}{n}=\probabilityn{x\in I_k}$$

Если устремить количество полуинтервалов к бесконечности ($m\to\infty$),
то каждый полуинтервал будет сжиматься в точку.
При этом вероятность попадания $x$ в отрезок будет стремиться
к вероятности попадания $x$ в точку $y$.
Введём обозначения $|I_j|=\delta$, $I_j=\Delta_y$
$$\probabilityn{x=y}
\approx\probabilityn{x\in\Delta_y}=q_n\left(y\right)\cdot\delta,
\qquad m\to\infty$$

Очень напоминает ситуацию с плотностью распределения
непрерывной случайной величины $\xi$
$$\probability{\xi=x}\approx\pdf{x}\cdot\delta,\qquad\delta\to 0$$

Нужно отметить, что количество элементов выборки
должно стремиться к бесконечности ($n\to\infty$),
так как плотность может быть лишь у непрерывных случайных величин.
Чем больше будет элементов,
тем плотнее они будут стоять на числовой прямой.

\subsection{Оценка неизвестных параметров}
Снова у нас есть $x_1, \dots, x_n$ --- выборка из распределения $F_\theta$,
\index{неизвестный параметр}
где $\theta$ --- неизвестный параметр из множества $\Theta$.

\begin{example}
    Имеем нормальное распределение с известной дисперсией $\sigma^2 = 1$
    и неизвестным математическим ожиданием $a$ --- $N\left(a,1\right)$.
    Тогда $\theta$ --- математическое ожидание $a$, а множество неизвестных
    параметров будет множеством действительных чисел $\Theta = \mathbb{R}$.
\end{example}
\begin{example}
    Есть нормальное распределение, в котором неизвестны оба параметра.
    Тогда $\theta$ будет парой $(a,\sigma)$, а $\Theta$ будет плоскостью
    (декартовым произведением) $\mathbb{R}^2$.
\end{example}

Главный вопрос --- определение основных параметров распределения выборки.

\index{статистика}
\begin{definition}[Статистика]
    \label{def:statistic}
    Статистикой называют функцию $S$ от выборки
    $X=\left(x_1,x_2,\dots,x_n\right)$
    \begin{align*}
        S\left(X\right) = S\left(x_1, x_2, \dots, x_n\right)
    \end{align*}
\end{definition}
\index{оценка}
\begin{definition}[Оценка]Статистику,
    значение которой заменяет неизвестный параметр,
    называют оценкой
\end{definition}
\begin{example}\label{example:bernulliEstimator}
    Возьмём выборку из распределения из распределения Бернулли,
    то есть $\left\{x_i\right\}$ --- набор одинаково распределённых
    случайных величин, причём
    \begin{align*}
    x_i=
    \begin{cases}
        1,&p\\
        0,&1-p
    \end{cases}
    \end{align*}

    Тогда неизвестным параметром будет величина $p$
    (вероятность удачного эксперимента)
    $$\theta=p\in\left[0;1\right]=\Theta$$

    Введём разные оценки $\hat{p}$
    \begin{align*}
        \hat{p}_1&=\frac{1}{n}\sum_{k=1}^n x_k\\
        \hat{p}_2&=x_1\\
        \hat{p}_3&=
            \frac{2}{n}\sum_{k=1}^{\left\lfloor \frac{n}{2} \right\rfloor} x_k
    \end{align*}
\end{example}
Замечание:
Поскольку $\hat{p}$ --- случайная величина, то может оказаться,
что она не равна настоящему параметру $p$
$$\Probability{\hat{p}=p}=0$$
\begin{enumerate}
    \item Возникает мысль о том, что разность $\hat{p}-p$
        должна быть ``маленькой''. Например, чтобы
        $\mean{\left(\hat{p}-p\right)}^2$ было самое маленькое из возможных.
    \item Также логично желать того,
        чтобы оценка $\hat{p}$ сходилась к истинному значению параметра $p$
        по вероятности ($\hat{p}\pcovergence p$)
        или почти всюду ($\hat{p}\acovergence p$)
    \item При многократном повторении эксперимента
        даже самая (на первый взгляд) плохая оценка может оказаться полезной
        \begin{align*}
            \mean{\hat{p}_1}=p\\
            \mean{\hat{p}_2}=p\\
            \mean{\hat{p}_3}=p
        \end{align*}
        Например, если целый год каждый день дают набор чисел,
        а статистик считает значение параметра $p$ с помощью оценки $\hat{p_2}$,
        то в среднем за год у него получится величина, близкая к истинному $p$.
\end{enumerate}

\index{оценка!состоятельная}
\begin{definition}[Состоятельная оценка]
    Оценка $\hat{\theta}$ называется состоятельной,
    если стремится к истинному значению $\theta$ по вероятности
    $$\hat{\theta}\pcovergence\theta$$
\end{definition}
\index{оценка!сильно состоятельная}
\begin{definition}[Сильно состоятельная оценка]
    Оценка $\hat{\theta}$ называется сильно состоятельной,
    если стремится к истинному значению $\theta$ почти наверное
    $$\hat{\theta}\acovergence\theta$$
\end{definition}
\begin{example}
    Оценка $\hat{p}_1$ из примера \ref{example:bernulliEstimator}
    является сильно состоятельной.
\end{example}
\begin{definition}[Несмещённая оценка]
    \label{def:estimatorBias}
    \index{оценка!несмещённая}
    Оценка $\hat{\theta}$ несмещённая, если
    $$\forall\theta\in\Theta: \meanof{\theta}{\hat{\theta}}=\theta$$
\end{definition}
\begin{remark}Несмещённая оценка существует не всегда
\end{remark}
\begin{definition}Несмещённая оценка $\hat{\theta}\in K$
называется оптимальной\footnote{В учебнике Боровкова А. А.
``Математическая статистика'' оценка, удовлетворяющая этим условиям,
носит название \textbf{эффективная оценка} \cite[стр.~130]{BorovkovMS},
но у нас этот термин будет использоваться далее в другом смысле}
в классе квадратично интегрируемых оценок $K$ (интегрируемых с квадратом),
если для всякой другой несмещённой оценки $\tilde{\theta}\in K$
$$\dispersionof{\theta}{\hat{\theta}}\le\dispersionof{\theta}{\tilde{\theta}},
\qquad \forall\theta\in\Theta$$
или же
$$\meanof{\theta}{\left(\hat{\theta}-\theta\right)^2}
\le\meanof{\theta}{\left(\tilde{\theta}-\theta\right)^2},
\qquad \forall\theta\in\Theta$$
\end{definition}

\begin{example}Сравним $\hat{p}_1$ и $\hat{p}_3$
    \begin{align*}
    \dispersionof{p}{\hat{p}_1}
        &=\frac{1}{n^2}\cdot n\cdot p\cdot\left(1-p\right)
        =\frac{p\cdot\left(1-p\right)}{n}\\
    \dispersionof{p}{\hat{p}_3}
        &=\frac{2\cdot p\cdot\left(1-p\right)}{n}
    \end{align*}
\end{example}
\subsection{Выборочные оценки. Метод моментов}
\index{метод!моментов}
Как восстановить неизвестный параметр $\theta\in\Theta$
из функции распределения $\cdfof{\theta}{x}$?

Вспомним распределения и их параметры
\begin{enumerate}
    \item Нормальное распределение $N\left(a,\sigma^2\right)$.
        В нём параметр $a$ является средним,
        а параметр $\sigma^2$ --- дисперсией
    \item Пуассоновское распределение $Poi\left(\lambda\right)$.
        Тут параметр $\lambda$ является и средним, и дисперсией
    \item Экспоненциальное распределение $Exp\left(\lambda\right)$.
        $\frac{1}{\lambda}$ --- среднее,
        $\frac{1}{\lambda^2}$ --- дисперсия
\end{enumerate}

Как правило, неизвестный параметр $\theta$ можно искать следующим образом
$$\exists\varphi\in C\left(\mathbb{R}\right):
    \int\limits_{\mathbb{R}}\varphi\left(x\right)d\cdfof{\theta}{x}
        =g\left(\theta\right)$$

Значит, у нас есть уравнение для поиска оценки $\hat{\theta}$
при непрерывной и монотонной $g\left( \hat{\theta} \right)$
\begin{align*}
g\left(\hat{\theta}\right)
    =\int\limits_{\mathbb{R}}\varphi\left(x\right)d\cdfn{x}
\end{align*}

\begin{example} Если $\theta$ --- среднее, то $\varphi\left(x\right)=x$
$$\int\limits_{-\infty}^{+\infty}xd\cdfof{\theta}{x}
    =\theta=g\left(\theta\right)$$
\end{example}
\begin{theorem}[Оценка метода моментов]
    \index{теорема!оценка метода моментов}
    \index{метод!моментов!оценка}
    Есть такая непрерывная $\varphi\left( x \right)$, что функция
    $g\left( \hat{\theta} \right)$ монотонная, непрерывная и ограниченная
    \cite[с.~87]{BorovkovMS}
    \begin{equation}\label{eq:unknown_parameter}
        g\left(\hat{\theta}\right)
            =\int\limits_{\mathbb{R}}\varphi\left(x\right)d\cdfof{\theta}{x}
    \end{equation}

    Тогда оценка $\hat{\theta}$ существует и является сильно состоятельной.
\end{theorem}
\begin{proof}
    Подставим эмпирическую функцию выборки вместо неизвестной в
    \eqref{eq:unknown_parameter}
    $$g\left(\hat{\theta}\right)
            =\int\limits_{\mathbb{R}}\varphi\left(x\right)d\cdfn{x}$$

    Поскольку функция $g\left(\hat{\theta}\right)$ непрерывна и монотонна,
    то она имеет обратную функцию
    $g^{-1}:g^{-1}\left(g\left(\hat{\theta}\right)\right)=\hat{\theta}$.

    Применим обратную функцию к обеим частям уравнения
    $$\hat{\theta}
            =g^{-1}\left(
                \int\limits_{\mathbb{R}}\varphi\left(x\right)d\cdfn{x}\right)$$

    Поскольку выборочная функция распределения при достаточно большом объёме
    выборки равна неизвестной функции распределения почти всюду, то
    $$\integrall{\mathbb{R}}{d\cdfn{x}}{\varphi\left( x \right)}
        \acovergence
            \integrall{\mathbb{R}}{d\cdfof{\theta}{x}}{
                \varphi\left( x \right)}$$

    Функция $g^{-1}\left(x\right)$ непрерывна
    $$\hat{\theta}
        = g^{-1}\left( \integrall{\mathbb{R}}{
            d\cdfn{x}}{\varphi\left( x \right)} \right)
        \acovergence
            g^{-1}\left( \integrall{\mathbb{R}}{
                d\cdfof{\theta}{x}}{\varphi\left( x \right)} \right)
            = \theta$$
    Теорема доказана
    $$\hat{\theta}\acovergence\theta$$
\end{proof}
\index{выборочное среднее}
\begin{definition}[Выборочное среднее]
    Выборочное средние обозначается через $\overline{x}$
    и считается по следующей формуле
    $$\overline{x}=\int\limits_{\mathbb{R}}xd\cdfn{x}$$

    Поскольку все элементы выборки равновероятны,
    получаем математическое ожидание
    дискретной равномерно распределённой случайной величины,
    принимающей $n$ значений
    $$\overline{x}=\int\limits_{\mathbb{R}}xd\cdfn{x}
        =\frac{1}{n}\cdot\sum_{k=1}^n x_k$$
\end{definition}
\index{выборочная дисперсия}
\begin{definition}[Выборочная дисперсия]
    Выборочная дисперсия $\overline{\sigma^2}$
    считается формуле
    $$\overline{\sigma^2}
        =\frac{1}{n-1}\cdot\sum_{k=1}^n \left(x_k-\overline{x}\right)^2$$
\end{definition}
\section{Свойства оценок}
\subsection{Неравенство Рао-Крамера}
\begin{theorem}[Колмогорова (теорема единственности)]
    \index{теорема!Колмогорова}
    \index{теорема!единственности}
    \label{theorem:Kolmogorov}
    Оптимальная оценка единственная или её нет вообще
\end{theorem}
\begin{proof}
    Допустим,
    есть две разные оптимальные и несмещённые оценки $\theta_1$ и $\theta_2$.
    Тогда по определению для любой несмещённой оценки $\hat{\theta}$ будет
    \begin{align*}
        \begin{cases}
            \dispersionof{\theta}{\theta_1}
                \le\dispersionof{\theta}{\hat{\theta}}\\
            \dispersionof{\theta}{\theta_2}
                \le\dispersionof{\theta}{\hat{\theta}}
        \end{cases}
        ,\forall\theta\in\Theta
    \end{align*}

    Поскольку неравенство выполняется
    для каждой несмещённой оценки $\hat{\theta}$,
    а оценки $\theta_1$ и $\theta_2$ являются несмещёнными,
    то можем их и поставить в неравенство в роли $\hat{\theta}$

    \begin{align*}
        \begin{cases}
            \dispersionof{\theta}{\theta_1}
                \le\dispersionof{\theta}{\theta_2}\\
            \dispersionof{\theta}{\theta_2}
                \le\dispersionof{\theta}{\theta_1}
        \end{cases}
        ,\forall\theta\in\Theta
    \end{align*}

    А это возможно только если дисперсии этих оценок равны.
    Обозначим эту дисперсию через $\sigma^2\left(\theta\right)$

    $$\dispersionof{\theta}{\theta_1}
        =\dispersionof{\theta}{\theta_2}
        =\sigma^2\left(\theta\right)$$

    Возьмём несмещённую оценку $\tilde{\theta}$,
    равную среднеарифметическому оценок $\theta_1$ и $\theta_2$
    $$\tilde{\theta}=\frac{1}{2}\cdot\theta_1+\frac{1}{2}\cdot\theta_2$$

    Тогда по определению $\theta_1$ и $\theta_2$ получаем,
    что дисперсия новой оценки не меньше, чем у оптимальных
    \begin{equation}\label{eq:estimator_ge}
        \dispersionof{\theta}{\tilde{\theta}}\ge\sigma^2\left(\theta\right)
    \end{equation}

    Честно посчитаем дисперсию оценки $\tilde{\theta}$
    \begin{align*}
    \dispersionof{\theta}{\tilde{\theta}}
        &=\meanof{\theta}{\left( \tilde{\theta}-\theta \right)^2}
        =\meanof{\theta}{\left[ \frac{1}{2}\cdot\left( \theta_1-\theta \right)
            +\frac{1}{2}\cdot\left( \theta_2-\theta \right) \right]^2}=\\
        &=\frac{1}{4}\cdot\dispersionof{\theta}{\theta_1}
            +\frac{1}{4}\cdot\dispersionof{\theta}{\theta_1}
            +\frac{1}{2}\cdot\meanof{\theta}
                {\left[ \left( \theta_1-\theta \right)
                    \cdot\left( \theta_2-\theta \right) \right]}
    \end{align*}

    Воспользуемся неравенством Коши (частный случай неравенства Гёльдера)
    \begin{eqnarray}\label{eq:koshi_eq}
        \meanof{\theta}{\left[ \left( \theta_1-\theta \right)
            \cdot\left( \theta_2-\theta \right) \right]}
        \le\sqrt{\meanof{\theta}{\left( \theta_1-\theta \right)^2}
            \cdot\meanof{\theta}{\left( \theta_2-\theta \right)^2}}=\\
        =\sqrt{\dispersionof{\theta}{\theta_1}
            \cdot\dispersionof{\theta}{\theta_2}}
        =\sqrt{\sigma_1^2\cdot\sigma_2^2}\nonumber
    \end{eqnarray}
    
    И вернёмся к вычислению дисперсии оценки $\tilde{\theta}$
    \begin{align*}
        \frac{1}{4}\cdot\dispersionof{\theta}{\theta_1}
            +\frac{1}{4}\cdot\dispersionof{\theta}{\theta_1}
            +\frac{1}{2}\cdot\meanof{\theta}
                {\left[ \left( \theta_1-\theta \right)
                    \cdot\left( \theta_2-\theta \right) \right]}\le\\
        \le\frac{1}{2}\cdot\sigma^2\left( \theta \right)
            +\frac{1}{2}\cdot\sqrt{\sigma^2\left( \theta \right)
                \cdot\sigma^2\left( \theta \right)}
        =\sigma^2\left( \theta \right)
    \end{align*}

    То есть, дисперсия оценки $\tilde{\theta}$ не больше дисперсии
    введённой оптимальной оценки
    \begin{equation}\label{eq:estimator_le}
        \dispersionof{\theta}{\tilde{\theta}}\le\sigma^2\left(\theta\right)
    \end{equation}

    Воспользовавшись неравенствами
    \eqref{eq:estimator_ge} и \eqref{eq:estimator_le}, получаем равенство
    $$\dispersionof{\theta}{\tilde{\theta}}=\sigma^2\left(\theta\right)$$

    Это значит, что в неравенстве \eqref{eq:koshi_eq}
    в данном случае тоже выходит равенство
    $$\meanof{\theta}{\left[ \left( \theta_1-\theta \right)
            \cdot\left( \theta_2-\theta \right) \right]}
        =\sqrt{\meanof{\theta}{\left( \theta_1-\theta \right)^2}}
            \cdot\sqrt{\meanof{\theta}{\left( \theta_2-\theta \right)^2}}$$

    Для дальнейших размышлений вспомним аналогию с векторами,
    а именно смысл равенства в неравенстве Коши
    для скалярного произведения векторов
    $$\vec{a}\cdot\vec{b}
        =\left|\vec{a}\right|\cdot\left|\vec{b}\right|
            \cdot\cos{\left(\widehat{\vec{a},\vec{b}}\right)}
        =\sqrt{\vec{a}^2}\cdot\sqrt{\vec{b}^2}
            \cdot\cos{\left(\widehat{\vec{a},\vec{b}}\right)}$$

    Скалярное произведение двух векторов равно произведению их модулей
    только тогда, когда они сонаправлены
    $$\left(\widehat{\vec{a},\vec{b}}\right)=0
        \Rightarrow \vec{a}\cdot\vec{b}
        =\sqrt{\vec{a}^2}\cdot\sqrt{\vec{b}^2}$$

    Положим математическое ожидание нормой,
    а $\theta_1-\theta$ и $\theta_2-\theta$ векторами
    пространства случайных событий.
    Получаем, что нормы и направления этих векторов совпадают
    \begin{align*}
        &\meanof{\theta}{\left[ \left( \theta_1-\theta \right)
            \cdot\left( \theta_2-\theta \right) \right]}
        =\sqrt{\meanof{\theta}{\left( \theta_1-\theta \right)^2}}
            \cdot\sqrt{\meanof{\theta}{\left( \theta_2-\theta \right)^2}}\\
        &\Rightarrow\left(\widehat{\theta_1-\theta,\theta_2-\theta}\right)\\
        \end{align*}

    Это значит, что они равны,
    что противоречит предположению о том, что они разные
    \begin{align*}
        \begin{cases}
            \left(\widehat{\theta_1-\theta,\theta_2-\theta}\right)=0\\
            \meanof{\theta}{\left( \theta_1-\theta \right)^2}
                =\meanof{\theta}{\left( \theta_2-\theta \right)^2}
        \end{cases}
        \Rightarrow \theta_1-\theta=\theta_2-\theta\\
        \Rightarrow\theta_1=\theta_2
    \end{align*}

    Теорема доказана
\end{proof}

\begin{remark}\label{remark:doubleDiff}
    Для дальнейших действий будем считать, что функция распределения
    $\cdfof{\theta}{x}$ имеет плотность $\pdf{x,\theta}$,
    которая дважды дифференцируема по $\theta$.
    То есть, её можно дифференцировать под знаком интеграла.\footnote{Подробнее
    с дифференцированием под знаком интеграла Римана можно почитать во втором
    томе курса дифференциального и интегрального исчисления Фихтенгольца
    \cite[с.~712]{Fichtenholz2}.

    Для тех, кто интересуется интегралом Лебега, прямой путь в книгу
    Дороговцева по общей теории меры и интеграла \cite[с.~102]{DorogovtsevIT}}

    Грубо говоря, проходит такой трюк
    $$\frac{\partial}{\partial \theta}\integrall{\Delta}{dx}{\pdf{x, \theta}}
        = \integrall{\Delta}{dx}{
            \frac{\partial}{\partial \theta}\pdf{x, \theta}}$$
\end{remark}

Также отметим, что выборка $\left( x_1, \dots, x_n \right)$
имеет плотность распределения,
так как является случайным вектором в $\mathbb{R}^n$,
все компоненты которого --- случайные величины.

\begin{definition}[Функция правдоподобия]
    \label{def:likehoodFunction}
    \index{функция!правдоподобия}
    Плотность распределения вектора независимых случайных величин,
    равная произведению плотностей распределения его компонент,
    называется функцией правдоподобия
    $$L\left( \vec{x},\theta \right)=\prod_{k=1}^n\pdf{x_k,\theta}$$
\end{definition}

Прологарифмировав функцию правдоподобия, получим симпатичную сумму
$$\ln{L\left( \vec{x},\theta \right)}=\sum_{k=1}^n\ln{\pdf{x_k,\theta}}$$

А симпатична она тем,
что это сумма независимых одинаково распределённых случайных величин.
Воспользовавшись законом больших чисел, можем сказать,
что она стремится к сумме $n$ одинаковых математических ожиданий
при достаточно большом размере выборки
$$\ln{L\left( \vec{x},\theta \right)}
    = n\cdot\frac{\ln{\pdf{x_1,\theta}}+\dots+\ln{\pdf{x_n,\theta}}}{n}
    \approx n\cdot\meanof{\theta}{\ln{\pdf{x_1,\theta}}}$$

Проблема в том, что мы не знаем среднего.
Для разрешения этого вопроса введём ещё одно определение

\begin{definition}[Вклад выборки]\label{def:defU}
    \index{вклад выборки}
    Вклад выборки --- частная производная по параметру $\theta$
    от логарифма функции правдоподобия
    $$U\left( \vec{x},\theta \right)
            =\frac{\partial}{\partial\theta}\ln{L\left(\vec{x},\theta\right)}$$
\end{definition}

\begin{remark}\label{remark:defU}
    Рекомендуется запомнить ещё две записи вклада выборки, так как они нам
    дальше пригодятся.

    Первая:
    \begin{align*}
        U\left( \vec{x},\theta \right)
        = \sum_{k=1}^n\frac{\partial}{\partial\theta}\ln{\pdf{x_k,\theta}}
    \end{align*}

    Вторая:
    $$U\left( \vec{x},\theta \right)
        =\frac{\frac{\partial}{\partial\theta}L\left(\vec{x},\theta\right)}
            {L\left(\vec{x},\theta\right)}$$
\end{remark}
\begin{proof}[Откуда это взялось]
    Первая формула следует непосредственно из определения функции правдоподобия
    $$U\left( \vec{x},\theta \right)
            =\frac{\partial}{\partial\theta}\ln{L\left(\vec{x},\theta\right)}
            =\sum_{k=1}^n
                \frac{\partial}{\partial\theta}\cdot\ln{\pdf{x_k,\theta}}$$

    Чтобы получить вторую запись, нужно взять производную.
    Вспоминаем, как правильно дифференцировать сложные функции
    \cite[с.~226]{Fichtenholz1}, \cite[с.~133]{DorogovtsevMA}
    $$\frac{\partial}{\partial x} \ln{f\left( x \right)}
        = \frac{1}{f\left( x \right)}
            \cdot \frac{\partial}{\partial x} f\left( x \right)$$

    И считаем
    $$U\left( \vec{x},\theta \right)
        =\frac{\partial}{\partial\theta}\ln{L\left(\vec{x},\theta\right)}
        =\frac{\frac{\partial}{\partial\theta}L\left(\vec{x},\theta\right)}
            {L\left(\vec{x},\theta\right)}$$
\end{proof}

\begin{remark}\label{remark:expectationU}
     Математическое ожидание вклада выборки равно нулю
    $$\meanof{\theta}{U\left( \vec{x},\theta \right)}=0$$
\end{remark}
\begin{proof}
Посчитаем математическое ожидание вклада выборки

\begin{align*}
    \meanof{\theta}{U\left( \vec{x},\theta \right)}
        &=\integral{\mathbb{R}^n}{}{\vec{u}}
            {U\left( \vec{u},\theta \right)
                \cdot L\left( \vec{u},\theta \right)}=\\
        &=\integral{\mathbb{R}^n}{}{\vec{u}}
            {\frac{\frac{\partial}{\partial\theta}L\left(\vec{x},\theta\right)}
                {L\left(\vec{x},\theta\right)}
                \cdot L\left( \vec{u},\theta \right)}=\\
        &=\integral{\mathbb{R}^n}{}{\vec{u}}
            {\frac{\partial}{\partial\theta}
                L\left( \vec{u},\theta \right)}
\end{align*}

Воспользовавшись предположением о том,
что функция распределения дважды дифференцируема,
вынесем взятие производной за знак интеграла
\begin{align*}
    \meanof{\theta}{U\left( \vec{x},\theta \right)}
        &=\frac{\partial}{\partial\theta}
            \integral{\mathbb{R}^n}{}{\vec{u}}{L\left( \vec{u},\theta \right)}
\end{align*}

Поскольку интегрируем плотность распределения случайного вектора
по всему пространству, то он равен единице.
Производная же от единицы равна нулю.
Это значит, что математическое ожидание вклада выборки равно нулю
\begin{align*}
    \meanof{\theta}{U\left( \vec{x},\theta \right)}
        &=\frac{\partial}{\partial\theta}
            \integral{\mathbb{R}^n}{}{\vec{u}}{L\left( \vec{u},\theta \right)}
        =\frac{\partial}{\partial\theta}1
        =0
\end{align*}
\end{proof}
\begin{remark}\label{remark:partialLikelihoodNull}
    Частная производная по оценке $\theta$ от функции правдоподобия
    $L\left( \vec{u},\theta \right)$ равна нулю.
\end{remark}
\begin{proof}
    Выше у нас было равенство
    $$\frac{\partial}{\partial\theta}
        \integral{\mathbb{R}^n}{}{\vec{u}}{L\left( \vec{u},\theta \right)}=0$$

    Так как производную можем заносить под знак интеграла
    (согласно замечанию \ref{remark:doubleDiff}), то получаем такое равенство
    $$\integral{\mathbb{R}^n}{}{\vec{u}}{
        \frac{\partial}{\partial\theta}L\left( \vec{u},\theta \right)}=0$$

    Поскольку интеграл не зависит от $\theta$,
    то такое возможно лишь в том случае, когда производная равна нулю
    $$\frac{\partial}{\partial\theta}L\left( \vec{u},\theta \right)=0$$
\end{proof}
\begin{definition}[Количество информации Фишера]
    \label{def:fisherInformation}
    \index{количество информации Фишера}
    Математическое ожидание квадрата вклада выборки называется
    количеством информации Фишера
    $$I_n\left(\theta\right)=
        \meanof{\theta}{U\left( \vec{x},\theta \right)^2}$$
\end{definition}
\begin{remark}
    Между математическим ожиданием квадрата вклада выборки и второй производной
    функции правдоподобия существует такое соотношение
    $$\meanof{\theta}{U\left( \vec{x},\theta \right)^2}
        =-\meanof{\theta}{
            \frac{\partial^2}{\partial\theta^2}
            \ln{L\left( \vec{x},\theta \right)}}$$
\end{remark}
\begin{proof}
    Будем доказывать справа налево
    \begin{align*}
        -\meanof{\theta}{
            \frac{\partial^2}{\partial\theta^2}
            \ln{L\left( \vec{x},\theta \right)}}=
        -\meanof{\theta}{
            \frac{\partial}{\partial\theta}
            \frac{\frac{\partial}{\partial\theta}L\left(\vec{x},\theta\right)}
                {L\left(\vec{x},\theta\right)}}=\\
        =-\meanof{\theta}{
            \left(
            \frac{\frac{\partial^2}{\partial\theta^2}
                L\left(\vec{x},\theta\right)\cdot L\left(\vec{x},\theta\right)-
                    \left[\frac{\partial}{\partial\theta}
                        L\left(\vec{x},\theta\right)\right]^2
                }
                {L\left(\vec{x},\theta\right)^2}
                \right)}=\\
        =-\meanof{\theta}{
            \frac{\frac{\partial^2}{\partial\theta^2}
                L\left(\vec{x},\theta\right)}
                {L\left(\vec{x},\theta\right)}}
            +\meanof{\theta}{
                \left[\frac{\frac{\partial}{\partial\theta}
                    L\left(\vec{x},\theta\right)}
                    {L\left(\vec{x},\theta\right)}\right]^2}
    \end{align*}

    Получили такое равенство
    \begin{equation}\label{eq:ULProofStart}
        - \meanof{\theta}{\frac{\partial^2}{\partial\theta^2}
            \ln{L\left( \vec{x},\theta \right)}}
        = -\meanof{\theta}{
            \frac{\frac{\partial^2}{\partial\theta^2}
                L\left(\vec{x},\theta\right)}
                {L\left(\vec{x},\theta\right)}}
            +\meanof{\theta}{
                \left[\frac{\frac{\partial}{\partial\theta}
                    L\left(\vec{x},\theta\right)}
                    {L\left(\vec{x},\theta\right)}\right]^2}
    \end{equation}
    
    Помним, что производная от функции правдоподобия по $\theta$ равна нулю
    (замечание \ref{remark:partialLikelihoodNull}).
    Значит, вторая производная тоже равна нулю
    $$\frac{\partial}{\partial\theta}L\left( \vec{u},\theta \right) = 0
        \Rightarrow
            -\meanof{\theta}{
            \frac{\frac{\partial^2}{\partial\theta^2}
                L\left(\vec{x},\theta\right)}
                {L\left(\vec{x},\theta\right)}} = 0$$


    От равенства \eqref{eq:ULProofStart} осталось лишь это
    $$- \meanof{\theta}{\frac{\partial^2}{\partial\theta^2}
            \ln{L\left( \vec{x},\theta \right)}}
        = \meanof{\theta}{\left[\frac{\frac{\partial}{\partial\theta}
                L\left(\vec{x},\theta\right)}
                {L\left(\vec{x},\theta\right)}\right]^2}$$

    После недолгих преобразований, пользуясь лишь замечанием \ref{remark:defU},
    видим, что получается нужный нам результат
    \begin{align*}
        \meanof{\theta}{
            \left[\frac{\frac{\partial}{\partial\theta}
                L\left(\vec{x},\theta\right)}
                {L\left(\vec{x},\theta\right)}\right]^2}
        =\meanof{\theta}{\left[
            \frac{\partial}{\partial\theta}
            \ln{L\left( \vec{x},\theta \right)}\right]^2}
        =\meanof{\theta}{U\left( \vec{x},\theta \right)^2}
    \end{align*}

    Утверждение доказано
    $$\meanof{\theta}{U\left( \vec{x},\theta \right)^2}
        =-\meanof{\theta}{
            \frac{\partial^2}{\partial\theta^2}
            \ln{L\left( \vec{x},\theta \right)}}$$
\end{proof}

Количество информации позволяет оценить точность,
с которой можем получить параметр $\theta$

\begin{theorem}[Неравенство Рао-Крамера]
    \label{theorem:Rao-Kramer}
    \index{неравенство!Рао-Крамера}
    Пусть $\hat{\theta}$ --- несмещённая оценка параметра $\theta$.
    Тогда имеет место неравенство
    $$\forall\theta\in\Theta:
        \dispersionof{\theta}{\hat{\theta}}
        \ge\frac{1}{I_n\left(\theta\right)}$$
\end{theorem}
\begin{proof}
    Выпишем, чему равно математическое ожидание оценки $\hat{\theta}$
    $$\begin{cases}
        \meanof{\theta}{\hat{\theta}}
            &=\theta\\
        \meanof{\theta}{\hat{\theta}}
            &=\integral{\mathbb{R}^n}{}{\vec{u}}{
                \hat{\theta}\left( \vec{u} \right)
                    \cdot L\left( \vec{u},\theta \right)}
        \end{cases}
        \Rightarrow
        \theta=\integral{\mathbb{R}^n}{}{\vec{u}}{
                \hat{\theta}\left( \vec{u} \right)
                    \cdot L\left( \vec{u},\theta \right)}$$

    Продифференцируем с двух сторон полученное для $\theta$ равенство
    по самому параметру $\theta$
    $$\frac{\partial}{\partial \theta}\theta
        = \frac{\partial}{\partial \theta}\integral{\mathbb{R}^n}{}{\vec{u}}{
                \hat{\theta}\left( \vec{u} \right)
                    \cdot L\left( \vec{u},\theta \right)}$$

    Левая часть равенства превращается в единицу,
    а справа происходит дифференцирование под знаком интеграла.
    Также помним, что оценка $\theta\left( \vec{u} \right)$
    не зависит от параметра $\theta$.
    Это значит, что производную нужно брать только от функции правдоподобия
    $$1 = \integral{\mathbb{R}^n}{}{\vec{u}}{\hat{\theta}\left( \vec{u} \right)
        \cdot \frac{\partial}{\partial \theta}L\left( \vec{u},\theta \right)}$$

    Далее нам нужно получить вклад выборки.
    Для этого умножим и поделим подынтегральное выражение
    на функцию правдоподобия
    $$\integral{\mathbb{R}^n}{}{\vec{u}}{\hat{\theta}\left( \vec{u} \right)
        \cdot \frac{\partial}{\partial \theta}L\left( \vec{u},\theta \right)}
    = \integral{\mathbb{R}^n}{}{\vec{u}}{\hat{\theta}\left( \vec{u} \right)
        \cdot \frac{
            \frac{\partial}{\partial \theta}L\left( \vec{u},\theta \right)}
            {L\left( \vec{u},\theta \right)}
                \cdot L\left( \vec{u},\theta \right)}$$

    Видим, что дробь под интегралом --- производная логарифма
    функции правдоподобия, которая является вкладом выборки
    $$\integral{\mathbb{R}^n}{}{\vec{u}}{\hat{\theta}\left( \vec{u} \right)
        \cdot \frac{
            \frac{\partial}{\partial \theta}L\left( \vec{u},\theta \right)}
            {L\left( \vec{u},\theta \right)}
                \cdot L\left( \vec{u},\theta \right)}
        = \integral{\mathbb{R}^n}{}{\vec{u}}{\hat{\theta}\left( \vec{u} \right)
            \cdot
                U\left( \vec{x},\theta \right)
                    \cdot L\left( \vec{u},\theta \right)}$$

    У нас есть математическое ожидание произведения оценки и вклада выборки,
    которое равно единице
    \begin{equation}\label{eq:rao_kramer}
        1 = \Meanof{\theta}{\hat{\theta}\cdot U\left( \vec{x},\theta \right)}
    \end{equation}

    Помним, что математическое ожидание вклада выборки равно нулю.
    Значит, умножение его на константу ничего не меняет
    $$\meanof{\theta}{U\left( \vec{x},\theta \right)}=0
        \Rightarrow 
        \theta\cdot\meanof{\theta}{U\left( \vec{x},\theta \right)}
        = \Meanof{\theta}{\theta\cdot U\left(\vec{x},\theta\right)}
        = 0$$

    Воспользовавшись полученным результатом,
    вернёмся к равенству \eqref{eq:rao_kramer}.
    Отнимем от обеих частей ноль (то есть, полученное только что выражение)
    $$1 = \Meanof{\theta}{\hat{\theta}\cdot U\left( \vec{x},\theta \right)}
        - \Meanof{\theta}{\theta\cdot U\left(\vec{x},\theta\right)}$$

    Получаем компактное равенство
    $$1 = \Meanof{\theta}{\left(\hat{\theta}-\theta\right)
        \cdot U\left( \vec{x},\theta \right)}$$

    Воспользовавшись неравенством Коши, узнаём
    произведение корней дисперсии и количества информации больше, чем единица
    \begin{equation}\label{eq:rao_kramer_koshi}
        \begin{split}
        1 = \meanof{\theta}{
            \left[\left(\hat{\theta}-\theta\right)
                \cdot U\left( \vec{x},\theta \right)\right]} \le \\
        \le \sqrt{\meanof{\theta}{
            \left(\hat{\theta}-\theta\right)^2}}
            \cdot\sqrt{\meanof{\theta}{U\left( \vec{x},\theta \right)^2}}
        = \sqrt{\dispersionof{\theta}{\hat{\theta}}}
            \cdot\sqrt{I_n\left(\theta\right)}
        \end{split}
    \end{equation}

    Возводим обе части равенства в квадрат и делим на количество информации
    $$\dispersionof{\theta}{\hat{\theta}}\ge \frac{1}{I_n\left(\theta\right)}$$

    Неравенство доказано
\end{proof}
\begin{remark}
    Иногда нужно оценивать не сам параметр, а функцию параметра.
    Тогда, если $\alpha$ --- несмещённая оценка для $f\left(\theta\right)$,
    справедливо следующее неравенство
    $$\forall\theta\in\Theta:\;\dispersionof{\theta}{\alpha}
        \ge\frac{\left|f'\left(\theta\right)\right|}{I_n\left(\theta\right)}$$
\end{remark}

\subsection{Метод максимального правдоподобия}
У нас есть нижняя оценка точности,
с которой можно отыскать желаемую оценку, а это значит,
что точнее определить просто не получится
и нужно стремиться к равенству в неравенстве Рао-Крамера.

\begin{definition}[Эффективная оценка]\index{оценка!эффективная}
    Оценка $\hat{\theta}$,
    для которой в неравенстве Рао-Крамера стоит равенство,
    называется эффективной
    $$\forall\theta\in\Theta:
        \dispersionof{\theta}{\hat{\theta}}=\frac{1}{I_n\left(\theta\right)}$$
\end{definition}

Выясним, какими свойствами должна обладать плотность,
чтобы можно было получить эффективную оценку.
Для этого в неравенстве Рао-Крамера нужно рассмотреть случай равенства
(так как в этом случае оценка будет самой точной)
    $$\dispersionof{\theta}{\hat{\theta}}=\frac{1}{I_n\left(\theta\right)}$$

Рассмотрим неравенство \eqref{eq:rao_kramer_koshi} и выясним,
в каком случае в нём будет стоять знак равенства
    $$1 = \meanof{\theta}{
        \left[\left(\hat{\theta}-\theta\right)
            \cdot U\left( \vec{x},\theta \right)\right]}
        = \sqrt{\meanof{\theta}{
        \left(\hat{\theta}-\theta\right)}^2}
        \cdot\sqrt{\meanof{\theta}{U\left( \vec{x},\theta \right)^2}}$$

Снова проводим аналогию с векторами и видим,
что скалярное произведение (математическое ожидание произведения)
векторов
(функций от параметра $\theta$:
$f_1\left( \theta \right)=\hat{\theta}-\theta$ и
$f_2\left( \theta \right)=U\left( \vec{x},\theta \right)$)
равно произведению их норм (корней математических ожиданий квадратов).

Это в свою очередь означает,
что угол между этими векторами (функциями) равен нулю,
а сами функции являются линейными комбинациями друг друга.
Значит, есть такая функция $k\left( \theta \right)$, что
$f_2\left( \theta \right)$ равняется произведению
$f_1\left( \theta \right)$ и $k\left( \theta \right)$.
\begin{align*}
    U\left( \vec{x},\theta \right)
        &=\left( \hat{\theta}-\theta \right)\cdot k\left( \theta \right)\\
    \frac{\partial}{\partial\theta}\ln{L\left( \vec{x},\theta \right)}
        &=\hat{\theta}\cdot k\left( \theta \right)
            -\theta\cdot k\left( \theta \right)\\
    \partial\ln{L\left( \vec{x},\theta \right)}
        &=\hat{\theta}\left( \vec{x} \right)
                \cdot k\left( \theta \right)\cdot\partial\theta
            -\theta\cdot k\left( \theta \right)\cdot\partial\theta
\end{align*}

Проинтегрируем обе части равенства
\begin{align*}
    \integralp{}{}{\ln{L\left( \vec{x},\theta \right)}}{}
        &=\hat{\theta}\left( \vec{x} \right)
                \cdot \integralp{}{}{\theta}{k\left( \theta \right)}
            -\integralp{}{}{\theta}{\theta\cdot k\left( \theta \right)}
\end{align*}

Получим следующее равенство
\begin{align*}
    \ln{L\left( \vec{x},\theta \right)}+c_1\left( \vec{x} \right)
        =\hat{\theta}\left( \vec{x} \right)
                \cdot \left[ a\left( \theta \right)+c_2\right]
            -\left[b^*\left( \theta \right)+c_3\right]
\end{align*}

Сгруппируем константы и введём замену
$b\left( \theta \right)=-b^*\left( \theta \right)$
$$\ln{L\left( \vec{x},\theta \right)}
    =\hat{\theta}\left( \vec{x} \right)\cdot a\left( \theta \right)
        +b\left( \theta \right)+c\left( \vec{x} \right)$$

Избавимся от логарифма слева, а для этого возьмём экспоненту от обеих частей
равенства
$$L\left( \vec{x},\theta \right)
    =\exp{\left\{\hat{\theta}\left( \vec{x} \right)\cdot a\left( \theta \right)
    +b\left( \theta \right)+c\left( \vec{x} \right)\right\}}$$

При конечном $n$ положим такую плотность распределения
$$\pdf{x_1,\theta}
    =\exp{\left\{\hat{\theta}\left(x_1\right)\cdot a_1\left(\theta\right)
        +b_1\left( \theta \right)+c_1\left( x_1 \right)\right\}}$$

В таком случае получим следующую функцию правдоподобия
\begin{align*}
    L\left( \vec{x},\theta \right)
    =\prod_{k=1}^n\pdf{x_1,\theta}=\\
    =\exp{\left\{\sum_{k=1}^n \hat{\theta}\left(x_k\right)\cdot a_1\left(\theta\right)
        +n\cdot b_1\left( \theta \right)
        +\sum_{k=1}^n c_1\left( x_k \right)\right\}}
\end{align*}

Отметим, что в этом случае оценка $\hat{\theta}\left( \vec{x} \right)$
является суммой оценок по каждой координате (случайной величине)
$$\hat{\theta}\left(\vec{x}\right)=\sum_{k=1}^n \hat{\theta}\left(x_k\right)$$

\begin{definition}[Экспоненциальное распределение]
    \label{def:exponentialDistribution}
    \index{распределение!экспоненциальное}
    \index{экспоненциальное!распределение}
    Распределения следующего вида называются экспоненциальными
    $$\pdf{x,\theta}
        =\exp{\left\{\hat{\theta}\left(x\right)\cdot a\left(\theta\right)
            +b\left( \theta \right)+c\left( x \right)\right\}}$$
\end{definition}

Также подведём итог предыдущих размышлений.

\begin{affirmation}
    \label{affirmation:efficientEstimator:exponentialExsistance}
    \index{эффективная оценка}
    \index{экспоненциальное!распределение!эффективная оценка}
    \index{распределение!экспоненциальное!эффективная оценка}
    Для экспоненциальных распределений существует эффективная оценка.
\end{affirmation}

Попробуем найти рецепт выяснения эффективной оценки. Начнём с примера
\begin{example}
    Есть выборка $x_1, x_2, \dots, x_n$ из нормального распределения
    с неизвестным математическим ожиданием $N\left( \theta,1 \right)$.
    Тогда плотность распределения $k$-ой случайной величины будет следующей
    $$\pdf{x_k}
        =\frac{1}{\sqrt{2\cdot\pi}}
            \cdot exp{\left\{-\frac{\left(x_k-\theta\right)^2}{2}\right\}}$$

    Её логарифм, очевидно, имеет такой вид
    $$\ln{\pdf{x_k}}
        =\ln{\frac{1}{\sqrt{2\cdot\pi}}}
            -\frac{\left(x_k-\theta\right)^2}{2}$$

    Теперь выпишем логарифм функции правдоподобия
    \begin{align*}
        \ln{L\left( \vec{x},\theta \right)}
        &=\sum_{k=1}^n \ln{\pdf{x_k}}=\\
        &=\sum_{k=1}^n \ln{\frac{1}{\sqrt{2\cdot\pi}}}
            -\sum_{k=1}^n \frac{\left(x_k-\theta\right)^2}{2}=\\
        &=n\cdot\ln{\frac{1}{\sqrt{2\cdot\pi}}}
            -\sum_{k=1}^n \frac{\left(x_k-\theta\right)^2}{2}
    \end{align*}

    Раскроем скобки
    \begin{align*}
        \ln{L\left( \vec{x},\theta \right)}
        =n\cdot\ln{\frac{1}{\sqrt{2\cdot\pi}}}
            -\sum_{k=1}^n \frac{x_k^2}{2}
            +\sum_{k=1}^n x_k\cdot\theta
            -\frac{n\cdot\theta^2}{2}
    \end{align*}

    Воспользуемся формулой для несмещённой и эффективной оценки среднего
    \begin{align*}
        \sum_{k=1}^n x_k\cdot\theta
            =\left( \frac{1}{n}\cdot\sum_{k=1}^n x_k \right) \cdot\theta\cdot n
            =\overline{x}\cdot\theta\cdot n\\
        \Rightarrow\ln{L\left( \vec{x},\theta \right)}
        =n\cdot\ln{\frac{1}{\sqrt{2\cdot\pi}}}
            -\sum_{k=1}^n \frac{x_k^2}{2}
            +\overline{x}\cdot\theta\cdot n
            -\frac{n\cdot\theta^2}{2}
    \end{align*}

    Сгруппировав множители при $n$, получаем
    $$\ln{L\left( \vec{x},\theta \right)}
        =n\cdot\ln{\frac{1}{\sqrt{2\cdot\pi}}}
            -\sum_{k=1}^n \frac{x_k^2}{2}
            -n\cdot\frac{\theta^2
                -2\cdot\overline{x}\cdot\theta}{2}$$

    Добавим и вычтем в числителе дроби выборочное среднее
    $$\ln{L\left( \vec{x},\theta \right)}
        =n\cdot\ln{\frac{1}{\sqrt{2\cdot\pi}}}
            -\sum_{k=1}^n \frac{x_k^2}{2}
            -n\cdot\frac{\theta^2
                -2\cdot\overline{x}\cdot\theta
                +\left(\overline{x}^2-\overline{x}^2\right)}{2}$$

    Теперь в числителе очевиден квадрат разности
    $$\ln{L\left( \vec{x},\theta \right)}
        =n\cdot\ln{\frac{1}{\sqrt{2\cdot\pi}}}
            -\sum_{k=1}^n \frac{x_k^2}{2}
            +n\cdot\frac{\overline{x}^2}{2}
            -n\cdot\frac{\theta^2
                -2\cdot\overline{x}\cdot\theta
                +\overline{x}^2}{2}$$

    Сворачиваем квадрат разности,
    а выборочное среднее заносим под знак суммы
    $$\ln{L\left( \vec{x},\theta \right)}
        =n\cdot\ln{\frac{1}{\sqrt{2\cdot\pi}}}
            -\sum_{k=1}^n \frac{x_k^2-\overline{x}^2}{2}
            -n\cdot\frac{\left(\theta-\overline{x}\right)^2}{2}$$

    Видим, что последнее слагаемое не может быть положительным,
    так как это квадрат со знаком ``минус''.
    Когда оценка $\theta$ равна выборочному среднему (идеальный случай),
    то последнее слагаемое обращается в нуль, а сама функция правдоподобия
    в таком случае принимает максимальное значение.

    Делаем предположение о том, как находить наилучшую оценку
    $$\theta_*=\argmaxof{\ln{L\left( \vec{x},\theta \right)}}{\theta}$$

    Оказывается, именно так она и находится.
\end{example}

\begin{definition}[Оценка максимального правдоподобия]
    \label{def:maximumLikelihoodEstimation}
    \index{оценка!максимального правдоподобия}
    Оценка максимального правдоподобия
    $\theta_*$ --- такое значение параметра $\theta$,
    при котором функция правдоподобия достигает своего максимального значения
    $$\theta_*=\argmaxof{\ln{L\left( \vec{x},\theta \right)}}{\theta}$$
\end{definition}

\begin{remark}
    Оценок максимального правдоподобия может быть несколько,
    а может не существовать ни одной.
\end{remark}

\begin{definition}[Уравнение правдоподобия]\index{уравнение!правдоподобия}
    Уравнением правдоподобия называется равенство вида
    $$U\left( \vec{x},\theta \right)=0$$

    Или же
    $$\frac{\partial}{\partial\theta}\ln{L\left( \vec{x},\theta \right)}=0$$
\end{definition}

\begin{remark}
    В гладком случае оценку $\theta_*$ можно искать
    с помощью уравнения правдоподобия.
    Тем не менее, нужно помнить, что равенство первой производной нулю
    является лишь необходимым условием максимума,
    поэтому полученные результаты необходимо проверять.
\end{remark}

\begin{definition}[Вариационный ряд]\index{вариационный ряд}
    Вариационный ряд выборки $x_1, x_2, \dots, x_n$ --- значения выборки,
    упорядоченные в порядке неубывания
    $$x_{\left(1\right)} \le x_{\left(2\right)} \le \dots
        \le x_{\left(n\right)},\;
        x_{\left(1\right)}=\underset{k}\min{x_k},
        x_{\left(n\right)}=\underset{k}\max{x_k}$$
\end{definition}

\begin{theorem}
    \index{теорема!о состоятельности оценки максимального правдоподобия}
    \index{оценка!максимального правдоподобия!теорема о состоятельности}
    Если плотность $\pdf{x,\theta}$
    непрерывна и дифференцируема по параметру $\theta$,
    а производная не равна нулю
    $\frac{\partial}{\partial\theta}\pdf{x,\theta}\neq 0$,
    то оценка максимального правдоподобия состоятельна
\end{theorem}
