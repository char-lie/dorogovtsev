\chapter{Основы}
\section{Методы оценок характеристик распределения
    наблюдаемых случайных величин}
$x_1, \dots, x_n$ --- независимые одинаково распределённые случайные величины
с неизвестной функцией распределения $F$.

Цель --- найти $F$ или сказать что-то о её свойствах.

\begin{definition} Эмпирической (выборочной) функцией распределения,
    построенной по выборке $x_1, \dots, x_n$ называется функция
    $$F_n\left(x\right)=\frac{1}{n}\cdot \sum_{k=1}^n
    \indicator_{\left\{x_k\le x\right\}}$$
\end{definition}

$$F\left(x\right)=\Probability{x_1 \le x}
    =\mean{\indicator_{\left\{ x_1\le x \right\}}}$$

\begin{theorem}
    $$\probability{F_n\left(x\right)\dCovergence F\left(x\right)}=1$$
\end{theorem}

Как можно попытаться отследить плотность распределения?
Постараемся найти функцию распределения, а потом и плотность.

Допустим, $F$ имеет хорошую (непрерывную) плотность.
Тогда как из $F$ получить $p$?

Мы знаем, что $F'=p$, но это никому не нужно, так как $F_n'$ --- производная
ступенчатой функции, которая почти везде будет равна нулю.

Но также мы помним, что
$$\cdf{b}-\cdf{a}=\int\limits_a^b \pdf{x} dx$$

Тогда, положив $a=x$, взяв некую $\Delta$, и постановив $b=x+\Delta$,
получаем следующее
$$\cdf{x+\Delta}-\cdf{x}=\int\limits_x^{x+\Delta} \pdf{y} dy$$

Делим обе части на $\Delta$ и при достаточно малых его значениях получаем
$$\frac{1}{\Delta}\cdot\int\limits_x^{x+\Delta} \pdf{y} dy
=\frac{\cdf{x+\Delta}-\cdf{x}}{\Delta}
\approx\frac{d\cdf{x}}{dx}=\pdf{x}$$

Значит, можем заменить $\pdf{x}$ не производной, а такой разностью.
$$\pdf{x}\approx\frac{\cdf{x+\Delta}-\cdf{x}}{\Delta}$$

Возьмём выборку из $m$ случайных величин в порядке возрастания
$a_1, \dots, a_m$, обозначим отрезки $I_j=[a_{j-1},a_j]$
и введём функцию $q\left(y\right)$
$$q\left(y\right)
=\sum_{j=1}^m \frac{\cdf{a_j}-\cdf{a_{j-1}}}{a_j-a_{j-1}}
    \cdot\indicator_{I_j}\left( y\right)$$

Теперь введём последовательность функций $q_n\left(y\right)$ и видим,
что она сходится к $q_n\left(y\right)$ почти наверное
согласно закону больших чисел,
а та в свою очередь имеет сходимость порядка $\frac{1}{n}$
к плотности распределения $\pdf{y}$
$$
q_n\left(y\right)
=\sum_{j=1}^m \frac{\cdfn{a_j}-\cdfn{a_{j-1}}}{a_j-a_{j-1}}
    \cdot\indicator_{I_j}\left( y\right)
\acovergence q\left(y\right)\covergence{}\pdf{y}
$$

$q_n$ --- гистограмма. И вот конечная формула
$$q_n\left(y\right)=\sum_{j=1}^m\frac{1}{n}\sum_{k=1}^n\indicator_{\left\{ x_k\in I_j \right\}}\frac{1}{\left|I_j\right|}\cdot\indicator_{I_j}\left(y\right)$$

\section{Оценка неизвестных параметров}
Снова у нас есть $x_1, \dots, x_n$ --- выборка из распределения $F_\theta$,
где $\theta$ --- неизвестный параметр из множества $\Theta$

\begin{example}Нормальное распределение с известным СКО $\sigma=1$
    и неизвестным математическим ожиданием,
    тогда $\theta$ --- математическое ожидание
\end{example}
\begin{example}Нормальное распределение, в котором неизвестны оба параметра.
    Тогда $\theta$ будет парой $(a,\sigma)$
\end{example}

Главный вопрос --- определение основных параметров.

\begin{definition}Функцию от выборки,
    значение которой заменяет неизвестный параметр,
    назвают оценкой
\end{definition}
\begin{example}Предположим, что выборка сделана из распределения Бернулли,
    то есть $\left\{x_i\right\}$ --- набор одинаково распределённых
    случайных величин, причём
    \begin{align*}
    x_i=
    \begin{cases}
        1,&p\\
        0,&1-p
    \end{cases}
    \end{align*}

    Тогда неизвестный параметр --- величина $p$
    (вероятность удачного эксперимента)
    $$\theta=p\in\left[0;1\right]=\Theta$$
\end{example}
