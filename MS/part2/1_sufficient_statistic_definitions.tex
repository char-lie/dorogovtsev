\section{Оптимальная оценка}
\begin{example}\label{example:optimalEstimate}
  Пусть \xsample[2 \cdot n] --- выборка из распределения Бернулли
  \begin{equation*}
    \theta \in \left[ 0; 1 \right]:\qquad
    x_i = \begin{cases}
      1,& \theta \\
      0,& 1 - \theta
    \end{cases}
  \end{equation*}
  Введём оценку
  \begin{equation*}
    \hat{\theta}
    = \frac{2 \cdot x_1 + x_2 + 2 \cdot x_3 + x_4 + \dots
      + 2 \cdot x_{2 \cdot n - 1} + x_{2 \cdot n}}{3 \cdot n}
  \end{equation*}
  Оценка $\hat{\theta}$ состоятельная
  \begin{equation*}
    \hat{\theta}
    = \frac{2 \cdot \overline{x}}{3}
      + \frac{x_1 + x_3 + \dots + x_{2 \cdot n - 1}}{3 \cdot n}
  \end{equation*}
  По закону больших чисел видим, что
  \begin{equation*}
    \begin{cases}
      \frac{2 \cdot \overline{x}}{3} 
        \covergence{} \frac{2 \cdot \theta}{3} \\ \\ % TODO: workaround
      \frac{x_1 + x_3 + \dots + x_{2 \cdot n - 1}}{3 \cdot n}
        \covergence{} \frac{\theta}{3}
    \end{cases}
  \end{equation*}
  Значит,
  \begin{equation*}
    \hat{\theta}
    \covergence{} \frac{2}{3} \cdot \theta + \frac{1}{3} \cdot \theta
    = \theta
  \end{equation*}
  Также оценка $\hat{\theta}$ является несмещённой
  \begin{equation*}
    \mean{\hat{\theta}}
    = \frac{2 \cdot n \cdot \theta + n \cdot \theta}{3 \cdot n}
    = \theta
  \end{equation*}
  Хоть оценка $\hat{\theta}$ и обладает такими хорошими свойствами как
  состоятельность и несмещённость, она не оптимальная --- не лучшая.
\end{example}

\begin{exercise}
  Проверить, что дисперсия $\hat{\theta}$ больше, чем дисперсия
  $\theta_* = \overline{x}$
\end{exercise}

\begin{definition}[Симметризация]\index{симметризация}
  Симметризация $\Lambda$ оценки $\hat{\theta}$ --- среднее
  оценок $\hat{\theta}$ для
  всевозможных перестановок $\sigma\in S_n$
  элементов выборки \xsample
  $$\Lambda\hat{\theta}
      = \frac{1}{n!}\cdot \sum_{\sigma\in S_n} \hat{\theta}\left(
      x_{\sigma\left(1\right)}, x_{\sigma\left(2\right)},
          \dots, x_{\sigma\left(n\right)}\right)$$
\end{definition}
\begin{lemma}
  \index{лемма!о симметризации несмещённой оценки}
  Для произвольной несмещённой оценки $\hat{\theta}$
  её симметризация $\Lambda{\hat{\theta}}$
  не хуже её самой в среднем квадратическом
  \begin{align*}
  \meanof{\theta}{\hat{\theta}}
      = \theta
  \Rightarrow
      \begin{cases}
      \meanof{\theta}{\Lambda{\hat{\theta}}}
          = \meanof{\theta}{\hat{\theta}}
          = \theta\\
      \dispersionof{\theta}{\Lambda{\hat{\theta}}}
          \le\dispersionof{\theta}{\hat{\theta}}
      \end{cases}
  \end{align*}
\end{lemma}
\begin{proof}
  Как и раньше, обозначим $\vec{x} = \left[ x_1, x_2, \dots, x_n \right]$.
  Для перестановки $\sigma \in S_n$ используем
  $\vec{x}_\sigma = \left[ x_{\sigma\left(1\right)}, x_{\sigma\left(2\right)},
  \dots, x_{\sigma\left(n\right)} \right]$
  Тогда
  \begin{equation*}
    \Lambda\hat{\theta}
    = \frac{1}{n!} \cdot \sum_{\sigma \in S_n}
      \hat{\theta}\left( \vec{x}_{\sigma} \right)
  \end{equation*}

  \begin{enumerate}
    \item
      Сначала докажем несмещённость $\Lambda\hat{\theta}$.

      Нетрудно показать, что  вектора $\vec{x}$ и $\vec{x}_\sigma$
      имеют одинаковое распределение для любой перестановки $\sigma$,
      а это значит, что и оценки $\hat{\theta}\left(\vec{x}\right)$
      и $\hat{\theta}\left(\vec{x}_\sigma\right)$
      распределены одинаково как результаты применения оодной и той же
      функции к одинаково распределённым случайным векторам.
      Следовательно, их математические ожидания равны между собой
      при любой перестановке $\sigma$
      \begin{equation*}
        \meanof{\theta}{\hat{\theta}\left(\vec{x}\right)}
        = \meanof{\theta}{\hat{\theta}\left(\vec{x}_\sigma\right)}
        = \theta
      \end{equation*}

      Посчитаем математическое ожидание симметризации оценки $\hat{\theta}$
      \begin{align*}
        \meanof{\theta}{\Lambda\hat{\theta}}
        = \meanof{\theta}{\left\{\frac{1}{n!}\cdot \sum_{\sigma\in S_n}
          \hat{\theta}\left( \vec{x}_{\sigma} \right)\right\}}
        = \frac{1}{n!} \cdot \sum_{\sigma \in S_n}
          \meanof{\theta}{\hat{\theta}
            \left( \vec{x}_{\sigma} \right)}
        = \frac{1}{n!} \cdot \sum_{\sigma \in S_n} \theta
        = \theta
      \end{align*}
      \begin{comment}
      Помним, что математическое ожидание линейно и
      константы можно выносить за знак математического ожидания,
      а математическое ожидание суммы равно сумме математических ожиданий
      \begin{align*}
          \meanof{\theta}{\left\{\frac{1}{n!}\cdot \sum_{\sigma\in S_n}
        \hat{\theta}\left(\vec{x}_\sigma\right)\right\}}
        = \frac{1}{n!}\cdot \sum_{\sigma\in S_n}
            \meanof{\theta}{\hat{\theta}\left(\vec{x}_\sigma\right)}
      \end{align*}

      Не забываем, что математическое ожидание
      $\hat{\theta}\left( \vec{x}_{\sigma} \right)$ равно параметру
      $\theta$

      \begin{align*}
          \frac{1}{n!}\cdot \sum_{\sigma\in S_n}
        \meanof{\theta}{\hat{\theta}\left(\vec{x}_\sigma\right)}
        = \frac{1}{n!}\cdot \sum_{\sigma\in S_n}\theta
      \end{align*}

      Сумма имеет $n!$ слагаемых (количество перестановок $\sigma\in S_n$)
      \begin{align*}
          \frac{1}{n!}\cdot \sum_{\sigma\in S_n}\theta
        = \frac{1}{n!}\cdot n!\cdot \theta
        = \theta
      \end{align*}

      А это значит, что первый пункт доказан и симметризация
      несмещённой оценки $\hat{\theta}$ действительно несмещённая
      $$\meanof{\theta}{\Lambda\hat{\theta}}= \theta$$
      \end{comment}
    \item
      Теперь посмотрим, чему равна дисперсия симметризации
      оценки $\hat{\theta}$

      Воспользуемся определением
      \begin{align*}
          \dispersionof{\theta}{\Lambda\hat{\theta}}
        = \meanof{\theta}{
            \left(\Lambda\hat{\theta}-\theta\right)^2}
        = \meanof{\theta}{
            \left\{\frac{1}{n!}\cdot \sum_{\sigma\in S_n}
        \hat{\theta}\left(\vec{x}_\sigma\right)
        -\theta\right\}^2}
      \end{align*}

      Внесём параметр $\theta$ в сумму
      \begin{align*}
        &\meanof{\theta}{\left\{ \frac{1}{n!} \cdot \sum_{\sigma\in S_n}
          \hat{\theta}\left( \vec{x}_\sigma \right) - \theta \right\}^2}
        = \meanof{\theta}{
          \left\{ \frac{1}{n!}\cdot \sum_{\sigma\in S_n}
          \left( \hat{\theta}\left( \vec{x}_\sigma \right)
            - \theta \right) \right\}^2} = \\
        &= \meanof{\theta}{
          \left\{ \sum_{\sigma\in S_n}\frac{1}{n!}
            \cdot \left( \hat{\theta}\left(\vec{x}_\sigma \right)
              -\theta \right) \right\}^2}
      \end{align*}

      Воспользуемся неравенством Коши-Буняковского.
      Помним, что в $S_n$ находится $n!$ перестановок
      \begin{equation*}
        \begin{split}
          \left\{ \sum_{\sigma\in S_n} \frac{1}{n!}
            \cdot \left( \hat{\theta}\left( \vec{x}_{\sigma} \right)
              -\theta \right) \right\}^2
          & \le \sum_{\sigma \in S_n} \left( \frac{1}{n!} \right)^2
            \cdot \sum_{\sigma \in S_n} \left\{
              \hat{\theta}\left( \vec{x}_{\sigma} \right) - \theta \right\}^2 \\
          & = \frac{1}{n!} \cdot \sum_{\sigma\in S_n}
              \left( \hat{\theta}\left( \vec{x}_\sigma \right) -\theta \right)^2
        \end{split}
      \end{equation*}

      Тогда
      \begin{equation}\label{eq:meanSymmetrization}
          \meanof{\theta}{\left\{\sum_{\sigma\in S_n}\frac{1}{n!}\cdot
        \left(\hat{\theta}\left(\vec{x}_\sigma\right)
        -\theta\right)\right\}^2}
        \le \frac{1}{n!} \cdot \sum_{\sigma\in S_n} \meanof{\theta}{
            \left( \hat{\theta}\left( \vec{x}_\sigma \right)
            -\theta \right)^2}
      \end{equation}

      Видим сумму дисперсий.
      Дисперсии одинаковы, так как оценки имеют одинаковые распределения
      \begin{align*}
        &\frac{1}{n!}\cdot \sum_{\sigma\in S_n}
          \meanof{\theta}{
            \left(\hat{\theta}\left(\vec{x}_\sigma\right)
            -\theta\right)^2}
        = \frac{1}{n!}\cdot \sum_{\sigma\in S_n}
          \dispersionof{\theta}{
            \hat{\theta}\left(\vec{x}_\sigma\right)}= \\
        &= \frac{1}{n!}\cdot \sum_{\sigma\in S_n}
          \dispersionof{\theta}{\hat{\theta}\left(\vec{x}\right)}
        = \frac{1}{n!}\cdot n!
          \cdot \dispersionof{\theta}{\hat{\theta}\left(\vec{x}\right)}
        = \dispersionof{\theta}{\hat{\theta}\left(\vec{x}\right)}
      \end{align*}

      Из неравенства \eqref{eq:meanSymmetrization} видим,
      что дисперсия симметризации не хуже дисперсии самой оценки
      \begin{equation*}
        \dispersionof{\theta}{\Lambda\hat{\theta}}
          \le \dispersionof{\theta}{\hat{\theta}\left({\vec{x}}\right)}
      \end{equation*}

  \end{enumerate}

  То есть симметризация не ухудшает оценку,
  а в общем случае (когда неравенство строгое) даже делает её лучше.
\end{proof}


\begin{comment}
\begin{definition}[Функция вариационного ряда]\index{функция!вариационного ряда}
  Если оценка $\hat{\theta}$ симметрична относительно перестановок аргументов,
  то она является функцией вариационного ряда
\end{definition}
\end{comment}

\begin{remark}
  В предыдущем примере \ref{example:optimalEstimate} выполняется равенство
  \begin{equation*}
    \Lambda \hat{\theta} = \overline{x}
  \end{equation*}
\end{remark}

\begin{remark}
  Любая оптимальная оценка является функцией вариационного ряда
\end{remark}

\section{$\sigma$-алгебра, порождённая случайной величиной}
Есть вероятностное пространство
$\left( \Omega, \mathfrak{F}, \mathbb{P} \right)$
и случайная величина $\xi$.

\begin{definition}[Сигма-алгебра, порождённая случайной величиной]
  \index{сигма-алгебра!порождённая!случайной величиной}
  $\mathfrak{F}_\xi = \sigma\left( \xi \right)$
  --- $\sigma$-алгебра, порождённая случайной величиной $\xi$
  $$\mathfrak{F}_\xi
      = \left\{ \xi^{-1}\left( \Delta \right)
      \mcond \Delta\in\mathfrak{B} \right\}$$
  $\mathfrak{B}$ --- борелевская $\sigma$-алгебра в $\mathbb{R}$.
\end{definition}

Из курса теории вероятностей помним лемму, которая утверждает,
что $\xi$ --- случайная величина тогда и только тогда, когда
\begin{equation*}
  \forall\Delta\in\mathfrak{B}:\qquad
  \left\{ \omega \mcond \xi\left( \omega \right) \in \Delta \right\}
  = \left\{ \xi\in\Delta \right\}
  = \xi^{-1}\left( \Delta \right) \in \mathfrak{F}
\end{equation*}

А это значит, что все элементы $\mathfrak{F}_\xi$ входят в $\sigma$-алгебру
$\mathfrak{F}$, а сама $\mathfrak{F}_\xi$ является подмножеством
$\mathfrak{F}$
\begin{align*}
  \begin{cases}
      \mathfrak{F}_\xi
      = \left\{ \xi^{-1}\left( \Delta \right)
          \mcond \Delta\in\mathfrak{B} \right\}\\
      \forall\Delta\in\mathfrak{B}:
      \xi^{-1}\left( \Delta \right) \in \mathfrak{F}
  \end{cases}
  \Rightarrow
  \mathfrak{F}_\xi \subset \mathfrak{F}
\end{align*}

Проверим, что $\mathfrak{F}_\xi$ действительно является $\sigma$-алгеброй.
\begin{enumerate}
  \item Множество элементарных исходов $\Omega$ входит в $\mathfrak{F}_\xi$.
      Поскольку случайная величина $\xi$ принимает действительные значения,
      то прообраз множества действительных чисел $\mathbb{R}$
      и будет множеством элементарных исходов $\Omega$.
      \begin{align*}
        \begin{cases}
          \xi^{-1}\left( \Delta \in \mathfrak{B} \right) \in \mathfrak{F} \\
          \mathbb{R} \in \mathfrak{B} \\
          \xi^{-1}\left( \mathbb{R} \right)= \Omega
        \end{cases}
        \Rightarrow
        \Omega \in \mathfrak{F}_\xi
      \end{align*}
  \item Если событие $A$ принадлежит $\mathfrak{F}_\xi$,
      то его дополнение $\stcomp{A}$ тоже принадлежит $\mathfrak{F}_\xi$
      \begin{align*}
        A
        = \xi^{-1}\left( \Delta \right)
        = \left\{ \omega \mcond \xi\left( \omega \right) \in \Delta \right\}
      \end{align*}
      Значит,
      \begin{align*}
        \stcomp{A}
        = \left\{ \omega \mcond \xi\left( \omega \right) \notin \Delta \right\}
        = \left\{ \omega \mcond \xi\left( \omega \right)
          \in \stcomp{\Delta}\right\}
        = \xi^{-1}\left( \stcomp{\Delta} \right)
      \end{align*}

      Поскольку $\mathfrak{B}$ является $\sigma$-алгеброй,
      а $\Delta$ --- её элемент,
      то дополнение $\stcomp{\Delta}$ тоже принадлежит
      $\sigma$-алгебре $\mathfrak{B}$.
      Из этого следует, что свойство выполняется
      \begin{align*}
      \begin{cases}
          \xi^{-1}\left( \Delta \right) \in \mathfrak{F}\\
          \Delta \in \mathfrak{B}
        \Rightarrow \stcomp{\Delta} \in \mathfrak{B}
      \end{cases}
      \Rightarrow
      \stcomp{\xi^{-1}\left( \Delta \right)}
          = \xi^{-1}\left( \stcomp{\Delta} \right) \in \mathfrak{F}
      \end{align*}
  \item Замкнутость относительно счётных пересечений.
    Пусть $A_n = \xi^{-1}\left( \Delta_n \right)$, $n \ge 1$,
    $\Delta_n \in \mathfrak{B}$.
    Тогда
    \begin{equation*}
      \bigcap_{n=1}^{\infty} A_n
      = \xi^{-1}\left( \bigcap_{n=1}^{\infty} \Delta_n \right).
    \end{equation*}
    Так как $\mathfrak{B}$ --- $\sigma$-алгебра, то
    \begin{equation*}
      \bigcap_{n=1}^{\infty} \Delta_n \in \mathfrak{B}.
    \end{equation*}
    Следовательно,
    \begin{equation*}
      \xi^{-1}\left( \bigcap_{n=1}^{\infty} \Delta_n \right)
      \in \mathfrak{F}_{\xi}
    \end{equation*}
\end{enumerate}

Как устроена эта $\sigma$-алгебра?
Каждому элементарному исходу отвечает одно и только одно значение
случайной величины, а каждому значению случайной величины
отвечает один и больше элементарных исходов.
Допустим, есть некое $a\in\mathbb{R}$, которое является образом по крайней мере
двух элементарных исходов $\omega_1$ и $\omega_2$
(рисунок \ref{fig:tikz:indistinguishableValuesImage})

$$\xi\left( \omega_1 \right) = \xi\left( \omega_2 \right) = a$$

\begin{figure}[h!]
  \center\includestandalone{tikz/indistinguishableValuesImage}
  %\center\includegraphics[width=\textwidth]{tikz/indistinguishableValuesImage.pdf}
  \caption{Множества уровней}
  \label{fig:tikz:indistinguishableValuesImage}
\end{figure}


Теперь рассмотрим элемент $\Delta$ борелевской $\sigma$-алгебры $\mathfrak{B}$.
Из вышесказанного следует, что,
если число $a$ принадлежит множеству $\Delta$, то прообраз этого множества
содержит элементы $\omega_1$ и $\omega_2$,
в противном случае оба элементарных исхода не входят в прообраз
\begin{align*}
  a \in \Delta
      \Rightarrow \xi^{-1}\left( \Delta \right) \ni \omega_1, \omega_2 \\
  a \notin \Delta
      \Rightarrow \xi^{-1}\left( \Delta \right) \not\ni \omega_1, \omega_2 \\
\end{align*}

То есть множество $\mathfrak{F}_\xi$ не будет различать
элементы $\omega_1$ и $\omega_2$.
Это в свою очередь означает, что можно разбить $\Omega$
на уровни --- непересекающиеся подмножества, не различимые с помощью
$\mathfrak{F}_{\xi}$.

\begin{definition}[Множество уровня]\index{множество уровня}
  Множество уровня $H_t$ --- полный прообраз
  значения $t\in\mathbb{R}$ случайной величины $\xi$
  $$H_t
      = \left\{ \omega \mcond \xi\left( \omega \right) = t \right\}
      = \xi^{-1}\left( t \right)$$
\end{definition}

\begin{remark}
  Уровни $H_i$ составляют разбиение множества элементарных исходов $\Omega$.
  \begin{enumerate}
      \item Множества $H_i$ не пересекаются
        \begin{equation*}
          H_{t_1} \cap H_{t_2} = \emptyset \Leftrightarrow t_1 \neq t_2 
        \end{equation*}
      \item Объединение всех $H_i$ даёт множество элементарных исходов
      $$\bigcup_{t \in \mathbb{R}} H_t
          = \bigcup_{t \in \mathbb{R}} \xi^{-1}\left( t \right)
          = \xi^{-1}\left( \mathbb{R} \right)
          = \Omega$$
  \end{enumerate}
\end{remark}

Очень похоже на гипотезы из курса теории вероятностей с той лишь разницей,
что уровней может быть бесконечное и даже континуальное количество,
из чего также следует, что вероятность некоторых из них может быть нулевой.

\section{Случайная величина, измеримая относительно $\sigma$-алгебры}
В общем случае вероятностное пространство может быть разбито
на континуальное количество множеств уровней
(для $\sigma$-алгебры, порождённой непрерывной случайной величиной).
Начнём же с рассмотрения того случая,
когда случайная величина $\xi$ принимает $n$ значений
$a_1, a_2, \dots, a_n$
$$\xi: \Omega \rightarrow \left\{ a_1, a_2, \dots, a_n \right\}$$
Это означает, что у нас есть $n$ уровней
$$H_k = \left\{ \omega \mcond \xi\left( \omega \right) = a_k \right\},
  k= \overline{1,n}$$
Нетрудно понять,
что $\sigma$-алгебра $\sigma\left( \xi \right)$ содержит $2^n$ элементов
$$\sigma\left( \xi \right) = \left\{ \bigcup_{k=1}^n H_k^{\eta_k}
  \mcond \eta_k = \overline{0,1}, H_k^0 = \emptyset, H_k^1 = H_k \right\}$$
Прежде чем продолжить, зафиксируем явление и дадим ему название.

\begin{definition}[Сигма-алгебра, порождённая полным набором гипотез]
  \index{сигма-алгебра!порождённая!полным набором гипотез}
  Возьмём набор множеств $H_1, \dots, H_n$ который является полным набором
  гипотез для пространства элементарных исходов $\Omega$
  $$\bigcap_{k=1}^n H_k = \emptyset,\; \bigcup_{k=1}^n H_k = \Omega,\;
      \Probability{H_k} \neq 0$$

  В таком случае $\sigma$-алгебра, содержащая всевозможные объединения этих
  множеств, будет называться $\sigma$-алгеброй, порождённой полным набором
  гипотез и будет выглядеть следующим образом
  $$\mathfrak{F}_1 = \left\{ \bigcup_{k=1}^n H_k^{\eta_k}
      \mcond \eta_k = \overline{0,1},
      H_k^0 = \emptyset, H_k^1 = H_k \right\}$$
\end{definition}

Нас интересует, как устроены случайные величины,
которые измеримы относительно $\sigma$-алгебры $\sigma\left( \xi \right)$.

\begin{definition}[Случайная величина, измеримая относительно сигма-алгебры]
  \index{случайная величина!измеримая относительно сигма-алгебры}
  Тот факт, что случайная величина $\varkappa$ измерима относительно
  $\mathfrak{F}_1$, значит, что
  \begin{align*}
      \forall c \in \mathbb{R}:\qquad
      \left\{ \omega \mcond \varkappa\left( \omega \right) \le c \right\}
      \in \mathfrak{F}_1
  \end{align*}
\end{definition}

Возьмём случайную величину $\kappa$, измеримую относительно $\sigma$-алгебры
$\sigma\left( \xi \right)$
\begin{align*}
  \left\{ \omega \mcond \varkappa\left( \omega \right) \le c \right\}
      \in \sigma\left( \xi \right)
\end{align*}
То есть прообразы $\varkappa$ выражаются через объединения уровней $H_k$
$$\left\{ \omega \mcond \varkappa\left( \omega \right) \le c \right\}
  = \bigcup_{k=1}^n H_k^{\eta_k}$$
Введём обозначение
$$A\left( c \right)
  = \left\{ \omega \mcond \varkappa\left( \omega \right) \le c \right\}$$
Очевидно, что при $c\to-\infty$ прообразом является пустое множество,
а когда $c\to+ \infty$, то прообразом является всё множество элементарных исходов
\begin{align*}
  \left\{ \omega \mcond \varkappa\left( \omega \right) \le -\infty \right\}
      &= \left\{ \omega \mcond \varkappa\left(\omega\right)\in\emptyset \right\}
      = \varkappa^{-1}\left( \emptyset \right)
      = \emptyset \\
  \left\{ \omega \mcond \varkappa\left( \omega \right) \le + \infty \right\}
      &= \left\{ \omega \mcond\varkappa\left(\omega\right)\in\mathbb{R} \right\}
      = \varkappa^{-1}\left( \mathbb{R} \right)
      = \Omega
\end{align*}
Также ясно, что если имеются два борелевских множества и одно
включёно в другое, то полный прообраз первого тоже будет включён в
прообраз второго.
\begin{comment}
\begin{align*}
  \Delta_1, \Delta_2 \in \mathfrak{B},
  \Delta_1 \subseteq \Delta_2 \\
  \Rightarrow \varkappa^{-1}\left( \Delta_1 \right)
      \subseteq \varkappa^{-1}\left( \Delta_1 \right)
      \cup \varkappa^{-1}\left( \Delta_2 \right) = \\
      = \varkappa^{-1}\left( \Delta_1 \cup \Delta_2 \right)
      = \varkappa^{-1}\left( \Delta_2 \right)
\end{align*}
\end{comment}
Справедливо соотношение
\begin{equation*}
  c_1, c_2 \in \mathbb{R}, c_1 \le c_2
  \Rightarrow A\left( c_1 \right) \subseteq A\left( c_2 \right)
\end{equation*}

Объединим и проанализируем вышеописанное:
\begin{enumerate}
  \item Количество элементов в множестве $A\left( c \right)$
      не уменьшается с ростом $c$
      $$c_1 \le c_2
      \Rightarrow A\left( c_1 \right) \subseteq A\left( c_2 \right)$$

  \item Множество $A\left( c \right)$ ``разрастается''
      от пустого множества $\emptyset$
      до множества элементарных событий $\Omega$
      с ростом $c$ от $-\infty$ до $+ \infty$
      $$A\left( -\infty \right)= \emptyset, A\left( + \infty \right)= \Omega$$

  \item Множество $A\left( c \right)$ растёт дискретными шагами.
      Это связано с тем, что уровни $H_k$ в нашей $\sigma$-алгебре неделимые,
      а каждый её элемент должен состоять из
      объединений этих уровней и ничего другого.
\end{enumerate}

Из этого всего делаем более конкретные выводы о том,
как изменяется значение функции $A\left( c \right)$ с ростом параметра $c$.
Должны быть опорные точки, на которых происходит ``скачок'' ---
точки, на которых к объединению добавляется ещё один или более уровней.
Поскольку имеется $n$ уровней, то может быть не более $n$ скачков:
ведь самый ``медленный'' рост будет происходить,
если добавлять по одному уровню на определённых константах,
а нужно пройти всё от пустого множества $\emptyset$
до множества элементарных исходов $\Omega$.

Выделим $m$ точек ($m \le n$) $c_1<c_2<\dots<c_m$
на числовой прямой $\mathbb{R}$ как значения случайной величины $\varkappa$
$$\varkappa: \Omega \rightarrow \left\{ c_1, c_2, \dots, c_m \right\}$$
Посмотрим, как соотносятся между собой
$A\left( c_i \right)$ и $A\left( c_{i-1} \right)$,
чтобы лучше понять природу скачков.
Сначала покажем, что $A\left( c_1 \right)$ является прообразом $c_1$
$$\varkappa^{-1}\left( c_1 \right)
  = \left\{ \omega \mcond \varkappa\left( \omega \right) = c_1 \right\}$$
Поскольку случайная величина не принимает значений до $c_1$,
то множество $A\left( c_1-0 \right)
= \left\{ \omega \mcond \varkappa\left( \omega \right) < c_1 \right\}$ пустое.
Получаем то, что хотели
\begin{align*}
  \varkappa^{-1}\left( c_1 \right)
      &= \left\{ \omega \mcond \varkappa\left( \omega \right) = c_1 \right\}
      \cup \emptyset = \\
      &= \left\{ \omega \mcond \varkappa\left( \omega \right) = c_1 \right\}
      \cup \left\{ \omega
          \mcond \varkappa\left( \omega \right) < c_1 \right\} = \\
      &= \left\{ \omega \mcond \varkappa\left( \omega \right) \le c_1 \right\}
      = A\left( c_1 \right)
\end{align*}
Идём дальше. Обозначим $c_0 = -\infty$.
Тогда в каждой точке $A\left( c_i \right), i = \overline{1,m}$
происходит скачок на множество $\varkappa^{-1}\left( c_i \right)$, то есть 
$$A\left( c_i \right)
  = A\left( c_{i-1} \right) \cup \varkappa^{-1}\left( c_i \right)$$
\begin{comment}
Так происходит, потому что имеет место равенство,
которое выполняется из-за того,
что функция имеет скачки лишь на параметрах $c_i$,
а между ними не меняет значения
$$A\left( c_i \right) = A\left( c_{i+1} - 0 \right)$$

В таком случае тождество очевидно
\begin{align*}
A\left( c_i \right)
  = \left\{ \omega \mcond \varkappa\left( \omega \right) \le c_i \right\} = \\
  = \left\{ \omega \mcond \varkappa\left( \omega \right) < c_i \right\} \cup
      \left\{ \omega \mcond \varkappa\left( \omega \right) = c_i \right\} = \\
  = A\left( c_{i-1}-0 \right) \cup \varkappa^{-1}\left( c_i \right)
  = A\left( c_{i-1} \right) \cup \varkappa^{-1}\left( c_i \right)
\end{align*}

Поскольку $\varkappa$ --- случайная величина, принимающая $m$ значений,
то её прообразы составляют разбиение пространства элементарных исходов $\Omega$.
А поскольку $A\left( c_{i-1} \right)$ состоит из объединений этих прообразов,
то оно не пересекается с $\varkappa^{-1}\left( c_i \right)$.
То есть мы знаем, как вычислять прообраз $\varkappa$
\begin{align*}
  \begin{cases}
      A\left( c_{i-1} \right) \cap \varkappa^{-1}\left( c_i \right)
      = \emptyset \\
      A\left( c_i \right)
      = A\left( c_{i-1} \right) \cup \varkappa^{-1}\left( c_i \right)
  \end{cases}
  \Rightarrow \varkappa^{-1}\left( c_i \right) =
      A\left( c_{i} \right) \setminus A\left( c_{i-1} \right)
\end{align*}
\end{comment}
Значит, случайная величина $\varkappa$ принимает значение $c_i$
при выпадении любого элементарного исхода $\omega$
из множества $A\left( c_{i} \right) \setminus A\left( c_{i-1} \right)$
\begin{equation}\label{randomVariableFirst}
  \varkappa\left( \omega \right) = c_i,
      \omega \in A\left( c_{i} \right) \setminus A\left( c_{i-1} \right)
\end{equation}
Запишем это в более удобном виде
$$\varkappa\left( \omega \right)
  = \sum_{i=1}^m c_i \cdot \Indicator{\omega
      \in A\left( c_{i} \right) \setminus A\left( c_{i-1} \right)}$$
Но эта сумма кажется уродливой из-за длинного индикатора и непонятного $m$.
Попытаемся разобраться,
в чём же дело и как прийти к изначальным $H_k$.

Помним, что $A\left( c_{i} \right) \setminus A\left( c_{i-1} \right)$ ---
объединение нескольких множеств уровня $H_k$.
\begin{comment}
Для любого $t$ разность множеств
$A\left( c_{t} \right) \setminus A\left( c_{t-1} \right) \neq \emptyset$
(когда это множество пустое, то индикатор просто не сработает и нечего считать)
можно представить как объединение двух непересекающихся множеств,
которые обозначим $H_1^t \in \mathfrak{F}$ и $H_2^t \in \mathfrak{F}$,
причём $H_1^t$ --- множество уровня, а $H_2^t$ --- произвольное множество
из $\mathfrak{F}$ (в том числе и пустое, если разность и есть множество уровня).
Тогда $t$-ое слагаемое примет следующий вид
$$c_t \cdot \Indicator{\omega \in
  A\left( c_{t} \right) \setminus A\left( c_{t-1} \right)}
  = c_t \cdot \Indicator{\omega \in H_1^t \cup H_2^t}$$

Поскольку множества $H_1^t$ и $H_2^t$ по условию не пересекаются,
то можно разбить индикатор на сумму
\begin{align*}
c_t \cdot \Indicator{\omega \in H_1^t \cup H_2^t}
  &= c_t \cdot \left( \Indicator{\omega \in H_1^t}
      + \Indicator{\omega \in H_2^t} \right) \\
  &= c_t \cdot \Indicator{\omega \in H_1^t}
      + c_t \cdot \Indicator{\omega \in H_2^t}
\end{align*}

Если ввести две константы $c_1^t$ и $c_2^t$, которые будут равны старой $c_t$,
то равенство примет более симпатичный вид
$$c_t \cdot \Indicator{\omega \in H_1^t}
  + c_t \cdot \Indicator{\omega \in H_2^t}
  = c_1^t \cdot \Indicator{\omega \in H_1^t}
  + c_2^t \cdot \Indicator{\omega \in H_2^t}$$

Если же $H_2^t$ не является пустым множеством $\emptyset$
или множеством уровня $H_k$, то нужно повторить процедуру,
разбив $H_2^t$ на объединение двух непересекающихся множеств --- на множество
уровня и множество из $\mathfrak{F}$.
В итоге (вследствие конечности множества $\mathfrak{F}$)
индикатор разности $A\left( c_{t} \right) \setminus A\left( c_{t-1} \right)$
будет разбита на сумму индикаторов множеств уровней.

Таким же образом можно поступить со всеми остальными индикаторами.
В итоге получим $n$ констант $d_1, d_2, \dots, d_n$
вместо $m$ чисел $c_1, c_2, \dots, c_m$.

Теперь сумма примет более приятный для глаз
и понятный из контекста начала раздела вид
\end{comment}
Тогда найдётся такой набор $d_1$, $\dots$, $d_n$, для которого справедливо
равенство
\begin{equation*}
  \varkappa\left( \omega \right)
  = \sum_{i=1}^m c_i \cdot \Indicator{\omega
      \in A\left( c_{i} \right) \setminus A\left( c_{i-1} \right)}
  = \sum_{i=1}^n d_i \cdot \Indicator{\omega \in H_i}
\end{equation*}
\begin{comment}
Видим, что теперь можно определить отображение из множества значений,
принимаемых случайной величиной $\xi$, в множество значений,
принимаемых случайной величиной $\varkappa$

$$f: \left\{ a_1, a_2, \dots, a_n \right\}
  \rightarrow \left\{ d_1, d_2, \dots, d_n \right\}$$

Попробуем показать, что $\varkappa$ является функцией от $\xi$.
Очевидно, что случайная величина $\xi$ имеет такой же вид,
что и $\varkappa$ --- сумма констант, умноженных на индикаторы,
так как мы только что показали, что все функции,
измеримые относительно $\sigma$-алгебры, порождённой случайной величиной $\xi$,
выглядят именно так
$$f\left( \xi\left( \omega \right) \right)
  = f\left( \sum_{i=1}^n a_i \cdot \Indicator{\omega \in H_i} \right)$$

Поскольку уровни $H_i$ не пересекаются,
то лишь одно слагаемое не будет равно нулю:
$\omega$ может принадлежать лишь одному уровню.
В таком случае запись принимает свой изначальный вид без суммы
\eqref{randomVariableFirst}
$$f\left( \xi\left( \omega \right) \right)
  = f\left( a_i \right),\; \omega \in H_i$$

Замечаем, что $f\left( a_i \right) = d_i$, а это и есть то значение,
которое принимает случайная величина $\varkappa$ на уровне $H_i$
$$f\left( \xi\left( \omega \right) \right)
  = f\left( a_i \right) = d_i
  = \varkappa\left( \omega \right),\; \omega \in H_i$$
\end{comment}
Определим
\begin{equation*}
  f\left( a_i \right) = d_i
\end{equation*}
Тогда
\begin{equation*}
  \forall \omega \in H_i:\qquad
  f\left( \xi\left( \omega \right) \right)
  = \varkappa\left( \omega \right)
\end{equation*}
Поскольку мы не привязывались к конкретным $i$ и конкретным $\omega$,
то получаем желаемое равенство
\begin{equation*}
  \varkappa = f\left( \xi \right)
\end{equation*}

Отсюда делаем следующий вывод
\begin{affirmation}\label{measurableRandomVariable}
  Случайной величине $\varkappa$
  необходимо и достаточно быть функцией случайной величины $\xi$,
  чтобы быть измеримой относительно $\sigma$-алгебры,
  порождённой случайной величиной $\xi$.
\end{affirmation}

Справедливо гораздо более общее утверждение, касающееся случайных величин с произвольным распределением.

\begin{theorem}
  Случайная величина $\eta$ измерима относительно $\sigma$-алгебры,
  порождённой случайной величиной $\xi$, тогда и только тогда,
  когда существует борелевская функция
  \begin{equation*}
    f: \mathbb{R} \rightarrow \mathbb{R}
  \end{equation*}
  такая, что
  \begin{equation*}
    f\left( \xi \right) = \eta
  \end{equation*}
  \cite[с.~219]{Shiryayev1}.
\end{theorem}
