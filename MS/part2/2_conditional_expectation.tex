\section{Условное математическое ожидание}
\index{условное!математическое ожидание}
\index{математическое ожидание!условное}
Имеется произвольная случайная величина $\eta$, интегрируемая с квадратом.
Нужно найти случайную величину $\tilde{\eta}$ которая
измерима в $\sigma\left( \xi \right)$
и ближайшая в среднем квадратическом к $\eta$.

\subsection{Проекция вектора}
Для наглядности начнём с геометрической интерпретации задачи.
Если представить $\eta$ как вектор в некоем пространстве $\mathfrak{L}$,
а $\sigma\left( \xi \right)$ как подпространство пространства $\mathfrak{L}$,
то $\tilde{\eta}$ будет ни что иное, как проекция случайной величины $\eta$
на пространство $\sigma\left( \xi \right)$.

Отдохнём от случайных величин и вспомним геометрию.

Имеется точка $x$ в пространстве $L'$.
Мы ищем такую точку $y$ в подпространстве $L\subset L'$,
что расстояние между $x$ и $y$ минимальное.
Значит, надо опустить перпендикуляр от $y$ на $L$.

У нас есть $e_1, e_2, \dots, e_n$ --- ортонормированный базис в $L$,
тогда $y$ можно найти по формуле
\index{проекция!вектора}
\index{вектор!проекция}
\begin{equation}\label{vectorProjection}
    y = \sum_{k=1}^n \left( x, e_k \right) \cdot e_k
\end{equation}

Потому что $y \in L$ должен лежать в пространстве $L$ по условию,
а это значит, что он должен быть линейной комбинацией базисных векторов
$e_1, e_2, \dots, e_n$ и это очевидно выполняется

Также разностью $x-y$ должен быть вектор, перпендикулярный пространству $L$.
То есть, скалярное произведение этой разности с любым вектором $z$
из пространства $L$ должно равняться нулю
$$\left( x-y \right) \perp L
    \Leftrightarrow \forall z \in L:
    \left( x-y,z \right)=0$$

Вследствие линейности скалярного произведения можно переписать это условие иначе
\begin{align*}
    \begin{cases}
        \forall z \in L: \left( x-y,z \right)=0\\
        \left( a+b,c \right)=\left( a,c \right)+\left( b,c \right)
    \end{cases}
    \Rightarrow \forall z \in L: \left( x,z \right)=\left( y,z \right)
\end{align*}

Покажем, что и это выполняется.
$z$ является линейной комбинацией базисных векторов. Запишем это
$$z = \sum_{k=1}^n \beta_k\cdot e_k$$

В таком случае скалярное произведение $\left( x,z \right)$ будет таким
$$\left( x,z \right)=\sum_{k=1}^n \beta_k\cdot \left( x,e_k \right)$$

С произведением $\left( y,x \right)$ придётся чуть-чуть повозиться
$$\left( y,x \right)
    =\left( \sum_{k=1}^n\left( x,e_k \right)\cdot e_k,
        \sum_{k=1}^n \beta_k\cdot e_k \right)
    =\sum_{k=1}^n \left( x,e_k \right)\cdot\beta_k$$

Как видим, суммы равны, а значит, проекция $x$ на $L$ найдена верно.

\subsection{Проекция случайной величины}
Возьмём $L$ --- множество всех случайных величин, которые
измеримы относительно $\sigma\left( \xi \right)$.
$$L \ni \sum_{k=1}^n c_k \cdot \indicatorof{H_k}, c_k \in \mathbb{R}$$

Но что же взять в качестве ортонормированного базиса?
По внешнему виду элементов пространства $L$ кажется,
что это $\indicatorof{H_k}$.
В качестве скалярного произведения случайных величин
возьмём математическое ожидание произведения.

Оказывается, $H_k$ действительно ортогональны
$$k_1 \neq k_2
    \Rightarrow H_{k_1} \cap H_{k_2} = \emptyset
    \Rightarrow
    \Mean{ \indicatorof{H_{k_1}}\cdot \indicatorof{H_{k_1}} } = 0$$

Теперь нужно нормировать эти базисные вектора,
а для этого их надо поделить на их нормы.
В нашем пространстве норма порождена скалярным произведением,
то есть
$$\left\| x \right\| = \sqrt{\left( x,x \right)}
    = \sqrt{\Mean{ x \cdot x }}
    = \sqrt{\Mean{ x^2 }}, x \in L$$

Теперь у нас есть всё необходимое для того,
чтобы представить ортонормированный базис.
Начнём преобразования $H_k$
$$e_k = \frac{\indicatorof{H_k}}
    {\sqrt{\mean{\left( \indicatorof{H_k} \right)^2}}}$$

Поскольку индикатор может принимать лишь одно из двух значений $0$ или $1$,
а их квадраты равны им самим, то в формуле квадрат тоже можно убрать
$$e_k = \frac{\indicatorof{H_k}}{\sqrt{\mean{\indicatorof{H_k}}}}$$

Также помним, что математическое ожидание в знаменателе есть ни что иное,
как вероятность события $H_k$,
и теперь у нас есть красивый ортонормированный базис
\begin{equation}\label{orthonormalBasis}
e_k = \frac{\indicatorof{H_k}}{\sqrt{\probability{H_k}}}
\end{equation}

Идём дальше, ищем проекцию.
Вспомним снова пример с векторами \eqref{vectorProjection}
$$y = \sum_{k=1}^n \left( x, e_k \right) \cdot e_k$$

Если заменить $y$ на $\tilde{\eta}$, а $x$ на $\eta$,
то получаем следующую картину, имеющую непосредственное отношение к задаче
$$\tilde{\eta} = \sum_{k=1}^n \left( \eta, e_k \right) \cdot e_k$$

Осталось заменить $e_k$ на то, что получили выше \eqref{orthonormalBasis}
$$\tilde{\eta}
    = \sum_{k=1}^n
        \left( \eta, \frac{\indicatorof{H_k}}{\sqrt{\probability{H_k}}} \right) 
        \cdot \frac{\indicatorof{H_k}}{\sqrt{\probability{H_k}}}$$

Заменяем скалярное произведение на математическое ожидание произведения
и получаем то, с чем можно дальше работать, не отвлекаясь на геометрию
$$\tilde{\eta}
    = \sum_{k=1}^n
        \mean{\left( \eta
            \cdot \frac{\indicatorof{H_k}}{\sqrt{\probability{H_k}}} \right)}
        \cdot \frac{\indicatorof{H_k}}{\sqrt{\probability{H_k}}}$$

Поскольку вероятность $\probability{H_k}$ --- константа,
то её можно вынести за математическое ожидание
$$\tilde{\eta}
    = \sum_{k=1}^n
        \frac{\Mean{ \eta \cdot \indicatorof{H_k} }}
            {\sqrt{\probability{H_k}}}
        \cdot \frac{\indicatorof{H_k}}{\sqrt{\probability{H_k}}}$$

При умножении знаменателей получаем вероятность события $H_k$.
Теперь у нас есть красивая формула для проекции случайной величины
\index{проекция!случайной величины}
\begin{equation}\label{discreteConditionalExpectation}
    \tilde{\eta} = \sum_{k=1}^n
        \frac{\Mean{\eta \cdot \indicatorof{H_k}}}{\probability{H_k}}
        \cdot \indicatorof{H_k}
\end{equation}

На что стоит обратить внимание в этой формуле:
\begin{enumerate}
    \item $\tilde{\eta}$ --- \textbf{случайная величина},
        так как индикатор вне математического ожидания никуда не девается
        и результат суммы будет зависеть от произошедшего $\omega$,
        а точнее от того, какому уровню $H_k$ оно принадлежит
    \item Когда $\omega$ принадлежит $H_k$,
        то результатом суммы будет среднее значение случайной величины $\eta$
        на событии $H_k$
\end{enumerate}

Если с первым пунктом всё очевидно,
то небольшое пояснение ко второму не помешает.

Нужно показать, что $k$-я ``координата'' случайной величины $\tilde{\eta}$
действительно даёт среднее значение случайной величины $\eta$ на событии $H_k$
$$\frac{\Mean{ \eta \cdot \indicatorof{H_k} }}{\probability{H_k}}$$

Начнём с определения математического ожидания
\begin{equation}\label{meanIndicator}
    \begin{aligned}
    \Mean{ \eta \cdot \indicatorof{H_k} }
        &=\integrall{\Omega}{\probability{d\omega}}{
            \eta\left( \omega \right) \cdot \indicatorof{H_k}} =\\
        &=\integrall{H_k}{\probability{d\omega}}{\eta\left( \omega \right)}
            + \integrall{\Omega \setminus H_k}{
                \probability{d\omega}}{0}
    \end{aligned}
\end{equation}

Видим математическое ожидание случайной величины,
которая гарантированно принимает нулевое значение
на множестве $\Omega \setminus H_k$,
что в свою очередь искажает желаемую картину и притягивает результат к нулю
с силой, которая пропорциональна $\probability{\Omega \setminus H_k}$.
То есть, ``вес'' каждого ненулевого значения случайной величины уменьшился.

Почему так происходит?
Потому что вероятность события $H_k$ в общем случае не равна единице.
Если ввести новую меру
$\mathbb{P}_k\left( A \right)=\frac{\probability{A}}{\probability{H_k}}$,
то наступит гармония, а
вероятность $\mathbb{P}_k\left( H_k \right)$ будет равна единице.

Из контекста понятно, что эта мера будет использоваться лишь в интеграле
по событию $H_k$,
поэтому её значение будет колебаться в пределах $\left[ 0;1 \right]$,
но строгости ради введём небольшую поправку (и увидим, что не напрасно)
$$\mathbb{P}_k\left( A \right)
    =\frac{\probability{A\cap H_k}}{\probability{H_k}}$$

Видим условную вероятность, а это значит, что мы на правильном пути!
Логично, что в поисках условного математического ожидания
должна была встретиться условная вероятность
$$\mathbb{P}_k\left( A \right)
    =\frac{\probability{A\cap H_k}}{\probability{H_k}}
    =\probability{A \mcond H_k}$$

Теперь математическое ожидание \eqref{meanIndicator}
принимает несколько иной вид
$$\Mean{ \eta \cdot \indicatorof{H_k} }
        = \probability{H_k}
            \cdot \integrall{H_k}{\probability{d\omega \mcond H_k}}{
                \eta\left( \omega \right)}$$

Тут уже уровень $H_k$ играет роль целого множества элементарных исходов,
его мера $\probability{H_k \mcond H_k}$ равна единице,
а мы получаем действительно среднее значение случайной величины $\eta$
на множестве $H_k$, умноженное на вероятность $\probability{H_k}$.
Значит, осталось лишь поделить обе части на $\probability{H_k}$
$$\frac{\Mean{ \eta \cdot \indicatorof{H_k} }}{\probability{H_k}}
        = \integrall{H_k}{\probability{d\omega \mcond H_k}}{
            \eta\left( \omega \right)}$$

\begin{definition}[Условное математическое ожидание случайной величины
    относительно случайного события]
    \label{eventConditionalExpectationDefinition}
    Условное математическое ожидание случайной величины $\xi$
    относительно \textbf{события} $A$ \cite[стр.~68]{BorovkovPT}
    обозначается $\Mean{\xi \mcond A}$ и считается по формуле
    $$\Mean{\xi \mcond A}
        = \frac{\Mean{\xi \cdot \indicatorof{A}}}{\probability{A}}
            = \integrall{A}{\probability{d\omega \mcond A}}{
                \xi\left( \omega \right)}$$
\end{definition}

Пользуясь только что введённым обозначением,
можно более красиво переписать формулу \eqref{discreteConditionalExpectation}
для получения проекции случайной величины $\eta$ на $\sigma$-алгебру,
порождённой уровнями $H_1, H_2, \dots, H_n$ 
$$\tilde{\eta}
    = \sum_{k=1}^n \Mean{ \eta \mcond H_k } \cdot \indicatorof{H_k}$$

Забегая наперёд,
введём определение частного случая условного математического ожидания
случайной величины относительно \textbf{$\sigma$-алгебры},
чтобы обратить внимание на этот важный момент.

\begin{definition}[Условное математическое ожидание
    случайной величины относительно конечной сигма-алгебры]
    \index{условное!математическое ожидание!относительно конечной сигма-алгебры}
    \index{математическое ожидание!условное!относительно конечной сигма-алгебры}
    \label{discreteConditionalExpectationDefinition}
    Есть $\sigma$-алгебра $\mathfrak{F}_1$,
    разбитая на $n$ уровней $H_1, H_2, \dots, H_n$.
    Тогда условное математическое ожидание случайной величины
    $\eta$ относительно этой $\sigma$-алгебры --- \textbf{случайная величина},
    которая обозначается $\Mean{ \eta \mcond \mathfrak{F_1} }$
    и вычисляется по формуле
    $$\Mean{ \eta \mcond \mathfrak{F}_1 }
        = \sum_{k=1}^n
            \frac{\Mean{ \eta \cdot \indicatorof{H_k} }}
                {\probability{H_k}}
            \cdot \indicatorof{H_k}$$
\end{definition}

\begin{remark}
    У нас есть определения условного математического ожидания
    относительно \textbf{$\sigma$-алгебры} $\mathfrak{F}$ (определение
    \ref{conditionalExpectationDefinition})
    и относительно \textbf{случайного события} $A$ (определение 
    \ref{eventConditionalExpectationDefinition}).
    Из контекста будет ясно, какое именно определение используется,
    поэтому путаницы возникнуть не должно.

    Например, последнее определение может выглядеть немного странно
    $$\Mean{ \eta \mcond \mathfrak{F}_1 }
        = \sum_{k=1}^n \Mean{ \eta \mcond H_k }
            \cdot \indicatorof{H_k}$$

    Зато при более детальном рассмотрении из записи очевиден её смысл:
    условное математическое ожидание относительно $\sigma$-алгебры --- вектор
    в пространстве с базисом $\left\{ H_1, \dots, H_n \right\}$, элементами
    которого являются проекции вектора $\eta$ на соответствующие оси.
    
    Ведь $\Mean{ \eta \mcond H_k }$ --- ни что иное, как проекция
    вектора (случайной величины) $\eta$ на ось (уровень) $H_k$.
    Эта величина является скаляром, как и проекция вектора на ось.
\end{remark}

\begin{lemma}[Равенство скалярных произведений для конечной сигма-алгебры]
    Для случайной величины $\eta$ и её проекции $\tilde{\eta}$
    на $\sigma$-алгебру $\mathfrak{F}_\xi$,
    порождённую случайной величиной $\xi$,
    принимающей конечное количество значений,
    выполняется равенство скалярных произведений
    \begin{equation}\label{scalarMulEquality}
        \forall A \in \mathfrak{F}_\xi:
            \Mean{ \tilde{\eta} \cdot \indicatorof{A} }
                = \Mean{ \eta \cdot \indicatorof{A} }
    \end{equation}
\end{lemma}
\begin{proof}
Для начала распишем $\tilde{\eta}$ по определению
$$\Mean{ \tilde{\eta} \cdot \indicatorof{A} }
    = \mean{\left( \sum_{k=1}^n
        \frac{\Mean{ \eta \cdot \indicatorof{H_k} }}
            {\probability{H_k}}
        \cdot \indicatorof{H_k} \cdot \indicatorof{A} \right)}$$

Произведение индикаторов $\indicatorof{H_k}$ и $\indicatorof{A}$ --- индикатор
пересечения $\indicatorof{H_k \cap A}$.
Воспользуемся линейностью математического ожидания, не забывая,
что дробь в каждом слагаемом --- константа
и выносится за знак математического ожидания
$$\mean{\left( \sum_{k=1}^n
        \frac{\Mean{ \eta \cdot \indicatorof{H_k} }}
            {\probability{H_k}}
        \cdot \indicatorof{H_k} \cdot \indicatorof{A} \right)}
    = \sum_{k=1}^n
        \frac{\Mean{ \eta \cdot \indicatorof{H_k} }}
            {\probability{H_k}}
        \cdot \Mean{ \indicatorof{H_k \cap A} }$$

Помним, что математическое ожидание индикатора --- вероятность
$$\sum_{k=1}^n
    \frac{\Mean{ \eta \cdot \indicatorof{H_k} }}{\probability{H_k}}
        \cdot \Mean{ \indicatorof{H_k \cap A} }
    = \sum_{k=1}^n \frac{\Mean{ \eta \cdot \indicatorof{H_k} }}
        {\probability{H_k}}
        \cdot \probability{H_k \cap A}$$

Замечаем условную вероятность
\begin{align*}
    \sum_{k=1}^n \frac{\Mean{ \eta \cdot \indicatorof{H_k} }}
        {\probability{H_k}} \cdot \probability{H_k \cap A}
    &= \sum_{k=1}^n \Mean{ \eta \cdot \indicatorof{H_k} }
        \cdot \frac{\probability{H_k \cap A}}{\probability{H_k}} = \\
    &= \sum_{k=1}^n \Mean{ \eta \cdot \indicatorof{H_k} }
        \cdot \probability{A \mcond H_k}
\end{align*}

Поскольку $A$ принадлежит множеству случайных событий $\mathfrak{F}_\xi$,
то условная вероятность $\probability{A\mcond H_k}$ равна либо нулю, либо единице,
поскольку $A$ либо включает в себя уровень $H_k$, либо не пересекается с ним.
То есть, получился индикатор $\indicator{H_k \subseteq A}$.
А этот индикатор говорит о том, что теперь надо суммировать лишь по тем уровням,
которые являются частью события $A$,
а дальше можно смело воспользоваться линейностью математического ожидания
\begin{align*}
\sum_{k=1}^n \Mean{ \eta \cdot \indicatorof{H_k} }
    \cdot \probability{A \mcond H_k}
    = \sum_{k=1}^n \Mean{ \eta \cdot \indicatorof{H_k} }
        \cdot \indicator{H_k \subseteq A} = \\
    = \sum_{H_k \subseteq A} \Mean{ \eta \cdot \indicatorof{H_k} }
    = \Mean{ \sum_{H_k \subseteq A} \eta \cdot \indicatorof{H_k} }
\end{align*}

Далее мы имеем полное математическое и моральное право
вынести $\eta$ за знак суммы.
Если с математикой всё очевидно (работает закон дистрибутивности),
то напомню о морально-этической стороне дела: нам нужно,
пройтись по всем возможным индикаторам $\indicatorof{H_k}$,
из которых лишь один сработает (будет равен единице, а не нулю),
поэтому сумма нужна лишь для того,
чтобы не писать в конце каждой строчки ``для тех $\omega$, что входят в $H_k$''
(помним, что случайная величина и индикатор --- функции
от элементарного события $\omega$)
$$\Mean{ \sum_{H_k \subseteq A} \eta \cdot \indicatorof{H_k} }
    = \Mean{\eta \left( \omega \right) \cdot \sum_{H_k \subseteq A}
        \indicatorof{H_k}\left( \omega \right)}$$

Сумма индикаторов непересекающихся событий --- индикатор их объединения,
которое является множеством $A$.
Не забываем, что оно может состоять из объединений уровней и только из них
(или же быть пустым)
$$\Mean{ \eta \cdot \sum_{H_k \subseteq A} \indicatorof{H_k} }
    = \Mean{ \eta \cdot \indicatorof{A} }$$

Значит, равенство \eqref{scalarMulEquality} выполняется.
\end{proof}

\begin{remark}
    В связи с выполнением равенства скалярных произведений можем сделать вывод,
    что математическое ожидание случайной величины и её проекции тоже равны.
    Это нетрудно показать,
    установив $A$ равным всему множеству элементарных исходов
    (индикатор в таком случае станет просто тождественной единицей)
    $$\mean{\eta}
        = \Mean{ \eta \cdot \indicatorof{\Omega} }
        = \Mean{ \tilde{\eta} \cdot \indicatorof{\Omega} }
        = \mean{ \tilde{\eta} }$$
\end{remark}

\subsection{Условное математическое ожидание}
Введём же общее определение для условного математического ожидания
случайной величины относительно $\sigma$-алгебры

\begin{definition}[Условное математическое ожидание случайной величины
    относительно сигма-алгебры]
    \index{условное!математическое ожидание!в общем виде}
    \index{математическое ожидание!условное!в общем виде}
    Условным математическим ожиданием случайной величины $\eta$
    относительно $\sigma$-алгебры $\mathfrak{F}_1$
    называется такая случайная величина $\tilde{\eta}$, что
    \begin{enumerate}
        \item Случайная величина $\tilde{\eta}$
                измерима относительно $\sigma$-алгебры $\mathfrak{F}_1$
        \item Выполняется равенство скалярных произведений
            $$\forall A \in \mathfrak{F}_1:
                \Mean{ \tilde{\eta} \cdot \indicatorof{A} }
                    = \Mean{ \eta \cdot \indicatorof{A} }$$
    \end{enumerate}
    Обозначение $\tilde{\eta} = \Mean{ \eta \mcond \mathfrak{F}_1 }$

\end{definition}

\begin{remark}
    Условное математическое ожидание случайной величины $\eta$
    относительно $\sigma$-алгебры, порождённой случайной величиной $\xi$,
    будем обозначать $\Mean{ \eta \mcond \sigma\left( \xi \right) }$,
    а более кратко $\Mean{ \eta \mcond \xi }$.

    То есть, имеем три эквивалентных записи
    $$\Mean{ \eta \mcond \mathfrak{F}_{\xi} }
        = \Mean{ \eta \mcond \sigma\left( \xi \right) }
        = \Mean{ \eta \mcond \xi }$$
\end{remark}

Немного остановимся на примере, чтобы понять, что у нас есть на данный момент

\begin{example}\label{discreteConditionalExpectationExample}
    У нас есть две дискретные случайные величины $\xi$ и $\eta$
    с совместным дискретным распределением
    $$\Probability{\xi = a_i, \eta = b_j} = p_{ij}$$

    Очевидно, что числа $p_{ij}$ обладают некоторым свойствам
    $$p_{ij} \ge 0, \qquad \sum_{i,j=1}^\infty p_{ij} = 1,
        \qquad \Probability{\eta = b_j} = \sum_{i=1}^\infty p_{ij}$$

    Посчитаем условное математическое ожидание согласно формуле
    из определения \eqref{discreteConditionalExpectationDefinition}
    $$\Mean{\xi \mcond \sigma\left( \eta \right)} = \sum_{k=1}^n
        \frac{\Mean{\eta \cdot \indicatorof{H_k}}}{\probability{H_k}}
        \cdot \indicatorof{H_k}$$

    Для этого выясним, чему равно математическое ожидание $\xi$
    при определённом значении $\eta$ по формуле из определения
    \eqref{eventConditionalExpectationDefinition}
    $$\Mean{\xi \mcond \eta = b_j}
        = \frac{\Mean{\xi \cdot \indicatorof{\eta = b_j}}}{
            \Probability{\eta = b_j}}
        = \frac{\sum_{i=1}^{\infty} a_i \cdot p_{ij}}{
            \sum_{i=1}^{\infty} p_{ij}}$$
\end{example}

Попробуем обобщить определение условного математического ожидания,
чтобы обладать универсальной формулой, из которой можно делать какие-то выводы.
Начнём с того, что у нас уже есть
$$\Mean{ \eta \mcond \mathfrak{F}_\xi }
    = \sum_{k=1}^n \Mean{ \eta \mcond H_k } \cdot \indicatorof{H_k}$$

Множество уровня $H_k$ --- прообраз одного из значений случайной величины $\xi$,
которой порождена $\sigma$-алгебра $\mathfrak{F}_\xi$.
Если назвать эти значения $a_1, a_2, \dots, a_n$, то запись примет следующий вид
\begin{equation}\label{conditionalExpectationUgly}
    \Mean{ \eta \mcond \sigma\left( \xi \right) }
        = \sum_{k=1}^n \Mean{\eta \mcond \xi^{-1}\left( a_k \right)}
            \cdot \Indicator{\xi^{-1}\left( a_k \right)}
\end{equation}

Вспомним альтернативные записи прообраза
$$\xi^{-1}\left( a_k \right)
    = \left\{ \omega \mcond \xi\left( \omega \right) = a_k \right\}
    = \left\{ \xi = a_k \right\}$$

И перепишем формулу \eqref{conditionalExpectationUgly}
$$\Mean{ \eta \mcond \sigma\left( \xi \right) }
    = \sum_{k=1}^n \Mean{\eta \mcond \xi = a_k}
        \cdot \indicatorof{\xi = a_k}$$

Теперь введём функцию
$\varphi^{\eta}\left( x \right) = \Mean{ \eta \mcond \xi=x }$
и условное математическое ожидание примет следующий вид
$$\Mean{ \eta \mcond \xi }
    = \sum_{k=1}^n \varphi^{\eta}\left( a_k \right)
        \cdot \indicatorof{\xi = a_k}$$

Вновь вспоминаем роль суммы и индикаторов и видим,
что условное математическое ожидание в нашей формуле
принимает значение $\varphi^{\eta}\left( x \right)$ в зависимости от того,
какое значение приняла случайная величина $\xi\left( \omega \right)$.
То есть, можно переписать равенство следующим образом
$$\Mean{ \eta \mcond \xi }
    = \varphi^{\eta}\left( a_k \right): \xi\left( \omega \right) = a_k$$

То есть, можно просто подставить значение случайной величины $\xi$
в качестве аргумента функции $\varphi^{\eta}$
и получим условное математическое ожидание
$$\Mean{ \eta \mcond \xi }
    = \varphi^{\eta}\left( \xi \right)$$

Остановимся ещё немного на функции $\varphi^{\eta}$.
Она является случайной величиной, поэтому перепишем равенство следующим образом
$$\varphi^{\eta}\left( \xi \right)\left( \omega \right)
    = \Mean{ \eta \mcond \xi }\left( \omega \right)$$

Тогда будет корректна следующая запись
$$\varphi^{\eta}\left( \xi \right)\left( \omega \right)
    = \Bigl. \Mean{ \eta \mcond \xi = t }
        \Bigr|_{t=\xi\left( \omega \right)}$$

Не путаем случайную величину $\xi\left( \omega \right)$ саму по себе
со случайной величиной в случайном событии
$$H_t = \left\{ \xi = t \right\}
    = \left\{ \tilde{\omega}\mcond \xi\left( \tilde{\omega} \right) = t \right\}$$

Для удобства вернёмся к обозначению $H_t$
$$\varphi^{\eta}\left( \xi \right)\left( \omega \right)
    = \Bigl. \Mean{ \eta \mcond H_t }
        \Bigr|_{t=\xi\left( \omega \right)}
    = \left. \frac{\Mean{ \eta \cdot \indicatorof{H_t} }}
        {\probability{H_t}} \right|_{t=\xi\left( \omega \right)}$$

Покажем, что такая формула вычисления условного математического ожидания
подходит для общего случая.\footnote{Так как формула была выведена из
условного математического ожидания относительно $\sigma$-алгебры,
порождённой случайной величиной, принимающей конечное количество значений,
то справедливость формулы для этого случая доказывать уже нет нужды}

\begin{lemma}[Равенство скалярных произведений в общем случае]
    В общем случае случайная величина $\varphi^{\eta}\left( \xi \right)$
    является условным математическим ожиданием случайной величины $\eta$
    относительно $\sigma$-алгебры, порождённой случайной величиной $\xi$
    $$\Mean{ \eta \mcond \sigma\left( \xi \right) }
        = \varphi^{\eta}\left( \xi \right)$$
\end{lemma}
\begin{proof}
Нужно доказать то, что выполняются оба свойства
условного математического ожидания.

То, что $\varphi^{\eta}\left( \xi \right)$ измерима
относительно $\sigma\left( \xi \right)$,
очевидно из определения: $\varphi^{\eta}\left( \xi \right)$ является функцией
случайной величины $\xi$, а это и есть измеримость.

Дальше придётся немного повозиться.

$$\forall A \in \sigma\left( \xi \right):
    \Mean{ \varphi^{\eta}\left( \xi \right) \cdot \indicatorof{A} }
        =\Mean{ \eta \cdot \indicatorof{A} }$$

Следуем определению. Пока что ничего очевидного нет кроме надежды на то,
что была выведена достаточно общая формула, которая должна работать
$$\Mean{ \varphi^{\eta}\left( \xi \right) \cdot \indicatorof{A} }
    = \integrall{\Omega}{d\mathbb{P}}{
    \varphi^{\eta}\left( \xi \right) \cdot \indicatorof{A}}$$

Применим индикатор и будем интегрировать
не по всему множеству элементарных исходов, а лишь по событию $A$,
а также в явном виде покажем элементарный исход $\omega$,
так как сейчас с ним надо будет поработать основательно
\begin{equation}\label{conditionalExpectationIntegralStart}
    \integrall{\Omega}{d\mathbb{P}}{
        \varphi^{\eta}\left( \xi \right) \cdot \indicatorof{A}}
    = \integrall{A}{\probability{d\omega}}{
        \varphi^{\eta}\left( \xi\left( \omega \right) \right)}
\end{equation}

Теперь нужно немного остановиться и подумать, что же делать дальше.
Немного выше оказалось, что сама по себе запись
$\varphi^{\eta}\left( \xi \right)$ не даёт ничего полезного.
Копнём немного глубже и посмотрим на то, что есть у нас.
Значение случайной величины использовалось лишь
для восстановления случайного события,
которому принадлежит произошедший элементарный исход $\omega$\footnote{Ведь
именно по значению случайной величины мы и находили уровни, элементарные исходы
которых для нас неразличимы внутри одного множества уровня
$$H_t = \xi^{-1}\left( a_t \right)
    = \left\{ \omega \mcond \xi\left( \omega \right) = a_t \right\}$$}.
То есть, мы знали, чему равна случайная величина,
но не знали, какое именно событие произошло,
зато могли определить, какому уровню принадлежит произошедшее событие.
Тут же у нас есть интеграл
и мы проходим по каждому мельчайшему событию $d\omega$.
Вспомним, чему равна $\varphi^{\eta}\left( x \right)$
$$\varphi^{\eta}\left( x \right) = \Mean{\eta \mcond \xi = x}$$

А теперь распишем условное математическое ожидание
$$\varphi^{\eta}\left( x \right)
    = \Mean{\eta \mcond \xi = x}
    = \frac{\Mean{\eta \cdot \Indicatorof{\xi = x}}}{\Probability{\xi = x}}$$

В общем случае для непрерывных случайных величин такая запись не имеет смысла,
но мы как раз рассматриваем очень маленькие значения,
а усложнять нет желания.
Поэтому просто подставляем получившееся выражение
в интеграл \eqref{conditionalExpectationIntegralStart}
\begin{equation}\label{conditionalExpectationMagic}
    \integrall{A}{\probability{d\omega}}{
        \varphi^{\eta}\left( \xi\left( \omega \right) \right)}
        = \integrall{A}{\probability{d\omega}}{
            \frac{\Mean{\eta \cdot \Indicatorof{\xi=x}}}{\Probability{\xi = x}}}
\end{equation}


Дальше происходит магия, которую можно трактовать по-разному
\begin{enumerate}[label=\bfseries Формулировка \arabic*:]
    \item
        Воспользовавшись вышесказанным,
        заменим событие $\left\{ \xi = x \right\}$
        на $d\omega$ и продолжим колдовать
        $$\integrall{A}{\probability{d\omega}}{
            \frac{\Mean{\eta \cdot \Indicatorof{\xi = x}}}
                {\Probability{\xi = x}}}
            = \integrall{A}{\probability{d\omega}}{
                \frac{\Mean{ \eta \cdot \indicatorof{d\omega} }}
                    {\probability{d\omega}}}$$

        Вероятности сокращаются, хоть это и немного смущает,
        а $d\omega$ находится в индикаторе, что ещё больше нагнетает обстановку.
        Учтём внесённые изменения и перепишем
        математическое ожидание через интеграл
        $$\integrall{A}{\probability{d\omega}}{
            \frac{\Mean{ \eta \cdot \indicatorof{d\omega} }}
                {\probability{d\omega}}}
            = \int\limits_A \integrall{\Omega}{\probability{d\tilde{\omega}}}{
                \eta \cdot \indicatorof{d\omega} \cdot}$$

        Не путаемся: $d\omega$ принадлежит внешнему интегралу,
        а $d\tilde{\omega}$ внутреннему.
        Индикатор упрощает нашу задачу,
        сужая пределы интегрирования внутреннего интеграла
        до маленького события $d\omega$
        $$\int\limits_A \integrall{\Omega}{\probability{d\tilde{\omega}}}{
                \eta \cdot \indicatorof{d\omega} \cdot}
                = \int\limits_A \integrall{d\omega}
                    {\probability{d\tilde{\omega}}}
                    {\eta\left( \tilde{\omega}\right)}$$

        Поскольку событие $d\omega$ и без того маленькое,
        дробить его на более мизерные $d\tilde{\omega}$ смысла нет,
        а это значит, что внутренний интеграл просто уничтожается
        и остаётся произведение случайной величины $\eta$
        на вероятность события $d\omega$
        $$\integrall{d\omega}{\probability{d\tilde{\omega}}}{
            \eta\left( \tilde{\omega} \right)}
            = \eta\left( \omega \right) \cdot \probability{d\omega}$$
    \item
        Если посмотреть на исходный двойной интеграл,
        то можно увидеть условное математическое ожидание $\eta$
        относительно события $\left\{ \xi = x \right\}=d\omega$
        $$\integrall{A}{\probability{d\omega}}{
            \frac{\Mean{\eta \cdot \Indicatorof{\xi = x}}}
                    {\Probability{\xi = x}}}
            = \integrall{A}{\probability{d\omega}}{\Mean{\eta \mcond d\omega}}$$

        Если определить $d\omega$ как случайное событие,
        на котором случайная величина $\eta$ принимает
        одно и то же значение почти всюду,
        то математическое ожидание равно значению $\eta$
        при появлении почти любого события из $d\omega$\footnote{Нам достаточно
        постоянства значения $\xi\left( \omega \right)$ \textbf{почти всюду}
        на событии $d\omega$,
        так как интеграл Лебега простой функции (функции,
        что принимает конечное число значений \cite[стр.~53]{DorogovtsevIT})
        --- сумма значений функции, умноженных на меры
        соответствующих им прообразов \cite[стр.~69]{DorogovtsevIT};
        в противном случае результатом будет наибольшее значение
        из интегралов Лебега всех простых функций,
        не превышающих данную в каждой точке.
        А это значит, что, если и будут отклонения от основного значения функции
        $\xi$ на событии $d\omega$,
        то они будут уничтожаться мерой своих прообразов, равными нулю
        (в связи с тем, что функция $\xi\left( \omega \right)$
        равна одному и тому же значению почти всюду на $\omega$)
        }
        (если значение на промежутке $d\omega$ --- константа,
        то очевидно, что среднее значение будет равно ей же).
        $$\integrall{A}{\probability{d\omega}}{\Mean{\eta \mcond d\omega}}
            = \integrall{A}{\probability{d\omega}}{\eta}$$
\end{enumerate}

С этим моментом разобрались, вернёмся же к нашему двойному интегралу
\eqref{conditionalExpectationMagic}.
Получаем такой вот результат
$$\integrall{A}{\probability{d\omega}}{
    \frac{\Mean{\eta \cdot \Indicatorof{\xi = x}}}
        {\Probability{\xi = x}}}
    = \integrall{A}{\probability{d\omega}}{\eta}$$

Но ведь это и есть искомое математическое ожидание!
Значит, свойство доказано, формула верна
$$\integrall{A}{\probability{d\omega}}{\eta}
    = \integrall{\Omega}{\probability{d\omega}}{
        \eta \cdot \indicatorof{A} \cdot}
    = \Mean{ \eta \cdot \indicatorof{A} }$$
\end{proof}

Теперь вернёмся к менее абстрактным вещам и посмотрим,
как выглядит условное математическое ожидание,
когда случайные величины $\xi$ и $\eta$ имеют совместную плотность распределения
$$\Probability{\left( \xi, \eta \right) \in \Delta}
    = \iint\limits_\Delta\pdf{x,y} \, dx \, dy$$

В таком случае компонента $\xi$ имеет плотность $r$
$$r\left( x \right) = \integral{\mathbb{R}}{}{y}{\pdf{x,y}}$$

Компонента $\eta$ имеет плотность $q$
$$q\left( y \right) = \integral{\mathbb{R}}{}{x}{\pdf{x,y}}$$

Уточним определение функции $\varphi^{\eta}\left( x \right)$
для данного случая. Вот первоначальный вариант
$$\varphi^{\eta}\left( x \right)
    = \Mean{ \eta \mcond \xi=x }
    = \frac{\Mean{ \eta \cdot \indicator{\xi=x} }}
        {\Probability{\xi=x}}$$

В данном (непрерывном) случае вероятность события $\Probability{\xi=x}$ является
плотностью случайной величины $\xi$ в точке $x$
$$\Probability{\xi=x} = r\left( x \right)$$

Математическое ожидание случайной величины $\eta$,
умноженной на индикатор $\indicator{\xi=x}$,
есть ни что иное как математическое ожидание $\eta$ при фиксированном $\xi=x$
$$\Mean{ \eta \cdot \indicator{\xi=x} }
    = \integral{\mathbb{R}}{}{y}{y \cdot \pdf{x,y}}$$

Теперь у нас есть конкретная формула для $\varphi^{\eta}\left( x \right)$
для случая непрерывных случайных величин с общей плотностью распределения
\begin{equation}\label{phiIntegral}
    \varphi^{\eta}\left( x \right)
        = \frac{\integral{\mathbb{R}}{}{y}{y \cdot \pdf{x,y}}}
            {r\left( x \right)}
        = \frac{\integral{\mathbb{R}}{}{y}{y \cdot \pdf{x,y}}}
            {\integral{\mathbb{R}}{}{y}{\pdf{x,y}}}
\end{equation}

Докажем снова, что $\varphi^\eta\left( \xi \right)$ является
условным математическим ожиданием случайной величины $\eta$
относительно $\sigma$-алгебры, порождённой случайной величиной $\xi$.
Чтобы не было скучно, будем доказывать несколько иначе, чем ранее.

\begin{lemma}[Равенство скалярных произведений
    условного математического ожидания
    случайных величин с совместной плотностью]
    Пускай имеются две случайные величины $\left( \xi, \eta \right)$
    с совместной плотностью $\pdf{x,y}$.
    \index{условное!математическое ожидание!
        случайных величин с совместной плотностью}
    Тогда функция
    $$\varphi^{\eta}\left( \xi \right)
        = \left. \frac{\integral{\mathbb{R}}{}{y}{y \cdot \pdf{x,y}}}
            {\integral{\mathbb{R}}{}{y}{\pdf{x,y}}} \right|_{x=\xi}$$

    Является условным математическим ожиданием
    $\Mean{ \eta \mcond \xi }$

\end{lemma}
\begin{proof}
Первое свойство снова очевидно, поэтому надо доказать
\begin{equation}\label{conditionalExpectationContinualStart}
    \forall A \in \sigma\left( \xi \right):
        \Mean{\varphi^{\eta}\left( \xi \right) \cdot \indicatorof{A}}
            = \Mean{ \eta \cdot \indicatorof{A} }
\end{equation}

У нас есть совместная плотность и мы хотим посчитать математическое ожидание,
пользуясь именно ею.
Для этого превратим индикатор $\indicator{\omega \in A}$
в функцию случайной величины $\xi$.
Поскольку любое событие $A$ принадлежит $\sigma\left( \xi \right)$,
то оно представимо в виде
$\xi^{-1}\left( \Delta \right), \Delta \in \mathfrak{B}$.
Перепишем индикатор следующим образом:
$\indicator{\omega \in A} = \indicator{\xi \in \Delta}$.
И вот теперь мы готовы к тому,
чтобы записать определение математического ожидания
$$\Mean{ \varphi^{\eta}\left( \xi \right) \cdot \indicatorof{A} }
    = \integral{\mathbb{R}}{}{y}{ \integral{\mathbb{R}}{}{x}{
        \varphi^{\eta}\left( x \right) \cdot \indicator{x \in \Delta}
            \cdot \pdf{x,y}}}$$

От $y$ зависит лишь совместная плотность, а интеграл от неё по всей оси $y$
является плотностью распределения $\xi$.
То есть, интеграл по $y$ уходит,
а вместо $\pdf{x,y}$ появляется $r\left( x \right)$.
Также учтём индикатор и сузим область интегрирования с $\mathbb{R}$ до $\Delta$
$$\integral{\mathbb{R}}{}{y}{ \integral{\mathbb{R}}{}{x}{
    \varphi^{\eta}\left( x \right) \cdot \indicator{x \in \Delta}
        \cdot \pdf{x,y}}}
    = \integral{\Delta}{}{x}{
        \varphi^{\eta}\left( x \right) \cdot r\left( x \right)}$$

Дальше распишем функцию $\varphi^{\eta}$,
пользуясь формулой \eqref{phiIntegral}
$$\integral{\Delta}{}{x}{
    \varphi^{\eta}\left( x \right) \cdot r\left( x \right)}
    = \integral{\Delta}{}{x}{\left(
        \frac{\integral{\mathbb{R}}{}{y}{y \cdot \pdf{x,y}}}
            {r\left( x \right)}
        \cdot r\left( x \right) \right)}$$

Сократим одинаковые плотности и получим интересный двойной интеграл
$$\integral{\Delta}{}{x}{\left(
        \frac{\integral{\mathbb{R}}{}{y}{y \cdot \pdf{x,y}}}
            {r\left( x \right)}
        \cdot r\left( x \right) \right)}
    = \integral{\Delta}{}{x}{
        \integral{\mathbb{R}}{}{y}{y \cdot \pdf{x,y}}}$$

Вернём индикатор обратно в интеграл
$$\integral{\Delta}{}{x}{
    \integral{\mathbb{R}}{}{y}{y \cdot \pdf{x,y} \cdot}}
    =\integral{\mathbb{R}}{}{x}{
        \integral{\mathbb{R}}{}{y}{y \cdot \indicator{x \in \Delta}
            \cdot \pdf{x,y}}}$$

Видим, что это и есть то математическое ожидание, которое нам нужно
$$\integral{\mathbb{R}}{}{x}{
    \integral{\mathbb{R}}{}{y}{y \cdot \indicator{x \in \Delta}
        \cdot \pdf{x,y}}}
    = \Mean{ \eta \cdot \indicatorof{\xi \in \Delta} }
    = \Mean{ \eta \cdot \indicatorof{A} }$$

Это значит, что тождество доказано и условное математическое ожидание
для случайных величин с совместной плотностью считается с помощью
$$\varphi^{\eta}\left( x \right)
    = \frac{\integral{\mathbb{R}}{}{y}{y \cdot \pdf{x,y}}}
                {\integral{\mathbb{R}}{}{y}{\pdf{x,y}}}$$

По формуле
$$\Mean{ \eta \mcond \xi }
    = \varphi^{\eta}\left( \xi \right)
    = \left. \varphi^{\eta}\left( x \right) \right|_{x=\xi}$$
\end{proof}

\begin{theorem}[Существование условного математического ожидания]
    Условное математическое ожидание существует всегда
    и единственное почти наверное
\end{theorem}
\begin{proof}
\cite[стр.~142]{BorovkovMS}
\end{proof}

\subsection{Свойства условного математического ожидания}
\index{условное!математическое ожидание!свойства}
\index{математическое ожидание!условное!свойства}
Были даны определения условного математического ожидания для разных случай,
теперь настало время привести основные свойства,
которые позволят облегчить процедуру вычисления.\footnote{Также
со свойствами и их доказательствами можно ознакомиться в книгах
Ширяева \cite[стр.~270]{Shiryayev1} и Боровкова \cite[стр.~143]{BorovkovMS}
}

\begin{enumerate}[label=\Roman*]
    \item\label{conditionalExpectationProperty:totalProbability}
        Формула полной вероятности \cite[стр.~144]{BorovkovMS}
        $$\mean{\Mean{ \eta \mcond \mathfrak{F}_1 }} = \mean{\eta}$$
    \item Условное математическое ожидание неотрицательной случайной величины
        неотрицательно почти наверное
        $$\eta \ge 0
            \Rightarrow \Mean{ \eta \mcond \mathfrak{F}_1 } \ge 0$$
    \item\label{conditionalExpectationProperty:Jensen}
        Неравенство Йенсена. Если функция $\varphi$ выпуклая вниз, то
        $$\varphi\left( \Mean{ \eta \mcond \mathfrak{F}_1 } \right)
            \le \Mean{\varphi\left( \eta \right) \mcond \mathfrak{F}_1}$$
    \item Теорема о трёх перпендикулярах
        $$\mathfrak{F}_2 \subset \mathfrak{F}_1 \Rightarrow
            \Mean{ \mean{\left( \eta \mcond \mathfrak{F}_1 }
                \mcond \mathfrak{F}_2 \right)}
            = \Mean{ \eta \mcond \mathfrak{F}_2 }$$
    \item\label{conditionalExpectationProperty:measurableRandomVariable}
        Если случайная величина $\eta$ измерима
        относительно $\sigma$-алгебры $\mathfrak{F}_1$,
        то её условное математическое ожидание равно ей самой
        $$\Mean{ \eta \mcond \mathfrak{F}_1 } = \eta$$
    \item\label{conditionalExpectationProperty:measurableProduct}
        Если случайная величина $\eta$ измерима
        относительно $\mathfrak{F}_1$, то для любой случайной величины $\xi$
        $$\Mean{ \eta \cdot \xi \mcond \mathfrak{F}_1 }
            = \eta \cdot \Mean{ \xi \mcond \mathfrak{F}_1 }$$
    \item\label{conditionalExpectationProperty:independence}
        Если $\eta$ не зависит от $\mathfrak{F}_1$,
        то её условное математическое ожидание
        равно простому математическому ожиданию
        $$\forall \Delta \in \mathfrak{B}, A \in \mathfrak{F}_1:
            \probability{ \left\{ \eta \in \Delta \right\} \mcond A}
                = \left\{ \eta \in \Delta \right\}
            \Rightarrow \Mean{ \eta \mcond \mathfrak{F}_1 }
                = \mean{\eta}$$
    \item\label{conditionalExpectationProperty:linearity}
        Условное математическое ожидание линейно
        $$\forall a, b \in \mathbb{R}:
            \Mean{a \cdot \xi + b \cdot \eta \mcond \mathfrak{F}_1}
                = \Mean{a \cdot \xi \mcond \mathfrak{F}_1}
                    + \Mean{b \cdot \eta \mcond \mathfrak{F}_1}$$
    \item Сохраняется теорема Лебега о возможности предельного перехода
        под знаком условного математического ожидания
        \cite[стр.~302]{KolmogorovFA}.
        В книге Ширяева это называется теоремой о сходимости
        под знаком условных ожиданий \cite[стр.~272]{Shiryayev1}
        $$\left|\xi_n\right| \le \eta,
            \;\mean{\eta} < \infty,
            \;\xi_n \acovergence \xi
            \Rightarrow
            \Mean{\xi_n \mcond \mathfrak{F}_1}
                \acovergence \Mean{\xi \mcond \mathfrak{F}_1}$$
\end{enumerate}

Пара полезных частных случая неравенства Йенсена
(\ref{conditionalExpectationProperty:Jensen} свойство)
$$\begin{array}{crcl}
    \varphi\left( x \right) = \left|x\right|:&
        \left| \Mean{\eta \mcond \mathfrak{F}_1} \right|
            &\le& \Mean{\left| \eta \right| \mcond \mathfrak{F}_1} \\
    \varphi\left( x \right) = x^2:&
        \left( \Mean{\eta \mcond \mathfrak{F}_1} \right)^2
            &\le& \Mean{\eta^2 \mcond \mathfrak{F}_1}
\end{array}$$

\subsection{Условное математическое ожидание функции
    произвольной случайной величины}\label{conditionalExpectationSubsection}

Чем вызвала интерес эта тема?
Допустим, у нас есть $x_1, \dots, x_n$ --- выборка с функцией правдоподобия $L$
$$L\left( \vec{x}, \theta \right) = \prod_{k=1}^n \pdf{x_k, \theta}$$

Также есть $\hat{\theta}$ --- несмещённая оценка параметра $\theta$.
Как улучшить $\hat{\theta}$?

Возьмём статистику $T = T\left( \vec{x} \right)$,
обладающую определёнными свойствами.
Тогда улучшенной оценкой $\theta$ будет условное математическое ожидание
$\Mean{ \hat{\theta} \mcond T }$.

О свойствах, которыми должна обладать статистика $T$, поговорим позже.
Одно ясно уже сейчас: $T$ является функцией от выборки $\vec{x}$,
как и оценка $\hat{\theta}$.
Это значит, что нам не нужно погружаться в слишком абстрактные размышления,
а достаточно выяснить, как считать математическое ожидание
одной функции выборки (случайного вектора) $f\left( \vec{x} \right)$
при условии другой функции $g\left( \vec{x} \right)$
\textbf{той же} выборки $\vec{x}$.
$$f,g: \mathbb{R}^n \rightarrow \mathbb{R}$$

Вспомним, что для поиска условного математического ожидания
мы находили функцию
$\varphi^{\eta}\left( x \right) = \Mean{ \eta \mcond \xi = x }$.
Тут изменилось совсем немного --- лишь обозначения:
вместо $\eta$ у нас $f\left( \vec{x} \right)$,
а вместо $\xi$ тут $g\left( \vec{x} \right)$.
Значит, нужно найти вид такого условного математического ожидания
$$\Mean{f\left( \vec{x} \right) \mcond g\left( \vec{x} \right) = t} = ?$$

Для начала нужно понять, что из себя представляет множество точек
$S_t = \left\{ \vec{u} \mcond g\left( \vec{u} \right) = t \right\}$.

Очевидно, что функция $g\left( \vec{x} \right)$
описывает скалярное поле в $n$-мерном пространстве.
А для скалярного поля множество $S_t$
имеет своё название --- поверхность уровня (изоповерхность) --- то есть,
поверхность, на которой функция принимает одно и то же значение.

Для понимания ситуации рассмотрим несколько примеров.

\begin{example}
    Имеем двумерное пространство
    $$n=2, g: \mathbb{R}^2 \rightarrow \mathbb{R}$$

    Функция $g\left( x, y \right)$ будет возвращать первую координату
    $$g\left( x, y \right) = x$$

    Очевидно, что поверхности уровней --- вертикальные линии
    (параллельные оси ординат), так как при изменении $y$
    значение функции не меняется
    $$S_t
        = \left\{ \left( x, y \right) \mcond g\left( x, y \right) = t \right\}
        = \left\{ \left( x, y \right) \mcond x = t \right\}$$
\end{example}

\begin{example}
    Опять возьмём двумерное пространство
    $$n=2, g: \mathbb{R}^2 \rightarrow \mathbb{R}$$

    В этот раз функция $g\left( x, y \right)$ будет квадратом расстояния
    от начала координат $\left( 0, 0 \right)$ до точки $\left( x, y \right)$
    $$g\left( x, y \right) = x^2 + y^2$$

    Тут поверхностями уровня будут окружности радиуса $\sqrt{t}$,
    так как окружность по определению является геометрическим местом точек,
    равноудалённых от определённой точки (расстояния до которых одинаковые)
    $$S_t
        = \left\{ \left( x, y \right) \mcond x^2 + y^2 = t \right\}
        = \left\{ \left( x, y \right) \mcond x^2 + y^2 = \sqrt{t}^2 \right\}$$
\end{example}

\begin{example}
    Рассмотрим $n$-мерное пространство ($n \ge 1$).
    Возьмём единичный вектор $\vec{e}: \left\| \vec{e} \right\| = 1$.
    Функция $g$ будет скалярным произведением аргумента $\vec{u}$
    с только что определённым единичным вектором $\vec{e}$
        $$g\left( \vec{u} \right) = \left( \vec{u}, \vec{e} \right)$$

    В таком случае поверхностью уровня $S_t$ будет гиперплоскость,
    проходящая через точку $t \cdot \vec{e}$, с нормалью $\vec{e}$,
    которую описывает следующее уравнение\footnote{
        Уравнение гиперплоскости с нормалью $\vec{e}$,
        проходящую через точку с радиус-вектором $\vec{x}$,
        выглядит следующим образом
        $$\left( \vec{u}, \vec{e} \right) = \left( \vec{x}, \vec{e} \right)$$

        Вследствие линейности скалярного произведения получаем
        $$\left( t \cdot \vec{e}, \vec{e} \right)
            = t \cdot \left( \vec{e}, \vec{e} \right) = t$$

        Значит, указав $\vec{x} = t \cdot \vec{e}$, получим уравнение,
        данное в примере
        $$\left( \vec{u}, \vec{e} \right)
            = \left( \vec{x}, \vec{e} \right)
            = \left( t \cdot \vec{e}, \vec{e} \right)
            = t \cdot \left( \vec{e}, \vec{e} \right) = t$$}

    $$S_t = \left\{ \vec{u} \mcond \left( \vec{u}, \vec{e} \right) = t \right\}$$
\end{example}

В примерах увидели, что получаемые поверхности уровня не имеют объёма
в $n$-мерном пространстве, но у них есть площадь --- $(n-1)$-мерный объём.

Опираясь на предыдущий опыт (для величин с совместной плотностью),
хотелось бы найти совместную плотность случайных величин
$f\left( \vec{x} \right)$ и $g\left( \vec{x} \right)$.
И оказывается, что это желание является верной догадкой.

Нетрудно догадаться, что для того, чтобы найти ``вес'' поверхности уровня,
нужно будет взять поверхностный интеграл от плотности.

Чтобы мы не получали нулевой вес поверхности
$S_t = \left\{ g\left( \vec{x} \right) = t \right\}$,
будем считать объём её раздутия.
Поместим поверхность в своеобразный кокон,
толщина которого в каждой точке будет тем меньше,
чем больше скорость перехода в этой точке от текущего уровня к следующему.

Чтобы значение $t$ было близким к $g\left( \vec{u} \right)$,
нужно, чтобы точка $\vec{u}$ была близка к поверхности $S_t$.
Обозначим расстояние между $t$ и $g\left( \vec{u} \right)$
как $\tilde{\varepsilon}$,
а расстояние между точкой $\vec{u}$ и поверхностью $S_t$
как $\varepsilon$. Чему равны эти расстояния, будет выяснено ниже,
а значение $t$ и точки $\vec{u}$ будет ясно из контекста.

Вероятность того, что значение $g\left( \vec{x} \right)$
отдалено от $t$ не больше, чем на $\tilde{\varepsilon}$,
будет приблизительно равна плотности распределения $g\left( \vec{x} \right)$
в этой точке (если таковая имеется),
умноженной на это расстояние --- погрешность,
которая нас устроит (потом мы, естественно, устремим её к нулю).
Обозначим плотность случайной величины $g\left( \vec{x} \right)$
в точке $t$ через $q\left( t \right)$
$$\Probability{g\left( \vec{x} \right)
    \in \left[t-\tilde{\varepsilon}, t+\tilde{\varepsilon}\right]}
    \approx q\left( t \right) \cdot 2 \cdot \tilde{\varepsilon}$$

Вернёмся к раздутию. Помним, что $\varepsilon$ --- расстояние от
точки $\vec{u}$ до ближайшей к нему точки кокона, а также то,
что это расстояние должно быть обратно пропорционально
стремительности изменения уровней в этой окрестности.
Понимаем, что нам необходима численная мера этой скорости.
Под описание такой величины прекрасно подходит модуль градиента.
Поскольку значение $g\left( \vec{x} \right)$ не меняется вдоль поверхности $S_t$
и равно $t$, то градиент будет направлен по нормали к данной точке поверхности.

Норма градиента --- отношение прироста функции к приросту координат.
Нас интересует прирост координат $\varepsilon$ в окрестности точки $\vec{u}$,
мы располагаем приростом функции $\tilde{\varepsilon}$
и нормой градиента
$\left\| \vec{\nabla} \cdot {g\left( \vec{u} \right)} \right\|$.
Напрашивается формула
\begin{equation}\label{widthEpsilon}
    \varepsilon \approx \frac{\tilde{\varepsilon}}
        {\left\| \vec{\nabla} \cdot g\left( \vec{u} \right) \right\|}
\end{equation}

Обозначим раздутие поверхности $S_t$ как $G_{\tilde{\varepsilon}}$,
тогда вероятность попадания значения $g\left( \vec{x} \right)$ в коридорчик
ширины $\tilde{\varepsilon}$ будет приблизительно равно интегралу
плотности распределения вектора $\vec{x}$ по этому коридорчику

$$\Probability{g\left( \vec{x} \right)
    \in \left[t-\tilde{\varepsilon}, t+\tilde{\varepsilon}\right]}
    \approx q\left( t \right) \cdot 2 \cdot \tilde{\varepsilon}
    \approx \integrall{G_{\tilde{\varepsilon}}}{d\vec{u}}{\pdf{\vec{u}}}$$

Следовательно, у нас почти готова формула для плотности $q\left( t \right)$
случайной величины $g\left( \vec{x} \right)$
$$q\left( t \right) \cdot 2 \cdot \tilde{\varepsilon}
        \approx \integrall{G_{\tilde{\varepsilon}}}{d\vec{u}}{\pdf{\vec{u}}}
        \Rightarrow q\left( t \right)
            \approx \frac{1}{2 \cdot \tilde{\varepsilon}}
                \cdot \integrall{G_{\tilde{\varepsilon}}}{d\vec{u}}{
                    \pdf{\vec{u}}}$$

Чтобы убрать неточность и было обычное равенство,
устремим ширину коридорчика к нулю
\begin{equation}\label{isosurfaceDencity}
    q\left( t \right)
        = \lim_{\tilde{\varepsilon} \to 0} \frac{1}{2 \cdot \tilde{\varepsilon}}
            \cdot \integrall{G_{\tilde{\varepsilon}}}{d\vec{u}}{\pdf{\vec{u}}}
\end{equation}

Распишем $\tilde{\varepsilon}$, воспользовавшись формулой \eqref{widthEpsilon}
$$\varepsilon
    \approx \frac{\tilde{\varepsilon}}
        {\left\| \vec{\nabla} \cdot {g\left( \vec{u} \right)} \right\|}
    \Rightarrow
        \tilde{\varepsilon} \approx \varepsilon
            \cdot \left\| \vec{\nabla} \cdot g\left( \vec{u} \right) \right\|$$

Вернёмся к плотности в формуле \eqref{isosurfaceDencity}.
Заменив $\tilde{\varepsilon}$ на $\varepsilon
\cdot \left\| \vec{\nabla} \cdot g\left( \vec{u} \right) \right\|$,
нужно разобраться, что теперь нужно устремлять к нулю.
Поскольку модуль градиента --- величина, зависящая от координат,
и стремиться к нулю будет лишь при изменении поведения функции,
то устремлять будем $\varepsilon$ (толщину кокона)
$$q\left( t \right)
        = \lim_{\tilde{\varepsilon} \to 0} \frac{1}{2 \cdot \tilde{\varepsilon}}
            \cdot \integrall{G_{\tilde{\varepsilon}}}{d\vec{u}}{\pdf{\vec{u}}}
        = \lim_{\varepsilon \to 0} \frac{1}{2 \cdot \varepsilon
            \cdot \left\| \vec{\nabla} \cdot g\left( \vec{u} \right) \right\|}
            \cdot \integrall{G_{\tilde{\varepsilon}}}{d\vec{u}}{\pdf{\vec{u}}}$$

Поскольку $\varepsilon$ играет роль половины толщины
(отсюда и возникает множитель двойка),
а $d\vec{u}$ --- маленький элемент объёма,
то при делении объёма на толщину получим площадь.
Поскольку толщина стремится к нулю,
то она становится соразмерна с объёмом и мы получаем ненулевое значение площади,
а коридорчик $G_{\tilde{\varepsilon}}$ вырождается в поверхность уровня $S_t$.
Обозначив меру площади на поверхности $S_t$ как
$\sigma_{t}\left( d\vec{u} \right)$,
получаем поверхностный интеграл первого рода

\begin{equation}\label{formula:conditionDencity}
    q\left( t \right)
    = \integrall{S_t}{\sigma_{t}\left(d\vec{u} \right)}{
        \pdf{\vec{u}} \cdot \frac{1}
            {\left\| \vec{\nabla} \cdot g\left( \vec{u} \right) \right\|}}
\end{equation}

А теперь вспомним, как обстояло дело со случайными величинами,
имеющими совместную плотность \eqref{phiIntegral}
$$\varphi^{\eta}\left( x \right)
    = \frac{\integral{\mathbb{R}}{}{y}{y \cdot \pdf{x,y}}}
        {r\left( x \right)}
    = \frac{\integral{\mathbb{R}}{}{y}{y \cdot \pdf{x,y}}}
        {\integral{\mathbb{R}}{}{y}{\pdf{x,y}}}$$

В знаменателе у нас стоял вес поверхности уровня,
раскрыв который, мы получали интеграл от совместной плотности.
Что же есть у нас?
Вес поверхности уровня --- функция одного аргумента $q\left( t \right)$,
которая равна интегралу от плотности по всей той части пространства,
где случайная величина $g\left( \vec{x} \right)$
принимает одно и то же значение $t$ --- по поверхности уровня $S_t$.

То есть, в нашем случае роль $\mathbb{R}$ играет поверхность $S_t$,
роль совместной плотности $\pdf{x,y}$ играет плотность случайного вектора
$\pdf{\vec{u}}$, а вместо дифференциала $dy$ у нас мера площади,
делённая на норму градиента $\frac{\sigma_{t}\left( d\vec{u} \right)}
{\left\| \vec{\nabla} \cdot g\left( \vec{u} \right) \right\|}$.
В числителе дроби в формуле \eqref{phiIntegral} стоит $y$,
в нашем же случае это функция $f\left( \vec{u} \right)$,
поскольку там случайная величина присутствовала в плотности сама по себе,
тут же у нас есть плотность случайного вектора $\pdf{\vec{x}}$,
а найти нужно среднее функции случайной величины $f\left( \vec{x} \right)$.
Итого, получается переход

$$\frac{\integral{\mathbb{R}}{}{y}{y \cdot \pdf{x,y}}}{r\left( x \right)}
    \rightarrow
        \frac{\integrall{S_t}{\frac{\sigma_{t}\left( d\vec{u} \right)}{
            \left\| \vec{\nabla} \cdot g\left( \vec{u} \right) \right\|}}{
            f\left( \vec{u} \right) \cdot \pdf{\vec{u}}}}{q\left( t \right)}$$

И конечная формула

$$\Mean{ f\left( \vec{x} \right) \mcond g\left( \vec{x} \right) = t }
    = \frac{\integrall{S_t}{\sigma_{t}\left(d\vec{u} \right)}{
        f\left( \vec{u} \right) \cdot \pdf{\vec{u}} \cdot \frac{1}{
            \left\| \vec{\nabla} \cdot g\left( \vec{u} \right) \right\|}}}
        {\integrall{S_t}{\sigma_{t}\left(d\vec{u} \right)}{
            \pdf{\vec{u}} \cdot \frac{1}{
                \left\| \vec{\nabla} \cdot g\left( \vec{u} \right) \right\|}}}$$

\begin{theorem}[Условное математическое ожидание гладких функций]
    \index{условное!математическое ожидание!гладких функций}
    \index{математическое ожидание!условное!гладких функций}
    \label{conditionalExpectationDefinition}
    Если есть случайный вектор $\vec{x}$ с известной плотностью распределения
    $\pdf{\vec{x}}$, а также гладкая функция $g\left( \vec{x} \right)$
    с невырожденным градиентом, то математическое ожидание случайной величины
    $f\left( \vec{x} \right)$ при условии $g\left( \vec{x} \right) = t$
    считается по формуле
    $$\Mean{f\left( \vec{x} \right) \mcond g\left( \vec{x} \right) = t}
        = \frac{\integrall{S_t}{\sigma_{t}\left(d\vec{u} \right)}{
            f\left( \vec{u} \right) \cdot \pdf{\vec{u}} \cdot \frac{1}{
                \left\| \vec{\nabla} \cdot g\left( \vec{u} \right) \right\|}}}
            {\integrall{S_t}{\sigma_{t}\left(d\vec{u} \right)}{
                \pdf{\vec{u}} \cdot \frac{1}{
                    \left\| \vec{\nabla}
                        \cdot g\left( \vec{u} \right) \right\|}}}$$
\end{theorem}

\begin{remark}
    Формула остаётся справедливой,
    если поверхности уровня функции $g$ состоят из нескольких гладких кусков.
\end{remark}

\subsection{Пример}
Поскольку этот пример оказался достаточно громоздким,
было решено посвятить ему целый подраздел.

Есть выборка $x_1, \dots, x_n$ из равномерного распределения
с на отрезке $\left[ 0, \theta \right]$.
Нужно посчитать условное математическое ожидание
первого элемента выборки $f\left( \vec{x} \right) = x_1$
при условии максимального $g\left( \vec{x} \right) = x_{\left( n \right)}$

$$\Mean{x_1 \mcond x_{\left( n \right)}} = ?$$

\subsubsection{Поверхности уровня}
Для начала сообразим, что из себя представляет поверхность уровня
$S_t$
$$S_t = \left\{ \vec{u} \mcond
    u_i \ge 0,\; \max_{i=\overline{1,n}} u_i = t \right\}$$

На двумерной плоскости это будет два отрезка,
перпендикулярных друг другу и осям.
Они будут выходить из точки $\left( t, t \right)$
и заканчиваться на осях.

Рассмотрим, почему всё именно так
\begin{enumerate}
    \item Нас не устраивают точки, которые находятся за пределами
        квадрата, ограниченного прямыми, проходящими через точки
        $\left( 0, t \right)$ и $\left( t, 0 \right)$,
        так как в таком случае максимум будет больше $t$,
        что по условию быть не может
    \item Максимальное значение зафиксировано, а это значит,
        что хотя бы одна координата должна
        всегда равняться максимуму
\end{enumerate}

Таким образом, закрасив квадрат, нижняя левая грань которого
находится в начале координат, а верхняя правая в точке $\left( t, t \right)$,
удаляем все те точки, где нет такой координаты, которая равна $t$.
Все точки внутри контура пропадут, так как лишь на контуре квадрата
могут находиться точки, имеющие хоть одну координату равной $t$.
Теперь осталось отбросить те отрезки, что лежат на координатных осях,
потому что на них у точек меняется лишь одна координата
(например, на оси ординат меняется лишь $x$, а $y=0$ на всей оси),
а желаемое значение $t$ принимается лишь на самом конце отрезка,
но это лишь одна точка, и в пространстве размерностью $n \ge 1$
имеет меру Лебега $0$, поэтому её тоже можно отбросить.

Так уж получилось, что только что был выведен
практически универсальный способ построения поверхности уровня
$g\left( \vec{u} \right) = \max_{i=\overline{1,n}} u_i = t$.
Осталось лишь ввести небольшие правки и распространить его на
пространство размерности $n$.

Например, в трёхмерном пространстве поверхностью уровня будет
совокупность граней куба, где исключены те грани,
что соприкасаются с одной из осей --- те грани, одна точка которых
находится в начале координат $\left( 0, 0 \right)$.

Так же будет и в многомерном пространстве --- чертим гиперкуб
и отбрасываем те его грани, один из углов которых находится в начале координат.

Также отметим, что каждая грань перпендикулярна $n-1$ осям,
а пересекается с ними лишь в точках со значением $t$.
Так как нельзя провести две разные гиперплоскости, имеющих одну нормаль
и проходящих через одну точку (гиперплоскость по определению
определяется этой парой), то делаем вывод:
поверхность $S_t$ у нас состоит из $n$ граней.

\subsubsection{Норма градиента}
Найдём норму градиента функции
$g\left( \vec{x} \right) = \max_{i=\overline{1,n}} x_i$.

Возможны два случая:
\begin{enumerate}
    \item Максимальный элемент в векторе один
    \item Максимальных элементов в векторе несколько
        (от двух до $n$) и они равны друг другу (если $n \ge 2$)
\end{enumerate}

Рассмотрим первый случай. Без потери общности предположим,
что максимальный элемент --- первый
$$\max_{i=\overline{1,n}} x_i = x_1$$

Очевидно, что очень малое изменение $x_1$ приведёт к
очень малому (причём такому же) изменению максимального элемента выборки.
Грубо говоря, если у нас есть числа $x_1 = 3, x_2 = 2, x_3 = 1$,
то максимальное из них --- $x_1$. Если оно изменится на $1$
в какую-либо сторону, то максимальное значение выборки изменится
так же на $1$.

Мы это всё рассматриваем для того, чтобы показать,
что производная по максимальной координате будет равна единице,
так как производная --- отношение очень малого прироста функции
к очень малому изменению аргумента (который привёл к такому изменению функции).
Даже если у нас были числа $x_1 = 1$ и $x_2 = 1.001$,
мы всё равно сможем найти такой маленький прирост $\delta < 0.001$
(между двумя разными действительными числами найдётся ещё континуальное число
действительных чисел, поэтому такое $\delta$ найдётся всегда),
при котором $x_2$ останется максимальным элементом и прирост градиента
по этой координате будет равен приросту самой координаты.

Не забываем, что мы сейчас рассматриваем тот случай,
когда максимальное значение одно, а это значит,
что остальные значения выборки строго меньше максимального.

Поскольку очень малые изменения других элементов выборки не приведут к тому,
что они станут максимальными (а если приведут, то возьмём ещё более
маленький прирост), то их изменение не повлияет на значение функции $g$,
а это значит, что частные производные по ним обращаются в нули.

Итого, к чему мы пришли.
Когда $x_1$ является максимальным элементом выборки,
то градиент функции $g$ в точке $\vec{x}$ равен
$$\vec{\nabla} \cdot g\left( \vec{x} \right)
    = \frac{\partial g}{\partial x_1} \cdot \vec{e_1}
        + \frac{\partial g}{\partial x_2} \cdot \vec{e_2}
        + \cdots
        + \frac{\partial g}{\partial x_n} \cdot \vec{e_n}
    = \vec{e_1} + 0 + \cdots + 0 = \vec{e_1}$$

Когда у нас не $x_1$ является максимальным элементом, а $x_k$,
где $1 \le k \le n$, то очевидно, что результат будет следующим:
$$\vec{\nabla} \cdot g\left( \vec{x} \right) = \vec{e_k}$$

Нас интересует лишь норма градиента, которая в данном случае равна единице
$$\left\| \vec{\nabla} \cdot g\left( \vec{x} \right) \right\|
    = \left\| \vec{e_k} \right\| = 1$$

Второй случай (когда несколько элементов максимальны и равны между собой)
нас не интересует, так как такое возможно лишь на рёбрах ($n \ge 3$) и в точках
пересечения рёбер ($n \ge 2$).
Что те, что другие (прямые и точки в пространстве) имеют нулевой объём.
В одномерном пространстве (с одной осью) у нас есть лишь одна
случайная величина, и она же является максимальной.

\subsubsection{Условное математическое ожидание относительно события}
Посчитаем условное математическое ожидание по формуле
\begin{align*}
    \Mean{x_1 \mcond x_{\left( n \right)} = t}
        = \frac{\integrall{S_t}{
                    \sigma_{t}\left(d\vec{u} \right)}{
                    u_1 \cdot \theta^{-n} \cdot \frac{1}{1}}}
                {\integrall{S_t}{\sigma_{t}\left(d\vec{u} \right)}{
                    \theta^{-n} \cdot \frac{1}{1}}}
        = \frac{\integrall{S_t}{
                    \sigma_{t}\left(d\vec{u} \right)}{u_1}}{
                \integrall{S_t}{
                    \sigma_{t}\left(d\vec{u} \right)}{}}
\end{align*}

Поверхность уровня $S_t$ в верхнем интеграле можно (и нужно!)
разбить на две части: ту, где $u_1$ меняется, и ту, где $u_1$
принимает постоянное значение.

Вспоминаем, что поверхность $S_t$ --- часть гиперкуба,
содержащая лишь грани, на которых которых та случайная величина,
что не меняется, имеет значение $t$ (максимум).
Частью поверхности, на которой $u_1$ принимает постоянное значение $t$,
будет одна грань --- та грань гиперкуба, что проходит через точку $u_1 = t$
и перпендикулярна оси (вектору $\vec{e_1}$).

Обозначим грань, где $u_1 = t$, как $U_t$,
а оставшуюся часть поверхности как $Y_t = S_t \setminus U_t$.

Имеем более конкретную формулу для подсчёта условного математического ожидания
\begin{align*}
    \Mean{x_1 \mcond x_{\left( n \right)} = t}
    = \frac{\integrall{S_t}{\sigma_{t}\left(d\vec{u} \right)}{u_1}}{
        \integrall{S_t}{\sigma_{t}\left(d\vec{u} \right)}{}}
    = \frac{\integrall{U_t}{\sigma_{t}\left(d\vec{u} \right)}{
                \Bigl. u_1 \Bigr|_{u_1 = t}}
            + \integrall{Y_t}{\sigma_{t}\left(d\vec{u} \right)}{
                u_1}}{
        \integrall{S_t}{\sigma_{t}\left(d\vec{u} \right)}{}}
\end{align*}

Вспоминаем, что имеется $n$ граней, а это значит, что в знаменателе
у нас $n$ объёмов $\left( n-1 \right)$-мерных гиперкубов
($n$ площадей $n$-мерных квадратов)
со стороной $t$, которые в свою очередь равняются числу $t^{n-1}$.

В числителе у нас два интеграла.
Первый интеграл --- интеграл по одной грани, что опять же является
$\left( n-1 \right)$-мерным объёмом ($n$-мерной площадью).
Не забываем про константу $t$, что там находится: она умножается
на результат интегрирования $t^{n-1}$ и в результате получаем $t^n$.

Теперь подошли к самому сложному кусочку этой дроби ---
интеграл по оставшимся $n-1$ граням.
Поскольку интеграл не зависит ни от чего кроме $u_1$, то мы преспокойнейше
можем вынести $\left( n-2 \right)$-мерный объём, а по оставшемуся измерению
придётся интегрировать от $0$ до $t$
$$\frac{\integrall{U_t}{\sigma_{t}\left(d\vec{u} \right)}{
        \Bigl. u_1 \Bigr|_{u_1 = t}}
            + \integrall{Y_t}{\sigma_{t}\left(d\vec{u} \right)}{
                u_1}}{
        \integrall{S_t}{\sigma_{t}\left(d\vec{u} \right)}{}}
    = \frac{t^n + \left( n-1 \right) \cdot t^{n-2}
        \cdot \integral{0}{t}{u_1}{u_1}}{n \cdot t^{n-1}}$$

Дальше идут нехитрые математические преобразования,
которые называются интегрированием
\begin{align*}
    \Mean{x_1 \mcond x_{\left( n \right)} = t}
    = \frac{t^n + \left( n-1 \right) \cdot t^{n-2} 
        \cdot \integral{0}{t}{u_1}{u_1}}{n \cdot t^{n-1}} = \\
    = \frac{t^n + \left( n-1 \right) \cdot t^{n-2}
        \cdot \frac{t^2}{2}}{n \cdot t^{n-1}}
    = \frac{t + \left( n-1 \right) \cdot t \cdot \frac{1}{2}}{n}
    = t \cdot \frac{n+1}{2 \cdot n}
\end{align*}

\subsubsection{Условное математическое ожидание}
Мы получили условное математическое ожидание относительно события
$\left\{ g\left( x \right) = t \right\}$ --- конкретное значение
$t \cdot \frac{n+1}{2 \cdot n}$.
Чтобы получить условное математическое ожидание одной случайной величины
относительно другой, нужно лишь подставить вместо $t$
наше значение $g\left( \vec{x} \right) = x_{\left( n \right)}$
$$\Mean{x_1 \mcond x_{\left( n \right)}}
    = \Bigl. \Mean{x_1 \mcond x_{\left( n \right)} = t}
        \Bigr|_{t = x_{\left( n \right)}}
    = \Bigl. t \cdot \frac{n+1}{2 \cdot n} \Bigr|_{t = x_{\left( n \right)}}
    = x_{\left( n \right)} \cdot \frac{n+1}{2 \cdot n}$$

Видим, что получили случайную величину,которая не зависит от параметра
$\theta$, а это значит, что мы на правильном пути
$$\Mean{x_1 \mcond x_{\left( n \right)}}
    = \frac{n+1}{2 \cdot n} \cdot x_{\left( n \right)}$$

\subsubsection{Проверка}
Проведём небольшую очевидную проверку: положим $n=1$ (одна случайная величина,
одномерное пространство). Тогда формула примет следующий вид
$$\Mean{x_1 \mcond x_{\left( 1 \right)}}
    = \frac{1+1}{2 \cdot 1} \cdot x_{\left( 1 \right)}
    = \frac{2}{2} \cdot x_{\left( 1 \right)}
    = x_{\left( 1 \right)} = x_1$$

Всё сходится с нашими интуитивными предположениями: у нас имеется лишь один
элемент в выборке, а мы знаем значение максимального. Значит, мы знаем
значение этого единственного элемента (иначе кому ещё быть максимальным
в этой выборке?).
