\section{Условные распределения}
\index{условное!распределение}

\begin{definition}[Условное распределение]
  Условное распределение случайной величины $\xi$
  при известной $\sigma$-алгебре $\mathfrak{F}_1$ --- это функция $\pi$
  $$\pi: \Omega \times \mathfrak{B} \rightarrow \left[ 0, 1 \right]$$
  
  Функция $\pi$ должна обладать следующими свойствами
  \begin{enumerate}
      \item На любом элементе $\Delta$ борелевского множества $\mathfrak{B}$
      функция $\pi\left( \cdot, \Delta \right)$ является измеримой
      относительно $\sigma$-алгебры $\mathfrak{F}_1$
      \item На любом элементарном исходе из множества $\Omega$
      функция $\pi\left( \omega, \cdot \right)$
      является вероятностной мерой
      \item Выполняется равенство
      $$\forall \Delta \in \mathfrak{B}: \pi\left( \cdot, \Delta \right)
      = \Mean{\Indicator{\xi \in \Delta} \mcond \mathfrak{F}_1}$$

      Это равенство нам уже знакомо, поэтому ничего принципиально нового
      не добавилось
      $$\probability{\xi \in \Delta} = \mean{\Indicator{\xi \in \Delta}}$$
  \end{enumerate}

  Обозначение
  $$\pi\left( \cdot, \Delta \right)
      = \probability{\xi \in \Delta \mcond \mathfrak{F}_1}$$

  Если $\sigma$-алгебра $\mathfrak{F}_1$ порождена случайной величиной
  $\eta$ ($\mathfrak{F}_1 = \sigma\left( \eta \right)$), будем использовать
  следующее обозначение
  $$\probability{\xi \in \Delta \mcond \sigma\left( \eta \right)}
      = \pdf{\eta, \Delta}$$

  Когда нас интересует событие $\eta = t$, обозначаем это так
  $$\probability{\xi \in \Delta \mcond \eta = t} = \pdf{t, \Delta}$$

  Связь с условным математическим ожиданием
  $$\Mean{f\left( \xi \right) \mcond \eta = t}
      = \integrall{\mathbb{R}}{\pdf{t, du}}{f\left( u \right)}$$
\end{definition}

\begin{remark}
  В обозначениях выше точка вместо аргумента означает,
  что на выходе мы получаем не определённое значение,
  а функцию от того аргумента, который заменён точкой.

  Например, запись $\pi\left( \cdot, \Delta \right)$
  означает некую функцию $\rho$
      $$\rho: \Omega \rightarrow \left[ 0, 1 \right]$$

  Значение этой функции будет считаться по формуле
      $$\rho\left( \omega \right) = \pi\left( \omega, \Delta \right)$$
\end{remark}

Рассмотрим примеры вычисления условных распределений

\begin{example}[См. пример \ref{discreteConditionalExpectationExample}]
  Случайные величины $\xi$ и $\eta$ имеют совместное дискретное распределение
  $$\Probability{\xi = a_i, \eta = b_j} = p_{ij}$$

  В таком случае условное распределение считается по формуле
  \begin{align*}
      \Probability{\xi = a_i \mcond \eta = b_j}
      = \frac{p_{ij}}{\sum_i p_{ij}}
  \end{align*}
\end{example}

\begin{example}[См. формулу \ref{phiIntegral}]
  Случайные величины $\xi$ и $\eta$ имеют
  совместную плотность распределения $\pdf{x,y}$
  \begin{align*}
      \Probability{\xi \in \Delta \mcond \eta = x}
      = \frac{\integral{\Delta}{}{y}{\pdf{x,y}}}
      {\integral{\mathbb{R}}{}{y}{\pdf{x,y}}}
  \end{align*}
\end{example}

\begin{example}[См. теорему \ref{conditionalExpectationDefinition}]
  У случайного вектора $\vec{x}$ есть плотность распределения $\pdf{\vec{u}}$.
  Тогда условное распределение $f\left( \vec{x} \right)$ относительно
  гладкой функции $g\left( \vec{x} \right)$ считается по формуле
  $$\probability{f\left( \vec{x} \right) \in \Delta
      \mcond g\left( \vec{x} \right) = t}
      = \frac{\integrall{S_t \cap \Delta}{\sigma_{t}\left(d\vec{u} \right)}{
      \pdf{\vec{u}} \cdot \frac{1}{
      \left\| \vec{\nabla} \cdot g\left( \vec{u} \right) \right\|}}}
      {\integrall{S_t}{\sigma_{t}\left(d\vec{u} \right)}{
      \pdf{\vec{u}} \cdot \frac{1}{
    \left\| \vec{\nabla}
        \cdot g\left( \vec{u} \right) \right\|}}}$$
\end{example}

\section{Достаточные статистики}
Как говорилось в подразделе \ref{conditionalExpectationSubsection},
условное математическое ожидание нам понадобилось из-за наличия
статистик $T$, обладающих особыми свойствами.
Пришло время о них поговорить

\begin{definition}[Достаточная статистика]\index{статистика!достаточная}
  Статистика $T\left( x_1, x_2, \dots, x_n \right)$ --- достаточная статистика
  для параметра $\theta$, если условное распределение выборки
  при известном $T$ не зависит от $\theta$
\end{definition}

Осмыслим написанное
\begin{enumerate}
  \item Речь идёт об условном распределении всей выборки
      $$\pi\left( T, \Delta \right)
      = \probability{\vec{x} \in \Delta \mcond T}$$
  \item Почему возникает определение достаточных статистик?
      Пусть $T$ --- достаточная статистика.
      Как с её помощью получить распределение всей выборки?
      $$\Probability{\vec{x} \in \Delta}
      = \mean{\Indicator{\vec{x} \in \Delta}}$$

      Далее воспользуемся
      \ref{conditionalExpectationProperty:totalProbability}
      свойством условного математического ожидания
      (формула полной вероятности)
      $$\mean{\Indicator{\vec{x} \in \Delta}}
      = \mean{\Mean{\Indicator{\vec{x} \in \Delta} \mcond T}}$$

      Из определения условного математического ожидания и утверждения
      \ref{measurableRandomVariable} следует, что для некоторой измеримой
      функции $f$ справедливо следующее равенство
      $$\Mean{\Indicator{\vec{x} \in \Delta \mcond T}} = f\left( T \right)$$

      Тогда получаем
      $$\Probability{\vec{x} \in \Delta} = \mean{f\left( T \right)}$$
\end{enumerate}

\begin{theorem}[Об улучшении оценки с помощью достаточной статистики]
  \index{оценка!улучшенная}
  \index{теорема!об улучшении!оценки с помощью достаточной статистики}
  Пусть \xsample --- выборка из распределения $\cdfof{\theta}{x}$,
  $\theta \in \Theta$.
  Есть $T$ --- достаточная статистика для параметра $\theta$,
  а также несмещённая оценка $\hat{\theta}$ параметра $\theta$.
  Введём оценку $\theta_*$
  $$\theta_* = \Mean{\hat{\theta} \mcond T}$$

  Оценка $\theta_*$ не хуже, чем оценка $\hat{\theta}$, в среднем квадратическом
  $$\begin{cases}
      \meanof{\theta}{\theta_*} = \theta \\
      \dispersionof{\theta}{\theta_*} \le \dispersionof{\theta}{\hat{\theta}}
  \end{cases}$$
\end{theorem}

\begin{remark}
  Оценка $\theta_* = \Mean{\hat{\theta} \mcond T}$ не зависит от $\theta$,
  так как $T$ --- достаточная статистика
\end{remark}

\begin{remark}
  ``Как правило'', одномерная достаточная статистика
  для одномерного параметра даёт не улучшаемую статистику
\end{remark}

\begin{proof}
  \begin{enumerate}
      \item С первым пунктом всё просто:
      пользуемся формулой полной вероятности
      (\ref{conditionalExpectationProperty:totalProbability} свойство)
      $$\meanof{\theta}{\theta_*}
      = \meanof{\theta}{\Mean{\hat{\theta} \mcond T}}
      = \meanof{\theta}{\hat{\theta}}
      = \theta$$
      \item Для доказательства второго утверждения распишем дисперсию и оценки
        $\theta_*$ по определению
      $$\dispersionof{\theta}{\theta_*}
      = \meanof{\theta}{\left( \theta_* - \theta \right)^2}
      = \meanof{\theta}{\left( \Mean{\hat{\theta} \mcond T}
    - \theta \right)^2}$$

      Поскольку $T$ --- достаточная статистика и не зависит от $\theta$,
      то $\theta$ измерима относительно $\sigma\left( T \right)$
      и является константой. Значит, можно переписать
      статистику $\theta$ как условное математическое ожидание
      (\ref{conditionalExpectationProperty:measurableRandomVariable}
      свойство), а затем воспользоваться линейностью
      условного математического ожидания
      (\ref{conditionalExpectationProperty:linearity} свойство)
      \begin{align*}
      \meanof{\theta}{\left( \Mean{\hat{\theta} \mcond T}
        - \theta \right)^2}
    = \meanof{\theta}{\left( \Mean{\hat{\theta} \mcond T}}
        - \Mean{\theta \mcond T} \right)^2 = \\
    = \meanof{\theta}{\left(
        \Mean{\left( \hat{\theta} - \theta \right) \mcond T}
        \right)^2}
      \end{align*}

      Дальше воспользуемся неравенством Йенсена
      (\ref{conditionalExpectationProperty:Jensen} свойство)
      и формулой полной вероятности
      (\ref{conditionalExpectationProperty:totalProbability} свойство)
      \begin{align*}
      \meanof{\theta}{\left( \Mean{\left( \hat{\theta}
        - \theta \right) \mcond T}\right)^2}
    \le \meanof{\theta}{\Mean{\left( \hat{\theta}
        - \theta \right)^2 \mcond T}} = \\
    = \meanof{\theta}{\left( \hat{\theta} -\theta \right)^2}
    = \dispersionof{\theta}{\hat{\theta}}
      \end{align*}
  \end{enumerate}
\end{proof}

\begin{remark}
  Равенство в неравенстве Йенсена выше возможно,
  когда условное распределение вырождается в одну точку.
  Когда условное распределение не вырождено,
  то неравенство оказывается строгим.
\end{remark}

\begin{remark}
  Оценка $\theta_* = \Mean{\hat{\theta} \mcond T}$ измерима относительно $T$
  по определению условного математического ожидания, а это значит,
  что она является функцией от $T$.

  Пусть $\tilde{\theta}$ --- оптимальная несмещённая оценка.
  Тогда $\theta_* = \Mean{\hat{\theta} \mcond T}$ --- оптимальная, а значит,
  эти оценки равны $\theta_* = \tilde{\theta}$ по теореме единственности
  (теорема Колмогорова \ref{theorem:Kolmogorov}),
  поскольку оптимальная оценка либо одна, либо не существует вовсе.
  Значит, оптимальная оценка --- функция достаточной статистики.
\end{remark}

\begin{theorem}[Факторизационная теорема
      (о характеризации достаточной статистики)]
  \index{теорема!факторизационная}
  \index{теорема!о характеризации!достаточной статистики}
  Пусть \xsample --- выборка из распределения
  с плотностью $\pdf{x,\theta}$, $\theta \in \Theta$.

  Статистика $T$ является достаточной тогда и только тогда, когда
  функция правдоподобия $L\left( \vec{x}, \theta \right)$
  допускает факторизацию, то есть может быть представлена
  произведением двух функций следующего вида
  \index{функция!правдоподобия!факторизация}
  $$L\left( \vec{x}, \theta \right)
      = h\left( T, \theta \right) \cdot g\left( \vec{x} \right)$$
\end{theorem}

\begin{remark}
  Также с теоремой и её доказательствами можно ознакомиться в источниках
  \cite[стр.~78]{MGTUXVII}, \cite[стр.~158]{BorovkovMS}.
\end{remark}

\begin{proof}[Наброски доказательства для гладкой функции правдоподобия]
  Условное распределение выборки при известной статистике
  $T=f\left( \vec{x} \right)$ определяется формулой
  \begin{equation}\label{formula:factorizationIntegral}
      \probability{\vec{x} \in \Delta \mcond T = t}
      = \frac{\integrall{S_t \cap \Delta}{\sigma_{t}\left(d\vec{u} \right)}{
      L\left( \vec{u}, \theta \right) \cdot \frac{1}{
      \left\| \vec{\nabla} \cdot f\left( \vec{u} \right) \right\|}}}
      {\integrall{S_t}{\sigma_{t}\left(d\vec{u} \right)}{
      L\left( \vec{u}, \theta \right) \cdot \frac{1}{
    \left\| \vec{\nabla}
        \cdot f\left( \vec{u} \right) \right\|}}}
  \end{equation}
  \begin{enumerate}
      \item[Достаточность]
      Пусть функция правдоподобия допускает факторизацию,
      то есть существуют такие функции $h\left( T, \theta \right)$
      и $g\left( \vec{x} \right)$, что
      $$L\left( \vec{x}, \theta \right)
      = h\left( T, \theta \right) \cdot g\left( \vec{x} \right)$$

      Тогда интеграл \eqref{formula:factorizationIntegral}
      примет следующий вид
      $$\probability{\vec{x} \in \Delta \mcond T = t}
      = \frac{\integrall{S_t \cap \Delta}{
    \sigma_{t}\left(d\vec{u} \right)}{
        h\left( T, \theta \right) \cdot g\left( \vec{u} \right)
    \cdot \frac{1}{\left\| \vec{\nabla}
        \cdot f\left( \vec{u} \right) \right\|}}}
    {\integrall{S_t}{\sigma_{t}\left(d\vec{u} \right)}{
        h\left( T, \theta \right) \cdot g\left( \vec{u} \right)
    \cdot \frac{1}{\left\| \vec{\nabla}
        \cdot f\left( \vec{u} \right) \right\|}}}$$

      Хоть $T$ и является функцией выборки, мы зафиксировали его значение,
      а $\theta$ является константой.
      Это значит, что функция $h\left( T, \theta \right)$ тоже не зависит
      от $\vec{u}$, поэтому сверху и снизу её можно сократить, избавив
      зависимость распределения от параметра $\theta$
      $$\probability{\vec{x} \in \Delta \mcond T = t}
      = \frac{\integrall{S_t \cap \Delta}{
    \sigma_{t}\left(d\vec{u} \right)}{
        g\left( \vec{u} \right)
    \cdot \frac{1}{\left\| \vec{\nabla}
        \cdot f\left( \vec{u} \right) \right\|}}}
    {\integrall{S_t}{\sigma_{t}\left(d\vec{u} \right)}{
        g\left( \vec{u} \right)
    \cdot \frac{1}{\left\| \vec{\nabla}
        \cdot f\left( \vec{u} \right) \right\|}}}$$

      Условное распределение $\vec{x}$ не зависит от $\theta$
      при известном $T$, что и требовалось доказать для того, чтобы
      показать достаточность факторизации.

      \item[Необходимость]
      Пусть $T$ --- достаточная статистика.
      Выпишем плотность распределения $T$ в точке $t$ согласно формуле
      \eqref{formula:conditionDencity}, но с небольшими поправками:
      плотность распределения выборки $\vec{x}$ заменяется
      функцией правдоподобия, а плотность $T$ будет зависеть не только
      от $t$, но и от параметра $\theta$.
      $$q\left( t, \theta \right)
      = \integrall{S_t}{\sigma_{t}\left(d\vec{u} \right)}{
    L\left( \vec{u}, \theta \right) \cdot \frac{1}{
        \left\| \vec{\nabla}
    \cdot f\left( \vec{u} \right) \right\|}}$$

      Поверхностная мера $\sigma_{t} \left( \vec{u} \right)$ не зависит
      от параметра $\theta$, а это значит, что можно поделить
      обе части равенства на $q\left( t, \theta \right)$ и внести
      плотность под знак интеграла.
      $$\integrall{S_t}{\sigma_{t}\left(d\vec{u} \right)}{
      \frac{1}{q\left( t, \theta \right)}
      \cdot L\left( \vec{u}, \theta \right) \cdot \frac{1}{
    \left\| \vec{\nabla}
        \cdot f\left( \vec{u} \right) \right\|}} = 1$$

      Область интегрирования и мера не зависят от $\theta$, а это значит,
      что и подынтегральное выражение тоже не зависит от $\theta$,
      а зависит лишь от вектора $\vec{u}$
      $$\frac{1}{q\left( t, \theta \right)}
      \cdot L\left( \vec{u}, \theta \right) \cdot \frac{1}{
    \left\| \vec{\nabla} \cdot f\left( \vec{u} \right) \right\|}
      = c\left( \vec{u} \right)$$

      Значит, функцию правдоподобия можно представить в следующем виде
      $$L\left( \vec{u}, \theta \right)
      = q\left( t, \theta \right)
    \cdot \left\| \vec{\nabla}
        \cdot f\left( \vec{u} \right) \right\|
        \cdot c\left( \vec{u} \right)$$

      И тут мы видим, что это и есть факторизация!

      Для начала вспомним, что у нас распределение при условии $T=t$,
      а это значит, что плотность $q\left( t, \theta \right)$
      может быть расписана следующим образом
      $$q\left( t, \theta \right)
      = q\left( T, \theta \right)$$

      Теперь, выделив функции $g$ и $h$, получаем необходимый результат
      $$\begin{cases}
      L\left( \vec{u}, \theta \right)
    = q\left( t, \theta \right)
        \cdot \left\| \vec{\nabla}
    \cdot f\left( \vec{u} \right) \right\|
    \cdot c\left( \vec{u} \right) \\
      h\left( T, \theta \right)
    = q\left( T, \theta \right)
    = q\left( t, \theta \right) \\
      g\left( \vec{u} \right)
    = \left\| \vec{\nabla}
    \cdot f\left( \vec{u} \right) \right\|
        \cdot c\left( \vec{u} \right)
      \end{cases}
      \Rightarrow
      L\left( \vec{u}, \theta \right)
    = g\left( \vec{u} \right) \cdot h\left( T, \theta \right)$$

     То есть чтобы $T$ было достаточной статистикой, необходимо, чтобы
     функция правдоподобия допускала факторизацию.
  \end{enumerate}
\end{proof}

\begin{remark}
  Теорема остаётся справедливой для дискретных распределений,
  где функция правдоподобия выглядит следующим образом
  $$L\left( \vec{u}, \theta \right) = \prod_{k=1}^n \Probability{x_k = u_k}$$
\end{remark}

\begin{example}
  \xsample --- выборка из нормального распределения с неизвестным
  математическим ожиданием $N\left( \theta, 1 \right), \theta \in \mathbb{R}$.

  Рассмотрим функцию правдоподобия
  \begin{equation*}
    L\left( \vec{x}, \theta \right)
    = \frac{1}{\sqrt{2 \cdot \pi}^n}
      \cdot \exp{\left\{ -\frac{1}{2}
      \cdot \sum_{k=1}^n \left( x_k - \theta \right)^2 \right\}}
  \end{equation*}
  Рассмотрим сумму под экспонентой и выделим из неё выборочное среднее
  \begin{align*}
      \sum_{k=1}^n \left( x_k - \theta \right)^2
      = \sum_{k=1}^n x_k^2 - 2 \cdot \theta \cdot \sum_{k=1}^n x_k
      + n \cdot \theta^2 = \\
      = \sum_{k=1}^n x_k^2 - 2 \cdot \theta \cdot n \cdot \overline{x}
      + n \cdot \theta^2
  \end{align*}

  Теперь выделим квадрат разности выборочного среднего и параметра $\theta$
  \begin{align*}
      \sum_{k=1}^n x_k^2 - 2 \cdot \theta \cdot n \cdot \overline{x}
      + n \cdot \theta^2 = \\
      = \sum_{k=1}^n x_k^2 - 2 \cdot \theta \cdot n \cdot \overline{x}
      + n \cdot \theta^2 + n \cdot \overline{x}^2
      - n \cdot \overline{x}^2 = \\
      = \sum_{k=1}^n \left( x_k^2 - \overline{x}^2 \right)
      + n \cdot \left( \overline{x} - \theta \right)^2
  \end{align*}

  Вернёмся к исходному выражению и видим экспоненту суммы.
  Заменим её на произведение экспонент
  \begin{align*}
      \frac{1}{\sqrt{2 \cdot \pi}^n} \cdot \exp{\left\{ -\frac{1}{2}
      \cdot \sum_{k=1}^n \left( x_k - \theta \right)^2 \right\}} = \\
      = \frac{1}{\sqrt{2 \cdot \pi}^n} \cdot \exp{\left\{ -\frac{1}{2}
      \cdot \sum_{k=1}^n \left( x_k^2 - \overline{x}^2 \right)
    + n \cdot \left( \overline{x} - \theta \right)^2
    \right\}} = \\
      = \frac{1}{\sqrt{2 \cdot \pi}^n}
      \cdot \exp{\left\{ -\frac{1}{2}
    \cdot \sum_{k=1}^n \left( x_k^2 - \overline{x}^2 \right)
        \right\}}
      \cdot \exp{\left\{ n \cdot \left( \overline{x}
    - \theta \right)^2 \right\}}
  \end{align*}

  Мы получили произведение двух функций:
  одна зависит лишь от статистики $\overline{x}$ и параметра $\theta$,
  а другая, зависит лишь от выборки (так как выборочное средние
  является функцией выборки)
  \begin{align*}
      &\begin{cases}
      L\left( \vec{x}, \theta \right)
      = \frac{1}{\sqrt{2 \cdot \pi}^n}
      \cdot \exp{\left\{ -\frac{1}{2}
    \cdot \sum_{k=1}^n \left( x_k^2 - \overline{x}^2 \right)
        \right\}}
      \cdot \exp{\left\{ n \cdot \left( \overline{x}
    - \theta \right)^2 \right\}} \\
      h\left( \overline{x}, \theta \right)
      = \exp{\left\{ n \cdot \left( \overline{x}
    - \theta \right)^2 \right\}}\\
      g\left( \vec{x} \right)
      = \frac{1}{\sqrt{2 \cdot \pi}^n}
      \cdot \exp{\left\{ -\frac{1}{2}
    \cdot \sum_{k=1}^n \left( x_k^2 - \overline{x}^2 \right)
        \right\}}
      \end{cases} \\
      &\Rightarrow
      L\left( \vec{x}, \theta \right)
      = g\left( \vec{x} \right) \cdot h\left( T, \theta \right)
  \end{align*}

  Функция выборки допускает факторизацию со статистикой $\overline{x}$,
  а это значит, что выборочное среднее является достаточной статистикой.
\end{example}
