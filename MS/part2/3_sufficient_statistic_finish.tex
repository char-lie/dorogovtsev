\section{Условные распределения}
\index{условное!распределение}

\begin{definition}[Условное распределение]
    Условное распределение случайной величины $\xi$
    при известной $\sigma$-алгебре $\mathfrak{F}_1$ --- это функция $\pi$
    $$\pi: \Omega \times \mathfrak{B} \rightarrow \left[ 0, 1 \right]$$
    
    Функция $\pi$ должна обладать следующими свойствами
    \begin{enumerate}
        \item На любом элементе $\Delta$ борелевского множества $\Delta$
            функция $\pi\left( \cdot, \Delta \right)$ является измеримой
            относительно $\mathfrak{F}_1$
        \item На любом элементарном исходе из множества $\Omega$
            функция $\pi\left( \omega, \cdot \right)$
            является вероятностной мерой
        \item Выполняется равенство
            $$\forall \Delta \in \mathfrak{B}: \pi\left( \cdot, \Delta \right)
                = \Mean{\Indicator{\xi \in \Delta} \mid \mathfrak{F}_1}$$
    \end{enumerate}

    Обозначение
    $$\pi\left( \cdot, \Delta \right)
        = \probability{\xi \in \Delta \mid \mathfrak{F}_1}$$

    Если же $\sigma$-алгебра $\mathfrak{F}_1$ порождена
    случайной величиной $\eta$: $\mathfrak{F}_1 = \sigma\left( \eta \right)$,
    работает следующее обозначение
    $$\probability{\xi \in \Delta \mid \sigma\left( \eta \right)}
        = \pdf{\eta, \Delta}$$

    Когда нас интересует событие $\eta = t$, работает следующее обозначение
    $$\probability{t, \Delta} = \pdf{\xi \in \Delta \mid \eta = t}$$

    Связь с условным математическим ожиданием
    $$\Mean{f\left( \xi \right) \mid \eta = t}
        = \integrall{\mathbb{R}}{\pdf{t, du}}{f\left( u \right)}$$
\end{definition}

\begin{remark}
    В обозначениях выше точка вместо аргумента означает,
    что на выходе мы получаем не определённое значение,
    а функцию от того аргумента, который заменён точкой.

    Например, запись $\pi\left( \cdot, \Delta \right)$
    означает некую функцию $\rho$
        $$\rho: \Omega \rightarrow \left[ 0, 1 \right]$$

    Значение этой функции будет считаться по формуле
        $$\rho\left( \omega \right) = \pi\left( \omega, \Delta \right)$$
\end{remark}

Рассмотрим примеры вычисления условных распределений

\begin{example}[См. пример \eqref{discreteConditionalExpectationExample}]
    Случайные величины $\xi$ и $\eta$ имеют совместное дискретное распределение
    $$\Probability{\xi = a_i, \eta = b_j} = p_{ij}$$

    В таком случае условное распределение считается по формуле
    $$\Probability{\xi = a_i, \eta = b_j} = \frac{p_{ij}}{\sum_j p_{ij}}$$
\end{example}

\begin{example}[См. формулу \eqref{phiIntegral}]
    Случайные величины $\xi$ и $\eta$ имеют
    совместную плотность распределения $\pdf{x,y}$
        $$\frac{\integral{\Delta}{}{y}{y \cdot \pdf{x,y}}}
            {\integral{\mathbb{R}}{}{y}{\pdf{x,y}}}$$
\end{example}

\begin{example}[См. теорему \eqref{conditionalExpectationDefinition}]
    У случайного вектора $\vec{x}$ есть плотность распределения $\pdf{\vec{u}}$.
    Тогда условное распределение $f\left( \vec{x} \right)$ относительно
    гладкой функции $g\left( \vec{x} \right)$ считается по формуле
    $$\probability{f\left( \vec{x} \right) \in \Delta
        \mid g\left( \vec{x} \right) = t}
        = \frac{\integrall{S_t \cap \Delta}{\sigma_{t}\left(d\vec{u} \right)}{
            \pdf{\vec{u}} \cdot \frac{1}{
                \left\| \vec{\nabla} \cdot g\left( \vec{u} \right) \right\|}}}
            {\integrall{S_t}{\sigma_{t}\left(d\vec{u} \right)}{
                \pdf{\vec{u}} \cdot \frac{1}{
                    \left\| \vec{\nabla}
                        \cdot g\left( \vec{u} \right) \right\|}}}$$
\end{example}
