\chapter{Достаточные статистики}
\section{Оптимальная оценка}
\begin{definition}[Симметризация]\index{симметризация}
    Симметризация $\Lambda$ оценки $\hat{\theta}$ --- среднее
    оценок $\hat{\theta}$ для
    всевозможных перестановок $\sigma\in S_n$
    элементов выборки $x_1, x_2, \dots, x_n$
    $$\Lambda\hat{\theta}
        =\frac{1}{n!}\cdot\sum_{\sigma\in S_n} \hat{\theta}\left(
            x_{\sigma\left(1\right)}, x_{\sigma\left(2\right)},
                \dots, x_{\sigma\left(n\right)}\right)$$
\end{definition}
\begin{lemma}
    Для произвольной несмещённой оценки $\hat{\theta}$
    её симметризация $\Lambda{\hat{\theta}}$
    не хуже её самой в среднем квадратическом
    \begin{align*}
    \meanof{\theta}{\hat{\theta}}
        =\theta
    \Rightarrow
        \begin{cases}
            \meanof{\theta}{\Lambda{\hat{\theta}}}
                =\meanof{\theta}{\hat{\theta}}
                =\theta\\
            \dispersionof{\theta}{\Lambda{\hat{\theta}}}
                \le\dispersionof{\theta}{\hat{\theta}}
        \end{cases}
    \end{align*}
\end{lemma}
\begin{proof}
    Берём $x_1, x_2, \dots, x_n$ --- независимые одинаково распределённые
    случайные величины.

    Нетрудно показать, что  вектора $\left(x_1, x_2, \dots, x_n\right)$ и
    $\left(x_{\sigma\left(1\right)}, x_{\sigma\left(2\right)},
        \dots, x_{\sigma\left(n\right)}\right)$
    имеют одинаковое распределение для любой перестановки $\sigma$,
    а это значит, что и оценки
    $\hat{\theta}\left(x_1, x_2, \dots, x_n\right)$ и
    $\hat{\theta}\left(x_{\sigma\left(1\right)}, x_{\sigma\left(2\right)},
        \dots, x_{\sigma\left(n\right)}\right)$
    распределены одинаково
    как функции случайных одинаково распределённых векторов.
    Следовательно, их математические ожидания равны между собой
    $$\meanof{\theta}{\hat{\theta}\left(x_1, x_2, \dots, x_n\right)}
        =\meanof{\theta}{
            \hat{\theta}\left(x_{\sigma\left(1\right)},
                x_{\sigma\left(2\right)},
                \dots, x_{\sigma\left(n\right)}\right)}
        =\theta$$

    Посчитаем математическое ожидание функции правдоподобия оценки
    $\hat{\theta}$
    \begin{align*}
        \meanof{\theta}{L\hat{\theta}}
            &=\frac{1}{n}\cdot\sum_{\sigma\in S_n}
            \meanof{\theta}{\hat{\theta}\left(x_{\sigma\left(1\right)},
                    x_{\sigma\left(2\right)},
                    \dots, x_{\sigma\left(n\right)}\right)}=\\
            &=\frac{1}{n!}\cdot\sum_{\sigma\in S_n}\theta
            =\theta
    \end{align*}

    Теперь посмотрим, чему равна дисперсия симметризации оценки $\hat{\theta}$
    \begin{align*}
        \dispersionof{\theta}{\Lambda\hat{\theta}}
            &=\meanof{\theta}{\left(\Lambda\hat{\theta}-\theta\right)^2}=\\
            &=\meanof{\theta}{\left\{\frac{1}{n}\cdot\sum_{\sigma\in S_n}
                \left[\hat{\theta}\left(x_{\sigma\left(1\right)},
                    x_{\sigma\left(2\right)}\right)
                    -\theta
                    \right]^2
                    \right\}}
    \end{align*}

    Вспомним неравенство Йенсена для математического ожидания
    (функция $f$ выпуклая вниз)
    $$\mean{f\left(\xi\right)}\ge f\left(\mean{\xi}\right)$$

    Применяем неравенство Йенсена и получаем желаемый результат
    \begin{align*}
        \dispersionof{\theta}{\Lambda\hat{\theta}}
            &\le\frac{1}{n!}\cdot\sum_{\theta\in S_n}\meanof{\theta}{
                \hat{\theta}\left(x_{\sigma\left(1\right)},
                    x_{\sigma\left(2\right)}\right)
                    -\theta}=\\
            &=\frac{1}{n!}\cdot\sum_{\theta\in S_n}
                \dispersionof{\theta}{\hat{\theta}}
            =\dispersionof{\theta}{\hat{\theta}}
    \end{align*}

    То есть, симметризация не ухудшает оценку,
    а в общем случае (когда неравенство строгое) даже делает её лучше.
\end{proof}

\begin{remark}
    Равенство в неравенстве Йенсена (в доказательстве выше)
    возможно только в случае симметричной функции.
    Значит, в качестве оценки достаточно брать симметричную функцию выборки
\end{remark}
\begin{remark}
    Если оценка $\hat{\theta}$ симметрична относительно перестановок аргументов,
    то она является функцией вариационного ряда
\end{remark}
\begin{remark}
    Все оценки, которые претендуют быть оптимальными,
    должны быть функциями вариационного ряда
\end{remark}
\section{Условное математическое ожидание и условные распределения}
\subsection{Условное математическое ожидание}
