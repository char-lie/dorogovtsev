Мы уже знаем, что нам не нужна вся выборка для построения хороших оценок ---
нам хватит достаточных статистик. Введя метод наименьших квадратов,
мы избавимся от неприятной процедуры вычисления интегралов.

Тем не менее, чтобы перейти непосредственно к изучению метода, необходимо
владеть инструментарием, коим являются случайные вектора.

\section{Основные характеристики случайного вектора}

\index{вектор!случайный}
Есть $\vec{\xi} = \left( \xi_1, \dots, \xi_n \right)$ --- случайный вектор.
С функцией распределения $\cdf{\vec{\xi}}$ возникают проблемы (скучновато и
громоздко), поэтому будем использовать плотность распределения.

\begin{definition}[Плотность распределения случайного вектора]
  \index{плотность распределения!случайного вектора}
  \index{случайный вектор!плотность распределения}
  $p$ --- плотность распределения случайного вектора
  $\vec{\xi} = \left( \xi_1, \dots, \xi_n \right)$, если
  \begin{enumerate}
      \item Вероятность того, что вектор $\vec{\xi}$ окажется
      в множестве $\Delta$, равна интегралу от плотности по этой области
      $$\Probability{\vec{\xi} \in \Delta}
          = \integrall{\Delta}{d\vec{u}}{\pdf{\vec{u}}}$$
      \item Во всех точках плотность неотрицательна
      $$\forall \vec{x} \in \mathbb{R}^n: \pdf{\vec{x}} \ge 0$$
      \item Выполняется условие нормировки
      $$\integrall{\mathbb{R}^n}{d\vec{u}}{\pdf{\vec{u}}} = 1$$
  \end{enumerate}
\end{definition}

Естественным образом вводится определение характеристической функции.

\begin{definition}[Характеристическая функция случайного вектора]
  \label{def:characteristicFunction}
  \index{характеристическая функция!случайного вектора}
  \index{случайный вектор!характеристическая функция}
  Значение характеристической функции случайного вектора $\vec{\xi}$
  в точке $\vec{\lambda}$ считается по формуле
  $$\varphi_{\vec{\xi}}\left( \vec{\lambda} \right)
      = \mean{e^{i \cdot \left( \vec{\lambda}, \vec{\xi} \right)}}
      = \mean{
      \exp{\left\{ i \cdot \sum_{k=1}^n \lambda_k \cdot \xi_k \right\}}}$$

  Когда существует плотность, имеем преобразование Фурье
  $$\varphi_{\vec{\xi}}\left( \vec{\lambda} \right)
      = \integrall{\mathbb{R}^n}{d\vec{u}}{\pdf{\vec{u}} \cdot
      e^{i \left( \vec{\lambda}, \vec{u} \right)}}$$
\end{definition}

\begin{definition}[Математическое ожидание случайного вектора]
  \index{математическое ожидание!случайного вектора}
  \index{случайный вектор!математическое ожидание}
  Математическое ожидание случайного вектора
  $\vec{\xi} = \left( \xi_1, \dots, \xi_n \right)$ --- вектор,
  элементы которого --- математические ожидания компонент
  случайного вектора $\vec{\xi}$
  $$\mean{\vec{\xi}} = \left( \mean{\xi_1}, \dots, \mean{\xi_n} \right)$$
\end{definition}

Но что же является дисперсией случайного вектора?

\subsection{Ковариационная матрица случайного вектора}

Начнём с определения ковариации двух случайных величин.

\begin{definition}[Ковариация]
  \index{ковариация}
  Ковариация двух случайных величин $\xi$ и $\eta$, принимающих действительные
  значения, обозначается $\cov{\xi, \eta}$ и считается по формуле
  $$\cov{\xi, \eta}
      = \Mean{\left( \xi - \mean{\xi} \right)
      \cdot \left( \eta - \mean{\eta} \right)}$$
\end{definition}

\begin{remark}
  Ковариация случайной величины $\xi$ с ней же --- её дисперсия
  $$\cov{\xi, \xi}
      = \Mean{\left( \xi - \mean{\xi} \right)
      \cdot \left( \xi - \mean{\xi} \right)}
      = \Mean{\left( \xi - \mean{\xi} \right)^2}
      = \dispersion{\xi}$$
\end{remark}

\begin{remark}
  Ковариация симметрична
  $$\cov{\xi, \eta}
      = \Mean{\left( \xi - \mean{\xi} \right)
      \cdot \left( \eta - \mean{\eta} \right)}
      = \Mean{\left( \eta - \mean{\eta} \right)
      \cdot \left( \xi - \mean{\xi} \right)}
      = \cov{\eta, \xi}$$
\end{remark}

\begin{remark}\label{rem:covIndepentent}
  Ковариация двух независимых случайных величин равна нулю
  \cite[с.~244]{Feller1}
\end{remark}

\begin{definition}[Ковариационная матрица случайного вектора]
  \label{def:vectorCovMatrix}
  \index{ковариационная матрица!случайного вектора}

  Ковариационная матрица случайного вектора
  $\vec{\xi} = \left( \xi_1, \dots, \xi_n \right)$ --- матрица, на пересечении
  $i$ строки и $j$ столбца которой находятся ковариации $i$ и $j$ элементов
  вектора $\xi$
  $$\dCov{\vec{\xi}}
      = \left\| \cov{\xi_i, \xi_j} \right\|_{i,j=1}^n
      = \left\| \mean{
      \left\{ \left( \xi_i - \mean{\xi_i} \right)
          \cdot \left( \xi_j - \mean{\xi_j} \right)
      \right\}} \right\|_{i,j=1}^n$$

  $$\dCov{\vec{\xi}} =
  \begin{bmatrix}
      \cov{\xi_1, \xi_1} & \cdots & \cov{\xi_1, \xi_n} \\
      \vdots & \ddots & \vdots \\
      \cov{\xi_n, \xi_1} & \cdots & \cov{\xi_n, \xi_n}
  \end{bmatrix}$$

\end{definition}

\begin{remark}
  На диагонали ковариационной матрицы $\dCov{\vec{\xi}}$
  случайного вектора $\xi$ стоят дисперсии компонент вектора.
\end{remark}

Случайный вектор находится во многомерном пространстве, а это значит,
что имеется много направлений его размазывания, поэтому в качестве дисперсии
нам нужна матрица.

\begin{example}
  Возьмём двумерный вектор с одним и тем же элементом
  в каждой координате --- случайной величиной из стандартного нормального
  распределения
  $$\vec{\xi} = \left( \xi, \xi \right),\; \xi \sim N\left( 0, 1 \right)$$

  Нетрудно посчитать, что ковариационная матрица будет заполнена единицами,
  так как во всех ячейках будет ковариация $\cov{\xi, \xi}$, равная
  дисперсии случайной величины $\xi$, то есть единице
  $$\dCov{\vec{\xi}} =
  \begin{bmatrix}
      1 & 1 \\
      1 & 1
  \end{bmatrix}$$
\end{example}

\begin{example}
  Возьмём опять же двумерный вектор, но с двумя независимыми
  случайными величинами из стандартного нормального распределения
  $$\vec{\xi} = \left( \xi_1, \xi_2 \right),\;
      \xi_1, \xi_2 \sim N\left( 0, 1 \right)$$

  На диагонали будут стоять единицы --- дисперсии случайных величин.
  Если две случайные величины независимы, то их ковариация равна нулю
  (замечание \ref{rem:covIndepentent}).
  Это в свою очередь означает, что вне диагонали
  будут нули
  $$\dCov{\vec{\xi}} =
  \begin{bmatrix}
      1 & 0 \\
      0 & 1
  \end{bmatrix}$$
\end{example}

\begin{definition}[Сопряжённая матрица]
  \index{сопряжённая матрица}
  \index{матрица!сопряжённая}
  Есть матрица $A$ размером $n \times m$ с комплексными элементами.
  Тогда сопряжённая к ней матрица $A^*$ получается путём транспонирования
  матрицы $A$ и замены всех элементов на комплексно-сопряжённые
  \cite[с.~243]{VoevodinLA}, то есть
  $$\left( a_{i,j}^* = \overline{a_{j,i}} \right),\;
  A \in \mathbb{C}^{n \times m}, A^* \in \mathbb{C}^{m \times n}$$

  Или же в таком виде
  $$A =
  \begin{bmatrix}
      a_{1,1} & \cdots & a_{1,m} \\
      \vdots & \ddots & \vdots \\
      a_{n,1} & \cdots & a_{n,m}
  \end{bmatrix}
      \Rightarrow
  A^* = 
  \begin{bmatrix}
      \overline{a_{1,1}} & \cdots & \overline{a_{n,1}} \\
      \vdots & \ddots & \vdots \\
      \overline{a_{1,m}} & \cdots & \overline{a_{n,m}}
  \end{bmatrix}$$
\end{definition}

\begin{remark}
  Отметим, что к матрице с действительными коэффициентами сопряжённой будет
  транспонированная матрица
  $$A \in \mathbb{R}^{n \times m} \Rightarrow A^* = A^T$$
\end{remark}

Чтобы не разрывать целостность дальнейших повествований, введём наперёд
небольшое утверждение. Точнее, просто вспомним комбинаторику.
\begin{affirmation}\label{affirmation:squaredSum}
  Квадрат суммы раскладывается в двойную сумму следующим образом
  $$\left( \sum_{k=1}^n x_k \right)^2 = \sum_{i, j = 1}^{n} x_i \cdot x_j$$
\end{affirmation}
\begin{proof}
  Чтобы убедиться в правильности формулы, вспомним мультиномиальные
  коэффициенты --- их значение и определение.

  Мультиномиальные коэффициенты --- множители при слагаемых
  $x_1^{m_1} \cdot x_n^{m_n}$ после разложения
  $\left( x_1 + \dots + x_n \right)^m$ в сумму и считаются по следующей
  формуле \cite[с.~23]{Grimaldi}
  \begin{align*}
      {m_1, \dots, m_n \choose m} = \frac{m!}{m_1! \cdots m_n!} \\
      0 \le m_1, \dots, m_n \le m,\; m_1 + \dots + m_n = m
  \end{align*}

  То есть вот общая формула раскрытия натуральной степени $m$ произвольной
  суммы выглядит так
  $$\left( x_1 + \dots + x_n \right)^m
      = \sum_{
          \substack{m_1 + \dots + m_n = m \\
          m_1, \dots, m_n \ge 0}}
      {m_1, \dots, m_n \choose m} \cdot x_1^{m_1} \cdots x_n^{m_n}$$

  Теперь вернёмся к нашему частному случаю: $m=2$. Тогда мультиномиальные
  коэффициенты будут иметь следующий вид
  \begin{align*}
  {m_1, \dots, m_n \choose 2} = \frac{2}{m_1! \cdots m_n!} \\
      0 \le m_1, \dots, m_n \le 2,\; m_1 + \dots + m_n = 2
  \end{align*}

  Из накладываемых ограничений видно, что в знаменателе будет либо одна
  двойка, либо две единицы, так как сумма должна равняться двойке.

  Таким образом, сумму можно разбить на две части --- квадраты ($m_k = 2$)
  и попарные произведения ($m_i \cdot m_j = 1,\; i \neq j$). Запишем
  \begin{equation}\label{eq:squaredSumStart}
      \begin{split}
      \left( x_1 + \dots + x_n \right)^2
          = \sum_{k=1}^{n} \frac{2}{2} \cdot x_k^2
        + \sum_{i=1}^{n-1}
            \sum_{j=i+1}^n \frac{2}{1} \cdot x_i \cdot x_j = \\
          = \sum_{k=1}^{n} x_k^2
        + 2 \cdot \sum_{i=1}^{n-1} \sum_{j=i+1}^n x_i \cdot x_j
      \end{split}
  \end{equation}

  В связи с коммутативностью умножения последнюю удвоенную двойную сумму можно
  раскрыть как сумму по всем недиагональным элементам
  \begin{align*}
      2 \cdot \sum_{i=1}^{n-1} \sum_{j=i+1}^n x_i \cdot x_j
      = \sum_{i=1}^{n-1} \sum_{j=i+1}^n x_i \cdot x_j
          + \sum_{i=1}^{n-1} \sum_{j=i+1}^n x_i \cdot x_j = \\
  \end{align*}
  \begin{align*}
      \sum_{i=1}^{n-1} \sum_{j=i+1}^n x_i \cdot x_j
          + \sum_{i=1}^{n-1} \sum_{j=i+1}^n x_i \cdot x_j = \\
      \sum_{i=1}^{n-1} \sum_{j=i+1}^n x_i \cdot x_j
          + \sum_{j=1}^{n-1} \sum_{i=i+1}^n x_j \cdot x_i = \\
      = \sum_{i=1}^{n-1} \sum_{j=i+1}^n x_i \cdot x_j
          + \sum_{i=2}^{n} \sum_{j=1}^{i-1} x_j \cdot x_i
      = \sum_{i \neq j}^{n} x_i \cdot x_j
  \end{align*}

  Вместе с суммой квадратов диагональных элементов получится сумма по всем
  произведением. Перепишем, во что превратится формула
  \eqref{eq:squaredSumStart}
  \begin{align*}
       \left( x_1 + \dots + x_n \right)^2
      = \sum_{k=1}^{n} x_k^2
          + 2 \cdot \sum_{i=1}^{n-1}
        \sum_{j=i+1}^n \cdot x_i \cdot x_j = \\
      = \sum_{k=1}^{n} x_k^2 + \sum_{i \neq j}^{n} x_i \cdot x_j
      = \sum_{i, j = 1}^{n} x_i \cdot x_j
  \end{align*}

  То есть действительно квадрат суммы равен сумме попарных произведений всех
  элементов, что и требовалось доказать
  $$\left( \sum_{k=1}^n x_k \right)^2 = \sum_{i, j = 1}^{n} x_i \cdot x_j$$
\end{proof}
\begin{proof}[Простое доказательство]
  Также можно доказать это утверждение, просто расписав квадрат как
  произведение
  \begin{align*}
      \left( \sum_{k=1}^{n} x_k \right)^2
      = \left( x_1 + \dots + x_n \right)
      \cdot \left( x_1 + \dots + x_n \right) = \\
      = x_1 \cdot x_1 + x_1 \cdot x_2 + \dots + x_1 \cdot x_n
      + x_2 \cdot x_1 + x_2 \cdot x_2 + \dots + x_n \cdot x_n
  \end{align*}

  Видим, что каждый элемент умножается с каждым, и всё это дело суммируется.
  Запишем в виде суммы (с красивым значком сигма)
  \begin{align*}
      \left( \sum_{k=1}^{n} x_k \right)^2
      = \left( x_1 + \dots + x_n \right)
      \cdot \left( x_1 + \dots + x_n \right) = \\
      = \sum_{i=1}^{n}
      \left( x_i \cdot x_1 + x_i \cdot x_2 + \dots + x_i \cdot x_n \right)
      = \sum_{i=1}^{n} \sum_{j=1}^{n} x_i \cdot x_j
  \end{align*}

  Что и требовалось доказать.
\end{proof}

Теперь мы готовы перейти к свойствам ковариационной матрицы
\begin{enumerate}
\index{ковариационная матрица!свойства}
  \item Симметричность. Ковариационная матрица случайного вектора $\vec{\xi}$
      равна своей сопряжённой
      $$\dCov{\vec{\xi}} = \dcCov{\vec{\xi}}$$
  \item Неотрицательная определённость\footnote{Больше о неотрицательно
      определённых операторах можно почитать в книге Ильина и Позняка
      ``Линейная алгебра'' \cite[с.~139]{IlinPoznyarLA}.
      В ней такой оператор называется положительным.}
      $$\dCov{\vec{\xi}} \ge 0$$

      Это значит следующее
      $$\forall \vec{u} \in \mathbb{R}^n:\;
      \left( \dCov{\vec{\xi}} \cdot \vec{u}, \vec{u} \right)
      = \sum_{i,j=1}^{n} \cov{\xi_i, \xi_j} \cdot u_j \cdot u_i
      \ge 0$$

      \begin{proof}
      Распишем ковариацию по определению и воспользуемся утверждением
      \ref{affirmation:squaredSum}
      \begin{align*}
          \sum_{i,j=1}^{n} \cov{\xi_i, \xi_j} \cdot u_j \cdot u_i = \\
          = \sum_{i,j=1}^{n} \Mean{\left( \xi_i - \mean{\xi_i} \right)
            \cdot \left( \xi_j - \mean{\xi_j} \right)}
            \cdot u_j \cdot u_i = \\
          = \mean{\left( \sum_{t=1}^{n} u_t
        \cdot \left( \xi_t - \mean{\xi_t} \right) \right)^2}
      \end{align*}

      Поскольку все коэффициенты действительные, а математическое
      ожидание константы равно самой константе, то делаем вывод,
      что сумма неотрицательна
      $$\sum_{i,j=1}^{n} \cov{\xi_i, \xi_j} \cdot u_j \cdot u_i
          = \mean{\left( \sum_{t=1}^{n} u_t
        \cdot \left( \xi_t - \mean{\xi_t} \right) \right)^2}
          \ge 0$$

      Вот мы и получили желаемый результат
      $$\forall \vec{u} \in \mathbb{R}^n:\;
          \left( \dCov{\vec{\xi}} \cdot \vec{u}, \vec{u} \right)
          = \sum_{i,j=1}^{n} \cov{\xi_i, \xi_j} \cdot u_j \cdot u_i
          \ge 0$$
      \end{proof}
\end{enumerate}

\begin{remark}\label{remark:linearAlgebra:selfAdjointMatrix}
  Вспомним линейную алгебру.

  Самосопряжённая неотрицательно определённая матрица $\dCov{\vec{\xi}}$ имеет
  собственный ортонормированный базис, в котором она превращается в
  диагональную матрицу с неотрицательным элементами
  $$\begin{bmatrix}
      \lambda_1 & & \mbox{\Huge{$\varnothing$}} \\
       & \ddots &  \\
       \mbox{\Huge{$\varnothing$}} & & \lambda_n
  \end{bmatrix},\; \lambda_k \ge 0$$

  Далее будем упускать символы пустоты $\varnothing$,
  подразумевая диагональные матрицы.

  Как эта матрица преобразует пространство?

  Единичная матрица не меняет ничего
  $$\begin{bmatrix}
      1 & &\\
      & \ddots & \\
      & & 1
  \end{bmatrix}$$

  Если первый элемент единичной матрицы сделать нулём, то такой оператор
  убивает первую координату вектора, на который подействует
  $$\begin{bmatrix}
      0 & & & \\
      & 1 & & \\
      & & \ddots & \\
      & & & 1
  \end{bmatrix}$$

  А такая матрица усиливает первую составляющую в десять раз и
  ослабляет остальные в десять раз
  $$\begin{bmatrix}
      10 & & &\\
      & 0.1 & & \\
      & & \ddots & \\
      & & & 0.1
  \end{bmatrix}$$

  Оказывается, через ковариационную матрицу вычисляются все характеристики
  линейных преобразований.
\end{remark}

\subsection{Ковариационная матрица}
\label{section:covMatrix}
Логичным обобщением ковариационной матрицы случайного вектора является
ковариационная матрица двух случайных векторов. Сейчас станет ясно, зачем мы
дважды писали вектор $\vec{\xi}$ в индексе оператора $\dCov{\vec{\xi}}$.

\begin{definition}[Ковариационная матрица]\label{def:covMatrix}
  \index{ковариационная матрица}
  Ковариационная матрица двух случайных векторов
  $\vec{\alpha} = \left( \alpha_1, \dots, \alpha_n \right)$ и
  $\vec{\beta} = \left( \beta_1, \dots, \beta_m \right)$ --- матрица,
  в которой на пересечении $i$ строки и $j$ столбца стоит ковариация случайных
  величин $\alpha_i$ и $\beta_j$
  $$\Cov{\vec{\alpha}}{\vec{\beta}}
      = \left\| \cov{\alpha_i, \beta_j} \right\|_{
      \substack{i= \overline{1,n},\\j= \overline{1,m}}}
      = \left\| \mean{
      \left\{ \left( \alpha_i - \mean{\alpha_i} \right)
          \cdot \left( \beta_j - \mean{\beta_j} \right)
      \right\}} \right\|_{
          \substack{i= \overline{1,n},\\j= \overline{1,m}}}$$

  $$\Cov{\vec{\alpha}}{\vec{\beta}} =
  \begin{bmatrix}
      \cov{\alpha_1, \beta_1} & \cdots & \cov{\alpha_1, \beta_m} \\
      \vdots & \ddots & \vdots \\
      \cov{\alpha_n, \beta_1} & \cdots & \cov{\alpha_n, \beta_m}
  \end{bmatrix}$$
\end{definition}

Ковариационная матрица обладает следующими свойствами
\index{ковариационная матрица!свойства}
\begin{enumerate}
  \item\label{item:covMatrix:property:transposition}
      Перестановка векторов ведёт к транспонированию матрицы (симметричность)
      $$\Cov{\beta}{\alpha} = \tCov{\alpha}{\beta}$$
  \item\label{item:covMatrix:property:linearityL}
      Линейность относительно первого аргумента
      $$\Cov{\operatorname{B} \alpha_1 + \operatorname{C} \alpha_2}{\beta}
      = \operatorname{B} \Cov{\alpha_1}{\beta}
          + \operatorname{C}\Cov{\alpha_2}{\beta}$$
  \item\label{item:covMatrix:property:linearityR}
      Линейность относительно второго аргумента
      $$\Cov{\alpha}{\operatorname{D} \beta_1 + \operatorname{E} \beta_2}
      = \Cov{\alpha}{\beta_1} \operatorname{D^T}
          + \Cov{\alpha}{\beta_2} \operatorname{E^T}$$
  \item\label{item:covMatrix:property:operatorOut}
      Следствие из свойств \ref{item:covMatrix:property:linearityL} и
      \ref{item:covMatrix:property:linearityR}
      $$\dCov{\operatorname{B} \vec{\xi}}
      = \operatorname{B} \dCov{\vec{\xi}} \operatorname{B^T}$$
\end{enumerate}



\section{Линейные преобразования случайных векторов}
\label{section:linearTransformations}

Рассмотрим всё тот же случайный вектор $\vec{\xi} = \left( \xi_1, \dots, \xi_n
\right)$ и произвольный константный вектор $\vec{\lambda} \in \mathbb{R}^n$.

\subsection{Скалярное произведение}
Определим случайную величину $\eta$ как скалярное произведения векторов
$\vec{\xi}$ и $\vec{\lambda}$
$$\eta = \left( \vec{\xi}, \vec{\lambda} \right)$$

Посчитаем математическое ожидание случайной величины $\eta$.

\begin{equation}\label{eq:scalarMulMean}
  \mean{\eta}
      = \mean{\sum_{k=1}^{n} \lambda_k \cdot \xi_k}
      = \sum_{k=1}^{n} \lambda_k \cdot \mean{\xi_k}
      = \left( \vec{\lambda}, \mean{\vec{\xi}} \right)
\end{equation}

Теперь посчитаем дисперсию
$$\dispersion{\eta}
  = \mean{\left( \eta - \mean{\eta} \right)^2}
  = \mean{\left( \sum_{k=1}^{n} \lambda_k \cdot \xi_k
      - \lambda_k \cdot \mean{\xi_k} \right)^2}$$

Полученное выражение сворачивается в математическое ожидание квадрата суммы,
которая превращается в двойную сумму произведений
$$\mean{\left\{ \sum_{k=1}^{n} \lambda_k
  \cdot \left( \xi_k - \mean{\xi_k} \right) \right\}^2}
  = \sum_{i,j=1}^{n}\Mean{\left( \xi_i - \mean{\xi_i} \right)
      \cdot \left( \xi_j - \mean{\xi_j} \right)}
      \cdot \lambda_i \cdot \lambda_j$$

А это, как мы уже знаем из утверждения \ref{affirmation:squaredSum},
произведение ковариационной матрицы вектора $\vec{\xi}$
на вектор $\vec{\lambda}$, умноженное на тот же вектор $\vec{\lambda}$.
То есть дисперсия $\eta$ выражается следующим образом
\begin{equation}\label{eq:linearQuadraticForm}
\dispersion{\eta}
  = \dispersion{\left( \vec{\xi}, \vec{\lambda} \right)}
  = \left( \dCov{\vec{\xi}} \vec{\lambda}, \vec{\lambda} \right)
\end{equation}

Оформим вывод в виде утверждения.

\begin{affirmation}\label{affirmation:scalarMulTransformations}
  \index{случайный вектор!скалярное произведение}
  \index{распределение!скалярного произведения}
  Скалярное произведение случайного вектора $\vec{\xi}$ и произвольного
  вектора $\vec{\lambda}$ является случайной величиной с математическим
  ожиданием $a$ и дисперсией $\sigma^2$
  $$a = \left( \lambda, \mean{\vec{\xi}} \right),\;
      \sigma^2
      = \left( \dCov{\vec{\xi}} \vec{\lambda}, \vec{\lambda} \right)$$
\end{affirmation}

\subsection{Поворот случайного вектора}
Обобщим задачу и попробуем выяснить, каким образом зависит случайный вектор
$\vec{\eta}$, полученный путём линейных преобразований вектора $\vec{\xi}$,
имеющего известное математическое ожидание и ковариационную матрицу.

Для линейных преобразований вектора нужен линейный оператор. Назовём его
$\operatorname{T}$. Этот оператор будет действовать из пространства
$\mathbb{R}^n$
в пространство $\mathbb{R}^m$, где $n$ --- размерность вектора $\vec{\xi}$,
а $m$ --- размерность вектора $\vec{\eta}$, который будет получен
в результате преобразования
$$\vec{\eta} = \operatorname{T} \vec{\xi} ,\; T \in \mathbb{R}^{m \times n}$$

Посчитаем математическое ожидание
$$\mean{\eta} = \Mean{\operatorname{T} \vec{\xi} }$$

Очевидно, что в связи с линейностью математического ожидания можно вынести
оператор $\operatorname{T}$ наружу.

Мы всё-таки проделаем математические выкладки по-честному.
Итак, у нас есть математическое ожидание случайного вектора
$$\Mean{\operatorname{T} \vec{\xi} }
  = \mean{\left\| \sum_{j=1}^n \left( T_{i,j} \cdot \xi_j \right)
      \right\|_{i=1}^m}$$

Математическое ожидание случайного вектора --- вектор математических ожиданий
соответствующих координат.
Дальше воспользуемся линейностью математического ожидания
$$\mean{\left\| \sum_{j=1}^n \left( T_{i,j} \cdot \xi_j \right)
      \right\|_{i=1}^m}
  = \left\| \Mean{\sum_{j=1}^n \left( T_{i,j} \cdot \xi_j \right)}
      \right\|_{i=1}^m
  = \left\| \sum_{j=1}^n \left( T_{i,j} \cdot \mean{\xi_j} \right)
      \right\|_{i=1}^m$$

Видим, что перед нами произведение матрицы $\operatorname{T}$ на вектор
математических ожиданий координат случайного вектора $\vec{\xi}$
$$\left\| \sum_{j=1}^n \left( T_{i,j} \cdot \mean{\xi_j} \right)
  \right\|_{i=1}^m = \operatorname{T} \mean{\vec{\xi}} $$

То есть интуиция нам подсказывала правильно и конечная формула такова
$$\mean{\eta}
  = \Mean{\operatorname{T} \vec{\xi} }
  = \operatorname{T} \mean{\vec{\xi}} $$

Теперь нужно посчитать ковариацию. Мы могли бы решать эту задачу,
расписав произведение матрицы или воспользовавшись свойством
\ref{item:covMatrix:property:operatorOut}, но в этот раз, пожалуй, освежим
наши знания в линейной алгебре.

Возьмём произвольный вектор $\vec{e} \in \mathbb{R}^n$
и выпишем квадратичную форму ковариационной матрицы вектора $\eta$
с аргументом $\vec{e}$. Из начала подраздела \eqref{eq:linearQuadraticForm}
помним, что такая квадратичная форма равна дисперсии скалярного произведения, а
дальше воспользуемся свойством симметричности скалярного произведения
(для удобства дальнейших вычислений)
$$\left( \dCov{\vec{\eta}} \cdot \vec{e}, \vec{e} \right)
  = \dispersion{\left( \vec{\eta}, \vec{e} \right)}
  = \dispersion{\left( \vec{e}, \vec{\eta} \right)}$$

Распишем наш случайный вектор $\vec{\eta}$ через случайный вектор $\vec{\xi}$
и матрицу $\operatorname{T}$
$$\dispersion{\left( \vec{e}, \vec{\eta} \right)}
  = \dispersion{\left( \vec{e}, \operatorname{T} \vec{\xi} \right)}$$

Далее воспользуемся ещё одним определением сопряжённого оператора\footnote{На
самом деле, это и есть изначальное определение сопряжённого оператора
\cite[с.~241]{VoevodinLA}, \cite[с.~126]{IlinPoznyarLA}}
и перенесём оператор $\operatorname{T}$ в левую часть скалярного произведения
$$\dispersion{\left( \vec{e}, \operatorname{T} \vec{\xi} \right)}
  = \dispersion{\left( \operatorname{T^*} \vec{e} , \vec{\xi} \right)}$$

Перейдём от дисперсии к квадратичной форме и посмотрим, что происходит
$$\dispersion{\left( \operatorname{T^*} \vec{e} , \vec{\xi} \right)}
  = \left( \operatorname{\dCov{\vec{\xi}}} \operatorname{T^*} \vec{e} ,
      \operatorname{T^*} \vec{e} \right)$$

Снова воспользуемся определением сопряжённого оператора и перенесём его
из правой стороны скалярного произведения в левую. Не забываем, что
сопряжённый оператор к сопряжённому оператору --- исходный оператор
$\left( \operatorname{T^*} \right)^* = \operatorname{T}$
$$\left( \operatorname{\dCov{\vec{\xi}}} \operatorname{T^*} \vec{e},
      \operatorname{T^*} \vec{e} \right)
  = \left( \operatorname{T} \operatorname{\dCov{\vec{\xi}}} \operatorname{T^*}
      \vec{e}, \vec{e} \right)$$

Видим, что квадратичные формы совпадают, а это значит, что и операторы равны
$$\left( \operatorname{T} \operatorname{\dCov{\vec{\xi}}} \operatorname{T^*}
      \vec{e}, \vec{e} \right)
      = \left( \dCov{\vec{\eta}} \cdot \vec{e}, \vec{e} \right)
  \Rightarrow
  \operatorname{T} \operatorname{\dCov{\vec{\xi}}} \operatorname{T^*} 
      = \dCov{\vec{\eta}}$$

Подведём итоги: если на случайный вектор $\vec{\xi}$ с известным математическим
ожиданием и ковариационной матрицей подействовать оператором $\operatorname{T}$,
то математическое ожидание полученного вектора будет считаться по формуле
$$\mean{\operatorname{T} \vec{\xi}} = \operatorname{T}\mean{\vec{\xi}}$$

Расчёт ковариационной матрицы происходит в базисе вектора $\vec{\xi}$
с матрицей перехода $\operatorname{T}$ и матрицей $\operatorname{T^*}$
для перехода обратно
$$\dCov{\operatorname{T} \vec{\xi}}
  = \operatorname{T} \operatorname{\dCov{\vec{\xi}}} \operatorname{T^*} $$

Подведём итоги в виде утверждения
\begin{affirmation}\label{affirmation:vectorRotated}
  \index{случайный вектор!вращение}
  Если на случайный вектор $\vec{\xi}$ подействовать линейным оператором
  $\operatorname{T}$, то получим случайный вектор с математическим ожиданием
  $\vec{a}$ и ковариационной матрицей $\operatorname{A}$

  $$\mean{\operatorname{T} \vec{\xi}} = \operatorname{T} \mean{\vec{\xi}}$$
  $$\dCov{\operatorname{T} \vec{\xi}}
      = \operatorname{T} \operatorname{\dCov{\vec{\xi}}} \operatorname{T^*}$$
\end{affirmation}

\subsection{Линейные преобразования случайных векторов}

Логичным продолжением всего вышесказанного будет вычисление характеристик
суммы двух случайных векторов $\vec{\xi}$ и $\vec{\eta}$, умноженных слева
на матрицы $\operatorname{C}$ и $\operatorname{D}$.

Условимся, что вектор $\vec{\xi}$ находится в пространстве $\mathbb{R}^n$,
а вектор $\vec{\eta}$ находится в $\mathbb{R}^m$. Тогда очевидно, что
оператор $\operatorname{C}$ должен принадлежать $\mathbb{R}^{k \times n}$,
а оператор $\operatorname{D}$ в свою очередь должен быть из множества
$\mathbb{R}^{k \times m}$.

Итого, задание: посчитать характеристики случайного вектора $\vec{\gamma}$
$$\vec{\gamma} = \operatorname{C} \vec{\xi} + \operatorname{D} \vec{\eta}$$

Математическое ожидание считается крайне просто --- достаточно воспользоваться
линейностью
$$\mean{\vec{\gamma}}
  = \operatorname{C} \mean{\vec{\xi}} + \operatorname{D} \mean{\vec{\eta}}$$

С ковариационной матрицей проблем возникнуть тоже не должно --- будем
использовать свойства из раздела \ref{section:covMatrix}.

Для начала распишем ковариационную матрицу в терминах исходной задачи
$$\dCov{\vec{\gamma}}
  = \dCov{\operatorname{C} \vec{\xi} + \operatorname{D} \vec{\eta}}$$

Сначала разобьём ковариацию на сумму двух ковариаций
$$
\dCov{\operatorname{C} \vec{\xi} + \operatorname{D} \vec{\eta}}
= \Cov{\operatorname{C} \vec{\xi}}{
      \operatorname{C} \vec{\xi} + \operatorname{D} \vec{\eta}}
  + \Cov{\operatorname{D} \vec{\eta}}{
      \operatorname{C} \vec{\xi} + \operatorname{D} \vec{\eta}}
$$

Воспользуемся симметричностью
$$
\Cov{\operatorname{C} \vec{\xi}}{
      \operatorname{C} \vec{\xi} + \operatorname{D} \vec{\eta}}
  + \Cov{\operatorname{D} \vec{\eta}}{
      \operatorname{C} \vec{\xi} + \operatorname{D} \vec{\eta}}
= \Cov[T]{\operatorname{C} \vec{\xi} + \operatorname{D} \vec{\eta}}{
      \operatorname{C} \vec{\xi}}
  + \Cov[T]{\operatorname{C} \vec{\xi} + \operatorname{D} \vec{\eta}}{
      \operatorname{D} \vec{\eta}}
$$

Разобьём на суммы и снова воспользуемся симметричностью, чтобы избавиться
от транспонированных матриц
\begin{align*}
\Cov[T]{\operatorname{C} \vec{\xi} + \operatorname{D} \vec{\eta}}{
      \operatorname{C} \vec{\xi}}
  + \Cov[T]{\operatorname{C} \vec{\xi} + \operatorname{D} \vec{\eta}}{
      \operatorname{D} \vec{\eta}} = \\
= \Cov[T]{\operatorname{D} \vec{\eta}}{\operatorname{C} \vec{\xi}}
  + \Cov[T]{\operatorname{C} \vec{\xi}}{\operatorname{C} \vec{\xi}}
  + \Cov[T]{\operatorname{C} \vec{\xi}}{\operatorname{D} \vec{\eta}}
  + \Cov[T]{\operatorname{D} \vec{\eta}}{\operatorname{D} \vec{\eta}} = \\
= \Cov{\operatorname{C} \vec{\xi}}{\operatorname{D} \vec{\eta}}
  + \Cov{\operatorname{C} \vec{\xi}}{\operatorname{C} \vec{\xi}}
  + \Cov{\operatorname{D} \vec{\eta}}{\operatorname{C} \vec{\xi}}
  + \Cov{\operatorname{D} \vec{\eta}}{\operatorname{D} \vec{\eta}}
\end{align*}

Вынесем операторы за знаки ковариаций
\begin{align*}
\Cov{\operatorname{C} \vec{\xi}}{\operatorname{D} \vec{\eta}}
  + \Cov{\operatorname{C} \vec{\xi}}{\operatorname{C} \vec{\xi}}
  + \Cov{\operatorname{D} \vec{\eta}}{\operatorname{C} \vec{\xi}}
  + \Cov{\operatorname{D} \vec{\eta}}{\operatorname{D} \vec{\eta}} = \\
= \operatorname{C} \Cov{\vec{\xi}}{\vec{\eta}} \operatorname{D^T}
  + \operatorname{C} \Cov{\vec{\xi}}{\vec{\xi}} \operatorname{C^T}
  + \operatorname{D} \Cov{\vec{\eta}}{\vec{\xi}} \operatorname{C^T}
  + \operatorname{D} \Cov{\vec{\eta}}{\vec{\eta}} \operatorname{D^T}
\end{align*}

Мне больше нравится, чтобы по бокам стояли ковариации $\vec{\xi}$ и
$\vec{\eta}$, а внутри уже ковариации обоих векторов. Запишем результат именно в
этом виде
\begin{align*}
\dCov{\operatorname{C} \vec{\xi} + \operatorname{D} \vec{\eta}}
  = \operatorname{C} \dCov{\vec{\xi}} \operatorname{C^T}
      + \operatorname{C} \Cov{\vec{\xi}}{\vec{\eta}} \operatorname{D^T}
      + \operatorname{D} \Cov{\vec{\eta}}{\vec{\xi}} \operatorname{C^T}
      + \operatorname{D} \dCov{\vec{\eta}} \operatorname{D^T}
\end{align*}

\begin{affirmation}\label{affirmation:randomVector:linearTransformations}
  \index{случайный вектор!линейные преобразования}
  Есть два линейных оператора $\operatorname{C} \in \mathbb{R}^{k \times n}$
  и $\operatorname{D} \in \mathbb{R}^{k \times m}$, два случайных вектора:
  вектор $\vec{\xi}$ из $\mathbb{R}^n$ с параметрами $\vec{a}$ и
  $\operatorname{A}$, вектор $\vec{\eta}$ из $\mathbb{R}^m$ с параметрами
  $\vec{b}$ и $\operatorname{B}$.

  В таком случае сумма случайных векторов, полученных с помощью операторов
  $\operatorname{C}$ и $\operatorname{D}$, будет случайным вектором
  в $\mathbb{R}^k$ с параметрами $\vec{m}$ и $\operatorname{M}$
  \begin{align*}
      \vec{m} &= \operatorname{C} \mean{\vec{\xi}}
      + \operatorname{D} \mean{\vec{\eta}} \\
      \operatorname{M} &= \operatorname{C} \dCov{\vec{\xi}} \operatorname{C^T}
      + \operatorname{C} \Cov{\vec{\xi}}{\vec{\eta}} \operatorname{D^T}
      + \operatorname{D} \Cov{\vec{\eta}}{\vec{\xi}} \operatorname{C^T}
      + \operatorname{D} \dCov{\vec{\eta}} \operatorname{D^T}
  \end{align*}
\end{affirmation}
