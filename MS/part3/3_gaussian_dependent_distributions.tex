\section{Распределения, связанные с нормальным}

\subsection{Распределение Пирсона (хи-квадрат, $\chi^2$)}
\index{распределение!Пирсона}
\index{Пирсона распределение}
\index{распределение!хи-квадрат@хи-квадрат ($\chi^2$)}
\index{хи-квадрат распределение@хи-квадрат ($\chi^2$) распределение}

\begin{definition}[Распределение Пирсона]
    Если $\xi_1, \dots, \xi_n$ --- независимые стандартные гауссовские величины,
    то случайная величина с распределением Пирсона с $n$ степенями свободы
    является суммой квадратов случайных величин $\xi_1, \dots, \xi_n$ и
    обозначается греческой буквой ``хи'' $\chi_n^2$
    $$\eta = \xi_1^2 + \dots + \xi_n^2 \sim \chi_n^2$$
\end{definition}

Выясним, какой вид имеет функция распределения случайной величины $\chi_n^2$.

Начнём с определения
\begin{align*}
    \cdfof{\chi_n^2}{t}
    = \probability{\xi_1^2 + \dots + \xi_n^2 \le t}
\end{align*}

Это, ясное дело, интеграл по всему объёму, где сумма квадратов случайных
гауссовских величин не превышает число $t$. Интегрировать нужно плотность
\begin{align*}
    \probability{\xi_1^2 + \dots + \xi_n^2 \le t}
    = \idotsint\limits_{x_1^2 + \dots + x_n^2 \le t}
        \left( 2 \cdot \pi \right)^{- \frac{n}{2}}
        \cdot e^{- \frac{u_1^2 + \dots + u_n^2}{2}}
        \, du_1 \dots du_n
\end{align*}

Перейдём в полярные координаты в этом многомерном пространстве, вынеся некую
константу за знак интеграла. Потом мы эту константу найдём по условию
нормировки. Якобиан перехода в полярные координаты из $n$-мерного декартового
пространства равен $\rho^{n-1}$
\begin{align*}
    \idotsint\limits_{x_1^2 + \dots + x_n^2 \le t}
        \left( 2 \cdot \pi \right)^{- \frac{n}{2}}
        \cdot e^{- \frac{u_1^2 + \dots + u_n^2}{2}}
        \, du_1 \dots du_n
    = c' \cdot \integral{0}{\sqrt{t}}{\rho}{
        \rho^{n-1} \cdot e^{- \frac{\rho^2}{2}}}
\end{align*}

Введём новое обозначение $S = \frac{\rho^2}{2}$
\begin{align*}
    c' \cdot \integral{0}{\sqrt{t}}{\rho}{
        \rho^{n-1} \cdot e^{- \frac{\rho^2}{2}}}
    = \begin{array}{|c|}
        S = \frac{\rho^2}{2} \\
        \rho = \sqrt{2 \cdot S} \\
        d\rho = \frac{dS}{\sqrt{2 \cdot S}}
    \end{array} = \\
    = c' \cdot \integral{0}{\frac{t}{2}}{S}{
        S^{\frac{n-1}{2}} \cdot e^{-S} \cdot S^{-\frac{1}{2}}}
    = c' \cdot \integral{0}{\frac{t}{2}}{S}{
        S^{\frac{n-1}{2} - 1} \cdot e^{-S}}
\end{align*}

Вспомним определение $\Gamma$-функции \cite[с.~416]{DorogovtsevMA}, устремим
$t$ к бесконечности, и найдём константу $c'$. Эта константа не зависит от
$t$, поэтому имеем право
\index{гамма-функция}
\index{гамма-функция!$\Gamma$-функция}
\begin{align*}
    c' \cdot \integral{0}{\frac{t}{2}}{S}{
        S^{\frac{n-1}{2} - 1} \cdot e^{-S}}
    \covergencen{t}{a}
    c' \cdot \integral{0}{+\infty}{S}{
        S^{\frac{n-1}{2} - 1} \cdot e^{-S}} = \\
    = \begin{array}{|c|}
        \Gamma\left( \alpha \right)
            = \integral{0}{+\infty}{x}{e^{-x} \cdot x^{\alpha-1}} \\
        \alpha = \frac{n-1}{2} \\
        x = S
    \end{array}
    = c' \cdot \Gamma\left( \frac{n-1}{2} \right)
    \Rightarrow c' = \frac{1}{\Gamma\left( \frac{n-1}{2} \right)}
\end{align*}

Итого, у нас есть функция распределения
\begin{align*}
    \cdfof{\chi_n^2}{t}
    =  \frac{1}{\Gamma\left( \frac{n}{2} \right)} \cdot
        \integral{0}{\frac{t}{2}}{S}{\rho^{\frac{n}{2}-1} \cdot e^S}
\end{align*}

Чтобы найти плотность распределения, нужно взять производную от интеграла
с переменным верхним пределом \cite[с.~353]{IlinMA1}. Делаем, но не забываем,
что результатирующая величина будет неотрицательной
\begin{align*}
    \pdfof{\chi_n^2}{t}
    = \frac{t^{\frac{n}{2}-1} \cdot e^{-\frac{t}{2}}}{
            2^{\frac{n}{2}} \cdot \Gamma\left( \frac{n}{2} \right)}
        \cdot \indicator{t>0}
\end{align*}

С двумя степенями свободы распределение Пирсона превращается в экспоненциальное
\begin{align*}
    \chi_2^2 = Exp\left( \frac{1}{2} \right)
\end{align*}

Математическое ожидание и дисперсию посчитать несложно
\begin{align*}
    \mean{\chi_n^2} &= \Mean{\xi_1^2 + \dots + \xi_n^2} = n \\
    \dispersion{\chi_n^2} &= 2 \cdot n
\end{align*}

\begin{lemma}[Фишера]
    \index{лемма!Фишера}
    \index{гауссовский вектор!стандартный}
    Есть стандартный гауссовский вектор
    $\vec{\xi} = \left( \xi_1, \dots, \xi_n \right)$, есть ортогональная матрица
    $\operatorname{C}$ ($\operatorname{C^T} = \operatorname{C^{-1}}$) и
    случайный вектор $\vec{\eta} = \left( \eta_1, \dots, \eta_r \right)$,
    полученная следующим образом
    \begin{align*}
        \vec{\eta} = \operatorname{C} \vec{\xi}
    \end{align*}

    Тогда, если отнять от суммы квадратов элементов вектора $\vec{\xi}$ сумму
    квадратов элемента вектора $\vec{\eta}$, получится случайная величина
    с распределением Пирсона с $n-r$ степенями свободы
    \begin{align*}
        \sum_{k=1}^{n} \xi_k^2 - \sum_{k=1}^{r} \xi_k^2 \sim \chi_{n-r}^2
    \end{align*}
\end{lemma}

\subsection{Распределение Фишера}
\index{распределение!Фишера}
\index{Фишера распределение}
\begin{definition}[Распределение Фишера]
    Отношение независимых случайных $\chi_{k_1}^2$ и $\chi_{k_2}^2$ называется
    распределением Фишера
    \begin{align*}
        F_{k_1, k_2} = \frac{\chi_{k_1}^2}{\chi_{k_2}^2}
    \end{align*}
\end{definition}

\subsection{Распределение Стьюдента}
\index{распределение!Стьюдента}
\index{Стьюдента распределение}
\begin{definition}[Распределение Стьюдента]
    Есть $n+1$ независимых стандартных гауссовских случайных величин
    $\xi_0, \dots, \xi_n$. Отношение первой (нулевой) случайной величины к
    среднеквадратическому отклонению остальных имеет распределение Стьюдента
    с $n$ степенями свободы
    \begin{align*}
        \frac{\xi_0}{\sqrt{\frac{1}{n} \cdot \sum_{k=1}^{n}\xi_k^2}} \sim t_n
    \end{align*}
\end{definition}

То есть, отношение стандартной гауссовской случайной величины к случайной
величине, имеющей распределение Пирсона с $n$ степенями свободы (они должны быть
независимыми), будет случайной величиной с распределением Стьюдента с теми же
$n$ степенями свободы
\begin{align*}
    \frac{N\left( 0, 1 \right)}{\chi_n^2} \sim t_n
\end{align*}

Отметим, что квадрат этой случайной величины, делённой на $n$, имеет
распределение Фишера с параметрами $1$ и $n$
\begin{align*}
    \frac{t_n^2}{n}
    = \frac{\xi_0^2}{\sum_{k=1}^{n}\xi_k^2}
    \sim F_{1, n}
\end{align*}

\begin{example}
    \index{независимость!выборочного среднего и выборочной дисперсии}
    Есть выборка $x_1, \dots, x_n$ из нормального распределения
    $N\left( a, \sigma^2 \right)$.

    Выборочное среднее $\overline{x}$ и выборочная дисперсия
    $\frac{1}{n-1} \cdot \sum_{k=1}^{n} \left( x_k - \overline{x} \right)^2$
    не зависят друг от друга, а это значит, что случайные величины
    $\overline{x}$ и $\sum_{k=1}^{n} \left( x_k - \overline{x} \right)^2$ тоже
    независимы\footnote{первым доказал Фишер \cite{FisherStudentApplication},
    сам же Стьюдент показал лишь некоррелированность этих случайных величин
    \cite{StudentProbableError}}.

    Распределения выглядят следующим образом
    \begin{align*}
        \frac{\sqrt{n} \cdot \left( \overline{x} - a \right)}{\sigma}
            \sim N\left( 0, 1 \right) \\
        \frac{1}{\sigma^2}
                \cdot \sum_{k=1}^{n}\left( x_k - \overline{x} \right)^2
                \sim \chi_{n-1}^2
    \end{align*}

    В таком случае получаем распределение Стьюдента таким образом
    \begin{align*}
        \frac{\sqrt{n} \cdot \left( \overline{x} - a \right)}{
            \sqrt{\frac{q}{n-1}
                \cdot \sum_{k=1}^{n}\left( x_k - \overline{x} \right)^2}}
    \end{align*}
\end{example}
