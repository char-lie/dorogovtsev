\chapter{Метод наименьших квадратов}
Мы уже знаем, что нам не нужна вся выборка для построения хороших оценок ---
нам хватит достаточных статистик. Введя метод наименьших квадратов,
мы избавимся от неприятной процедуры вычисления интегралов.

\section{Гауссовские случайные вектора}

\subsection{Основные характеристики случайного вектора}

Есть $\vec{\xi} = \left( \xi_1, \dots, \xi_n \right)$ --- случайный вектор.
С функцией распределения $\cdf{\vec{\xi}}$ возникают проблемы (скучновато и
громоздко), поэтому будем использовать плотность распределения.

\begin{definition}[Плотность распределения случайного вектора]
    \index{плотность распределения!случайного вектора}
    \index{случайный вектор!плотность распределения}
    $p$ --- плотность распределения случайного вектора
    $\vec{\xi} = \left( \xi_1, \dots, \xi_n \right)$, если
    \begin{enumerate}
        \item Вероятность того, что вектор $\vec{\xi}$ окажется
            в множестве $\Delta$, равна интегралу от плотности по этой области
            $$\Probability{\vec{\xi} \in \Delta}
                = \integrall{\Delta}{d\vec{u}}{\pdf{\vec{u}}}$$
        \item Во всех точках плотность неотрицательна
            $$\forall \vec{x} \in \mathbb{R}^n: \pdf{\vec{x}} \ge 0$$
        \item Выполняется условие нормировки
            $$\integrall{\mathbb{R}^n}{d\vec{u}}{\pdf{\vec{u}}} = 1$$
    \end{enumerate}
\end{definition}

Естественным образом вводится определение характеристической функции.

\begin{definition}[Характеристическая функция случайного вектора]
    \index{характеристическая функция!случайного вектора}
    \index{случайный вектор!характеристическая функция}
    Значение характеристической функции случайного вектора $\vec{\xi}$
    в точке $\vec{\lambda}$ считается по формуле
    $$\varphi_{\vec{\xi}}\left( \vec{\lambda} \right)
        = \mean{e^{i \cdot \left( \vec{\lambda}, \vec{\xi} \right)}}
        = \mean{
            \exp{\left\{ i \cdot \sum_{k=1}^n \lambda_k \cdot \xi_k \right\}}}$$

    Когда существует плотность, имеем преобразование Фурье
    $$\varphi_{\vec{\xi}}\left( \vec{\lambda} \right)
        = \integrall{\mathbb{R}^n}{d\vec{u}}{\pdf{\vec{u}} \cdot
            e^{i \left( \vec{\lambda}, \vec{u} \right)}}$$
\end{definition}

\begin{definition}[Математическое ожидание случайного вектора]
    \index{математическое ожидание!случайного вектора}
    \index{случайный вектор!математическое ожидание}
    Математическое ожидание случайного вектора
    $\vec{\xi} = \left( \xi_1, \dots, \xi_n \right)$ --- вектор,
    элементы которого --- математические ожидания компонент
    случайного вектора $\vec{\xi}$
    $$\mean{\vec{\xi}} = \left( \mean{\xi_1}, \dots, \mean{\xi_n} \right)$$
\end{definition}

Но что же является дисперсией случайного вектора?

\subsection{Ковариационная матрица}

Начнём с определения ковариации двух случайных величин.

\begin{definition}[Ковариация]
    Ковариация двух случайных величин $\xi$ и $\eta$ обозначается
    $\cov{\xi, \eta}$ и считается по формуле
    $$\cov{\xi, \eta}
        = \Mean{\left( \xi - \mean{\xi} \right)
            \cdot \left( \eta - \mean{\eta} \right)}$$
\end{definition}

\begin{remark}
    Ковариация случайной величины $\xi$ с ней же --- её дисперсия
    $$\cov{\xi, \xi}
        = \Mean{\left( \xi - \mean{\xi} \right)
            \cdot \left( \xi - \mean{\xi} \right)}
        = \Mean{\left( \xi - \mean{\xi} \right)^2}
        = \dispersion{\xi}$$
\end{remark}

\begin{remark}
    Ковариация симметрична
    $$\cov{\xi, \eta}
        = \Mean{\left( \xi - \mean{\xi} \right)
            \cdot \left( \eta - \mean{\eta} \right)}
        = \Mean{\left( \eta - \mean{\eta} \right)
            \cdot \left( \xi - \mean{\xi} \right)}
        =\cov{\eta, \xi}$$
\end{remark}

\begin{definition}
    Ковариационная матрица случайного вектора
    $\vec{\xi} = \left( \xi_1, \dots, \xi_n \right)$ --- матрица,
    на пересечении $i$ столбца и $j$ строки стоят произведения ковариаций
    $i$ и $j$ элемента случайного вектора $\vec{\xi}$
    $$\Cov{\vec{\xi}}{\vec{\xi}}
        = \left[ \cov{\xi_i, \xi_j} \right]_{i,j=1}^n
        = \left[ \mean{
            \left\{ \left( \xi_i - \mean{\xi_i} \right)
                \cdot \left( \xi_j - \mean{\xi_j} \right)
            \right\}} \right]_{i,j=1}^n$$
\end{definition}
