\chapter{Методы оценивания неизвестных параметров}

\section{Вступление}

Мы занимаемся гауссовскими векторами, поскольку они геометричны. Об этом говорят
такие важные свойства
\begin{enumerate}
  \item Некоррелированность гауссовских векторов эквивалентна их независимости
      (теорема \ref{theorem:gaussianVector:independence})
  \item После поворота гауссовский вектор остаётся гауссовским
      (лемма \ref{lemma:gaussMoved})
  \item Чтобы вычислить совместное распределение гауссовских векторов,
      достаточно матриц
      (теорема \ref{theorem:gaussVector:conditionalDistribution})
\end{enumerate}

Рассмотрим пример, который наведёт нас на необходимые размышления.

\begin{example}\label{example:leastSquares:scalarOptimalEstimator}
  Есть выборка \xsample из гауссовского распределения с неизвестным
  средним $x_i \sim N\left( \theta, 1 \right)$.

  Как проверить, что оценка $\theta_*$ оптимальная?

  Поскольку гауссовское распределение является экспоненциальным
  (определение \ref{def:exponentialDistribution}),
  то для него существует эффективная оценка
  (утверждение \ref{affirmation:efficientEstimator:exponentialExsistance}).

  Выпишем функцию правдоподобия (определение \ref{def:likehoodFunction})
  \begin{align*}
      L\left( \vec{x}, \theta \right)
      = \frac{1}{\sqrt{2 \cdot \pi}^2}
      \cdot e^{- \frac{1}{2}
          \cdot \sum_{k=1}^{n}\left( x_i - \theta \right)^2}
  \end{align*}

  Эффективная оценка считается по формуле (определение
  \ref{def:maximumLikelihoodEstimation})
  \begin{align*}
      \theta_* = \argmaxof{\ln{L\left( \vec{x},\theta \right)}}{\theta}
  \end{align*}

  Поскольку $\theta$ находится лишь в экспоненте, то нужно максимизировать
  экспоненту
  \begin{align*}
      \theta_* = \argmaxof{\left( e^{- \frac{1}{2}
      \cdot \sum_{k=1}^{n}\left( x_i - \theta \right)^2} \right)}{\theta}
  \end{align*}

  Сумма неотрицательна, а это значит, что для того, чтобы экспонента приняла
  максимальное значение, нужно минимизировать эту сумму
  \begin{align*}
      \theta_* = \argminof{
      \sum_{k=1}^{n}\left( x_i - \theta \right)^2}{\theta}
  \end{align*}

  Вспомнив геометрию с физикой,\footnote{
      Есть вектор $\vec{y}$, есть вектора $\vec{a}_1, \dots, \vec{a}_n$.

      В точке $\vec{a}$ находится центр масс тела, состоящего из точек
      $\vec{a}_k$ (массу каждой точки считаем равной единице)
      \begin{align*}
      \vec{a} = \frac{1}{n} \cdot \sum_{k=1}^{n} \vec{a}_k
      \end{align*}

      Момент инерции относительно точки $\vec{y}$ --- сумма квадратов
      расстояний, умноженных на массы точек
      \begin{align*}
      \sum_{k=1}^{n} \left\| \vec{y} - \vec{a}_k \right\|^2
      = n \cdot \left\| \vec{y} - \vec{a} \right\|^2
          + \sum_{k=1}^{n} \left\| \vec{a} - \vec{a}_k \right\|^2
      \end{align*}

      Момент инерции минимальный у центра масс.}
  понимаем, что в качестве оценки $\theta_*$ нужно взять выборочное среднее,
  так как момент инерции минимален у центра масс
  \begin{align*}
      \theta_* = \frac{1}{n} \cdot \sum_{k=1}^{n} x_k
  \end{align*}
\end{example}

\begin{example}\label{example:leastSquares:vectorOptimalEstimator}
  Обобщим задачу \ref{example:leastSquares:scalarOptimalEstimator}.
  Мы ведь изучали гауссовские вектора для того, чтобы иметь дело с многомерным
  случаем.

  Возьмём $m$-мерную оценку $\vec{\theta}$
  \begin{align*}
      \vec{\theta} \in \mathbb{R}^m,\; m \le n
  \end{align*}

  Матрица $\operatorname{A}$
  \begin{align*}
      A:\; \mathbb{R}^m \rightarrow \mathbb{R}^n
  \end{align*}

  Вместо выборки \xsample будем рассматривать вектор $\vec{y}$
  \begin{align*}
      \vec{y} = \operatorname{A} \vec{\theta} + \vec{\xi},\;
      \vec{\xi} \sim N\left( \vec{0}, \operatorname{I} \right)
  \end{align*}

  Эта ситуация является модификацией исходной задачи. Из леммы
  \ref{lemma:gaussMoved} Помним, что гауссовский вектор в сумме с произвольным
  --- смещённый гауссовский вектор
  \begin{align*}
      \left( x_1, \dots, x_n \right)
      = \overbrace{\left( \theta, \dots, \theta \right)}^{n} + \vec{\xi},\;
      m=1 \Rightarrow \operatorname{A} \theta
      = \overbrace{\left( \theta, \dots, \theta \right)}^{n}
  \end{align*}

  Мы чуть-чуть усложнили задачу и теперь нужно найти оптимальную оценку для
  векторной величины
  \begin{align*}
      \vec{\theta}_* = ?
  \end{align*}

  Поскольку имеем дело с экспоненциальным распределением, то дальше делаем то
  же, что и в предыдущем примере. Для начала выписываем функцию правдоподобия
  \begin{align*}
      L\left( \vec{\xi}, \vec{\theta} \right)
      = \frac{1}{\sqrt{2 \cdot \pi}} \cdot e^{-\frac{1}{2}
      \cdot \left\| \vec{x} - \operatorname{A} \vec{\theta} \right\|^2}
  \end{align*}

  Чтобы максимизировать функцию правдоподобия, нужно минимизировать ту норму,
  что в экспоненте
  \begin{align*}
      \vec{\theta}_*
      = \argminof{\left\| \vec{\xi}
      - \operatorname{A} \vec{\theta} \right\|^2}{
          \vec{\theta} \in \mathbb{R}^m}
  \end{align*}
\end{example}

Увидели, что поиск оптимальной оценки среднего для гауссовского вектора
теперь сводится не к дифференциальным уравнениям, а к поиску минимума.

\section{Оценка метода наименьших квадратов}
\index{метод!наименьших квадратов}
\index{оценка!метода наименьших квадратов}

\subsection{Размышления}

Есть случайные величины $x$ и $y$, между которыми нужно определить зависимость.

Гипотеза: эта зависимость может быть линейной
\begin{align*}
  y = \sum_{k=1}^{m}f_k\left( x \right) \cdot \theta_k
\end{align*}

В формуле $f_k$ --- известная функция, $\theta_k$ --- неизвестный коэффициент.

\begin{example}
  \begin{align*}
      \begin{cases}
      f_1\left( x \right) = 1 \\
      f_2\left( x \right) = x \\
      \end{cases} \Rightarrow
      f\left( x \right) = \theta_1 + \theta_2 \cdot x
  \end{align*}
\end{example}

\begin{example}
  Пускай функции будут степенями $x$
  \begin{align*}
      \begin{cases}
      f_1\left( x \right) = 1 \\
      f_2\left( x \right) = x \\
      \begin{array}{c}
      \vdots
      \end{array} \\
      f_m\left( x \right) = x^{m-1}
      \end{cases} \Rightarrow
      y = \sum_{k=1}^{m} \theta_k \cdot x^{k-1}
  \end{align*}

  Получили полиномиальную зависимость.
\end{example}

Когда есть наблюдения $y_1, \dots, y_n$, то возникают и помехи
$\xi_1, \dots, \xi_n \sim N\left( 0, 1 \right)$\footnote{Считаем, что смещение
помех нулевое, так как иначе это будет означать неправильную постановку
эксперимента. Также считаем, что эти случайные величины независимы между собой
по той же причине.

У нас идеальные условия, и экспериментаторы --- люди толковые.

Ковариация выбрана единичной для удобства. Потом мы это дело обобщим.}
\begin{align*}
  \begin{array}{c}
      y_1 = \sum_{k=1}^{m} f_k\left( x_1 \right) \cdot \theta_k + \xi_1 \\
      \vdots \\
      y_n = \sum_{k=1}^{m} f_k\left( x_1 \right) \cdot \theta_k + \xi_n
  \end{array}
\end{align*}

Какая же оценка будет оптимальной для
$\vec{\theta} = \left( \theta_1, \dots, \theta_m \right)$?

Рассмотрим в контексте задачи \ref{example:leastSquares:vectorOptimalEstimator}.

\begin{example}
  Помехи составляют стандартный гауссовский вектор как совокупность
  независимых стандартных гауссовских величин
  \begin{align*}
      \vec{\xi} \sim N\left( \vec{0}, \operatorname{I} \right)
  \end{align*}

  Также у нас есть параметр $\vec{\theta} \in \mathbb{R}^m$.

  Матрица $\operatorname{A}$ принимает следующий вид
  \begin{align*}
      \operatorname{A} = \left\| f_j\left( x_i \right) \right\|_{
      \substack{i = \overline{1, n} \\ j = \overline{1, m}}}
  \end{align*}

  Тогда вектор $\vec{y}$ примет понятный из контекста примера
  \ref{example:leastSquares:vectorOptimalEstimator} вид
  \begin{align*}
      \vec{y} = \operatorname{A} \vec{\theta} + \vec{\xi}
  \end{align*}

  Тут мы уже знаем, как искать минимум
  \begin{align*}
      \argminof{\left\| \vec{y} - \operatorname{A} \vec{\theta} \right\|^2}{
      \vec{\theta} \in \mathbb{R}^m}
      = \argminof{\sum_{k=1}^{n} \left( y_k
          - \left( \operatorname{A} \vec{\theta} \right)_k \right)^2}{
      \vec{\theta} \in \mathbb{R}^m}
  \end{align*}
\end{example}

Если для гауссовских случайных величин этот метод работает хорошо, то обобщим
его.

\subsection{Общая постановка задачи метода наименьших квадратов}

Вектор $\vec{\xi}$ --- случайный вектор в $\mathbb{R}^n$ с нулевым средним и
единичной ковариацией.

Оператор $\operatorname{A}$ действует из $\mathbb{R}^m$ в $\mathbb{R}^n$. Чтобы
он не уничтожал параметры, скажем, что его ранг равен $m$
\begin{align*}
  \rank{\operatorname{A}} = m
\end{align*}

Нужно найти оптимальную линейную оценку для параметра $\vec{\theta}$ по
наблюдениям $\vec{y}$
\begin{align*}
  \vec{\theta}_* = \operatorname{B} \vec{y},\;
      \vec{y} = \operatorname{A} \vec{\theta} + \vec{\xi}
\end{align*}

\begin{theorem}[Гаусса-Маркова]
  \index{теорема!Гаусса-Маркова}
  \label{theorem:gaussMarkov}
  Пусть матрица $\operatorname{A} \in \mathbb{R}^{n \times m}$ имеет ранг $m$,
  а случайный вектор $\vec{\xi}$ находится в $\mathbb{R}^n$ и обладает нулевым
  средним и единичной ковариацией.

  Тогда оценка $\vec{\theta}_*$ является единственным решением задачи
  минимизации
  \begin{align*}
      \vec{\theta}_*
      = \argminof{\left\| \vec{y} - \operatorname{A} \vec{\theta} \right\|^2}{
      \vec{\theta} \in \mathbb{R}^m}
  \end{align*}

  То есть ковариация оценки $\vec{\theta}_*$ не больше, чем ковариация любой
  несмещённой оценки $\vec{\tau}$ параметра $\vec{\theta}$ (разность
  ковариационных матриц --- неотрицательно определённая матрица)
  \begin{align*}
      \dCov{\vec{\theta}_*} \le \dCov{\vec{\tau}}
      \Rightarrow \dCov{\vec{\tau}} - \dCov{\vec{\theta}_*} \ge 0
  \end{align*}
\end{theorem}
\begin{proof}
  Разобьём доказательство на три части

  \begin{enumerate}
      \item Нужно найти минимум --- значит, возьмём производную по $m$
      аргументам. Это оператор набла по параметру $\vec{\theta}$
      \begin{align*}
          \nabla_{\vec{\theta}} \cdot \left\| \vec{y}
        - \operatorname{A} \vec{\theta} \right\|^2
          = \nabla_{\vec{\theta}} \cdot
        \left( \vec{y} - \operatorname{A} \vec{\theta},
        \vec{y} - \operatorname{A} \vec{\theta} \right) = \\
          = \nabla_{\vec{\theta}} \cdot \left( \vec{y}, \vec{y} \right)
        - 2 \cdot \nabla_{\vec{\theta}} \cdot
            \left( \operatorname{A} \vec{\theta}, \vec{y} \right)
        + \nabla_{\vec{\theta}} \cdot
            \left( \operatorname{A} \vec{\theta},
        \operatorname{A} \vec{\theta} \right) = \\
          = \vec{0} - 2 \cdot \operatorname{A^*} \vec{y}
        + 2 \cdot \operatorname{A^*} \operatorname{A} \vec{\theta}
        + \vec{0}
      \end{align*}

      Приравниваем к нулю
      \begin{align*}
          \operatorname{A^*} \operatorname{A} \vec{\theta}
          = \operatorname{A^*} \vec{y}
      \end{align*}

      И тут важный момент
      \begin{align*}
          \rank{\operatorname{A}} = \rank{\operatorname{A^*}} = m
      \end{align*}

      С помощью оператора $\operatorname{A}$ мы переходим из пространства
      $\mathbb{R}^m$ в пространство $\mathbb{R}^n$, а с помощью оператора
      $\operatorname{A^*}$ мы переходим обратно.

      Выполняется следующее тождество
      \begin{align*}
          \operatorname{A^*} \operatorname{A} \mathbb{R}^m = \mathbb{R}^m
      \end{align*}

      Значит, матрица $\operatorname{A^*} \operatorname{A}$ невырожденная
      \begin{align*}
          \det{\operatorname{A^*} \operatorname{A}} > 0
      \end{align*}

      Значит, можно решить уравнения единственным образом
      \begin{equation}\label{eq:leastSquares:formula}
          \index{оценка!метода наименьших квадратов!формула}
          \vec{\theta}_*
          = \left( \operatorname{A^*} \operatorname{A} \right)^{-1}
        \operatorname{A^*} \vec{y}
      \end{equation}

      Это и есть явный вид оценки метода наименьших квадратов.

      \item Возьмём случайный вектор $\vec{\eta}$ со средним $\vec{m}$
      и ковариацией $\operatorname{B}$.
      
      Возьмём матрицу $\operatorname{C}$. Тогда, согласно утверждению
      \ref{affirmation:vectorRotated}, среднее и ковариационная матрица
      вектора $\operatorname{C}\vec{\eta}$ примут следующий вид
      \begin{align*}
          \mean{\operatorname{C}\vec{\eta}}
        &= \operatorname{C} \vec{m}\\
          \dCov{\operatorname{C} \vec{\eta}}
        &= \operatorname{C} \operatorname{B} \operatorname{C^*}
      \end{align*}

      Вектор $\vec{y}$ --- случайный вектор со средним
      $\operatorname{A} \vec{\theta}$ и единичной ковариацией.

      Оценка $\vec{\theta}_*$ действительно несмещённая. Чтобы это
      показать, воспользуемся определением \ref{def:estimatorBias}
      и формулой \ref{eq:leastSquares:formula}
      \begin{align*}
          \mean{\vec{\theta}_*}
          = \left( \operatorname{A^*} \operatorname{A} \right)^{-1}
        \operatorname{A^*} \mean{\vec{y}}
          = \left( \operatorname{A^*} \operatorname{A} \right)^{-1}
        \operatorname{A^*} \operatorname{A} \vec{\theta}
          = \vec{\theta}
      \end{align*}

      Посмотрим, какая у оценки ковариационная матрица $\vec{\theta}_*$
      \begin{align*}
          \dCov{\vec{\theta}_*}
          = \left( \operatorname{A^*} \operatorname{A} \right)^{-1}
        \operatorname{A^*} \operatorname{I} \left\{
            \left( \operatorname{A^*} \operatorname{A} \right)^{-1}
        \operatorname{A^*} \right\}^*
      \end{align*}

      Вспомним, что сопряжение меняет местами множители
      \begin{align*}
          \left( \operatorname{A^*} \operatorname{A} \right)^{-1}
        \operatorname{A^*} \operatorname{I} \left\{
            \left( \operatorname{A^*} \operatorname{A} \right)^{-1}
        \operatorname{A^*} \right\}^* = \\
          = \left( \operatorname{A^*} \operatorname{A} \right)^{-1}
        \operatorname{A^*} \operatorname{A}
        \left\{ \left( \operatorname{A^*} \operatorname{A}
            \right)^{-1} \right\}^*
          = \left( \operatorname{A^*} \operatorname{A} \right)^{-1}
      \end{align*}

      \item Почему же оценка $\vec{\theta}_*$ будет оптимальной в классе
      линейных несмещённых оценок?

      \index{пространство!несмещённых оценок!линейных}
      Обозначим пространство линейных несмещённых оценок параметра
      $\vec{\theta}$ с вектором шума $\vec{\xi}$ как
      $\mathfrak{L}_{\vec{\xi}}$.

      \index{пространство!несмещённых оценок}
      \index{оценка!несмещённая!пространство}
      Пространство всех несмещённых оценок параметра $\vec{\theta}$ с
      помехами $\vec{\xi}$ будет $\mathfrak{M}_{\vec{\xi}}$.

      Очевидно, что пространство линейных несмещённых оценок параметра
      $\vec{\theta}$ является подпространством всех несмещённых оценок
      параметра $\vec{\theta}$
      \begin{align*}
          \mathfrak{L}_{\vec{\xi}} \subset \mathfrak{M}_{\vec{\xi}}
      \end{align*}

      Введём ещё одну группу обозначений.

      Пускай $\vec{\eta}$ --- случайный вектор в $\mathbb{R}^n$ с нулевым
      средним и единичной ковариацией, как у вектора $\vec{\xi}$
      \begin{align*}
          \vec{\eta} &\sim N\left( \vec{0}, \operatorname{I} \right) \\
          \mean{\vec{\eta}} &= \mean{\vec{\xi}} = \vec{0} \\
          \dCov{\vec{\eta}} &= \dCov{\vec{\xi}} = \operatorname{I}
      \end{align*}

      Вектору $\vec{\eta}$ поставим в соответствие пространства
      $\mathfrak{L}_{\vec{\eta}}$ и $\mathfrak{M}_{\vec{\eta}}$.

      \index{оценка!линейная несмещённая}
      Что такое линейная несмещённая оценка параметра$\vec{\theta}$?
      Это матрицы, применённые к выборке
      \begin{align*}
          \mathfrak{L}_{\vec{\xi}} \ni \operatorname{T} \vec{y}:\;
          \mean{\operatorname{T} \vec{y}}
          = \operatorname{T} \operatorname{A} \vec{\theta}
          = \vec{\theta}
      \end{align*}

      Возьмём последнее равенство из формулы и видим, что тут ничего не
      сказано о помехах --- ни о $\vec{\eta}$, ни о $\vec{\xi}$
      \begin{align*}
          \operatorname{T} \operatorname{A} \vec{\theta}
          = \vec{\theta}
      \end{align*}

      Тогда делаем вывод, что пространство линейных несмещённых оценок
      едино для любых шумов
      \begin{align*}
          \mathfrak{L}_{\vec{\eta}}
          = \mathfrak{L}_{\vec{\xi}}
          \ni \operatorname{T} \vec{y}
      \end{align*}

      Подсчитаем ковариацию вектора $\operatorname{T} \vec{y}$ и видим,
      что она от шумов тоже не зависит
      \begin{align*}
          \dCov{\operatorname{T} \vec{y}}
          = \operatorname{T} \operatorname{T^*}
      \end{align*}

      Следовательно, искать оптимальную линейную несмещённую оценку в
      $\mathfrak{L}_{\vec{\xi}}$ --- то же самое, что искать оптимальную
      линейную несмещённую оценку в $\mathfrak{L}_{\vec{\eta}}$.

      Значит, мы будем использовать пространство
      $\mathfrak{L}_{\vec{\eta}}$, поскольку вектор $\vec{\eta}$
      гауссовский, а для него мы уже умеем искать оптимальную оценку.

      Оценка в формуле \eqref{eq:leastSquares:formula} является
      оптимальной в пространстве несмещённых оценок
      $\mathfrak{M}_{\vec{\xi}}$. Ещё она линейна, а это значит, что
      входит в пространство $\mathfrak{L}_{\vec{\xi}}$.
      То есть она лучшая в пространстве линейных несмещённых оценок.
  \end{enumerate}
\end{proof}
\subsection{Оценка метода наименьших квадратов при неединичной ковариации}
\index{оценка!метода наименьших квадратов!при неединичной ковариации}
\index{метод!наименьших квадратов!неединичная ковариация}

Есть вектор помех $\vec{\xi}$ с нулевым математическим ожиданием и положительной
ковариацией. Между ``случайными'' помехами есть связь
\begin{align*}
  \mean{\vec{\xi}} &= \vec{0} \\
  \dCov{\vec{\xi}} &= \operatorname{V} > 0
\end{align*}

Умножим вектор $\vec{\xi}$ на матрицу $\operatorname{B}$, квадрат которой равен
обратной к $\operatorname{V}$ матрице, и назовём полученный вектор $\vec{\eta}$
\begin{align*}
  \operatorname{B} &= \operatorname{V^{-\frac{1}{2}}} \\
  \operatorname{\vec{\xi}} &= \vec{\eta} \\
  \dCov{\vec{\eta}} &= \operatorname{B} \dCov{\vec{\xi}} \operatorname{B^*}
      = \operatorname{V^{-\frac{1}{2}}} \operatorname{V}
      \operatorname{V^{-\frac{1}{2}}} = \operatorname{I}
\end{align*}

Тогда, подействовав оператором $\operatorname{B}$ на вектор измерений $\vec{y}$,
получим следующее
\begin{align*}
  \operatorname{B} \vec{y}
  = \operatorname{B} \operatorname{A} \vec{\theta}
      + \operatorname{B} \vec{\xi}
  = \operatorname{V^{-\frac{1}{2}}} \operatorname{A} \vec{\theta} + \vec{\eta}
\end{align*}

Матрица $\operatorname{V^{-\frac{1}{2}}}$ невырожденная и квадратная, а это
значит, что при умножении её на матрицу $\operatorname{A}$ получим результат
с рангом $m$
\begin{align*}
  \rank{\operatorname{V^{-\frac{1}{1}}} \operatorname{A}}
  = \rank{\operatorname{A}}
  = m
\end{align*}

Тогда оценка метода минимальных квадратов будет считаться как минимум такого
выражения
\begin{align*}
  \vec{\theta}_*
  = \argminof{\left\| \operatorname{V^{-\frac{1}{2}}} \operatorname{A}
      \operatorname{V^{-\frac{1}{2}}} \vec{y} \right\|^2}{\vec{\theta}}
\end{align*}

Подставим наши данные в исходную формулу \eqref{eq:leastSquares:formula}
\begin{align*}
  \vec{\theta}_*
  = \left( \operatorname{A^*} \operatorname{V^{-\frac{1}{2}}}
      \operatorname{V^{-\frac{1}{2}}} \operatorname{A} \right)^{-1}
      \operatorname{V^{-\frac{1}{2}}} \operatorname{A} \vec{y}
\end{align*}

Получили конечную формулу метода оценки наименьших квадратов при неединичной
ковариации
\begin{equation}\label{eq:leastSquares:formulaAdvanced}
  \index{оценка!метода наименьших квадратов!формула}
  \vec{\theta}_*
  = \left( \operatorname{A^*} \operatorname{V^{-1}}
      \operatorname{A} \right)^{-1}
      \operatorname{V^{-\frac{1}{2}}} \operatorname{A} \vec{y}
\end{equation}

\section{Интервальное оценивание}
\subsection{Определения}

Есть выборка \xsample из распределения $\cdfof{\theta}{x}$,
неизвестный параметр $\theta$ --- число действительное
$\theta \in \Theta \subset \mathbb{R}$.

Есть две статистики (функции от выборки, определение \ref{def:statistic})
$T_1 \le T_2$ и константа $\alpha \in \left( 0; 1 \right)$

\begin{definition}[Доверительный интервал]
  \index{доверительный интервал}
  Промежуток $\left[ T_1, T_2 \right]$ ($T_1 \le T_2$ --- статистики)
  называется доверительным интервалом для действительного параметра
  $\theta \in \mathbb{R}$ с уровнем доверия $\alpha$, если
  \begin{align*}
      \forall \theta \in \Theta:\;
      \Probability[\theta]{\theta \in \left[ T_1; T_2 \right]} \ge \alpha
  \end{align*}
\end{definition}

\begin{definition}[Центральная статистика]
  \index{центральная статистика}
  \index{статистика!центральная}
  Функция $G\left( \vec{x}, \theta \right)$, зависящая от выборки и
  неизвестного параметра $\theta$, называется центральной статистикой,
  если выполняются два свойства
  \begin{enumerate}
      \item $G\left( \vec{x}, \theta \right)$ имеет известное распределение
      \item $G\left( \vec{x}, \cdot \right)$ строго монотонная и непрерывная
      функция ($G\left( \vec{\xi}, \theta \right)$ строго монотонна и
      непрерывна по $\theta$)
  \end{enumerate}
\end{definition}

\subsection{Рецепт построения доверительного интервала с помощью центральной
  статистики}
\index{доверительный интервал!построение}

Нам дано $\alpha \in \left[ 0; 1 \right]$, центральная статистика имеет
распределение с функцией распределения $F$.

Выберем числа $a, b \in \mathbb{R}$ такие, что $a < b$. Тогда разность функций
распределения в этих точках будет вероятностью попадания достаточной статистики
в этот интервал
\begin{align*}
  \cdf{b} - \cdf{a} \ge \alpha \Rightarrow
  \Probability{G\left( \vec{\xi}, \theta \right) \in \left[ a; b \right]}
  \ge \alpha
\end{align*}

Поскольку центральная статистика $G$ монотонна относительно параметра $\theta$,
то событие, заключающееся в том, что она попала в промежуток
$\left[ a; b \right]$, эквивалентно тому событию, что параметр $\theta$ попал
в отрезок, ограниченный некими функциями от $a$ и $b$
\begin{align*}
  a \le G\left( \vec{x}, \theta \right) \le b
  \Leftrightarrow T_1 \le \theta \le T_2
\end{align*}

\subsection{Асимптотические доверительные интервалы}
\index{доверительный интервал!асимптотический}

Пускай \xsample --- выборка из гладкого распределения с дважды
интегрируемой плотностью (смотрите замечание \ref{remark:doubleDiff})
$\pdf{\vec{x}, \theta}$.

Вспомним, чему равен вклад выборки (определение \ref{def:defU} и замечание
\ref{remark:defU})
\begin{align*}
  U\left( \vec{x},\theta \right)
  = \sum_{k=1}^n\frac{\partial}{\partial\theta}\ln{\pdf{x_k,\theta}}
\end{align*}

Математическое ожидание каждого слагаемого равно нулю
(посмотрите замечание \ref{remark:expectationU})
\begin{align*}
  \meanof{\theta}{\frac{\partial}{\partial \theta} \ln{\pdf{x_k, \theta}}}
  = 0
\end{align*}

Также вспомним, чему равно количество информации Фишера (определение
\ref{def:fisherInformation}) по одному элементу выборки
\begin{align*}
  \dispersionof{\theta}{\frac{\partial}{\partial\theta}
      \ln{\pdf{x_k, \theta}}}
  = I_1\left( \theta \right)
\end{align*}

Выпишем следующее соотношение
\begin{align*}
  \frac{U\left( \vec{x}, \theta \right)}{
      \sqrt{n \cdot I_1\left( \theta \right)}}
  = \frac{1}{\sqrt{n}} \sum_{k=1}^{n}\frac{
      \frac{\partial}{\partial\theta}\ln{\pdf{x_k, \theta}}}{
      \sqrt{I_1\left( \theta \right)}}
\end{align*}

Видим сумму независимых случайных величин с нулевым математическим ожиданием и
единичной дисперсией, делённую на $\sqrt{n}$, как в центральной предельной
теореме. Это значит, что при больших $n$ такое отношение является центральной
статистикой со стандартным нормальным распределением.
\begin{align*}
  \frac{U\left( \vec{x}, \theta \right)}{
      \sqrt{n \cdot I_1\left( \theta \right)}} \approx N\left( 0, 1 \right)
\end{align*}

\section{Критерий согласия}
\index{критерий согласия}

Критерий согласия даёт ответ на вопрос, является ли данная выборка
\xsample выборкой из распределения $F$.

\subsection{Критерий Колмогорова-Смирнова}
\index{критерий согласия!Колмогорова-Смирнова}
\index{Смирнов!критерий согласия}
\index{Колмогоров!критерий согласия}

Есть выборка \xsample. Проверим согласие с распределением $F$.

Функция $F$ --- непрерывная, строго возрастающая.

\begin{lemma}
  \index{лемма!критерий Колмогорова-Смирнова}
  Пусть $\xi$ --- случайная величина с непрерывной и строго монотонной
  функцией распределения $F$.

  Тогда $\cdf{\xi}$ имеет равномерное распределение на $\left[ 0; 1 \right]$
\end{lemma}
\begin{proof}
  \begin{align*}
      \cdfof{\cdf{\xi}}{a}
      = \Probability{\cdf{\xi} \le a}
      = \Probability{\xi \le F^{-1}\left( a \right)}
      = \cdf{F^{-1}\left( a \right)}
      = a \in \left( 0; 1 \right)
  \end{align*}
\end{proof}

Если \xsample --- выборка из $F$, то $\cdf{x_1}, \dots, \cdf{x_n}$
--- выборка из $U\left( \left[ 0; 1 \right] \right)$.
\begin{align*}
  \frac{1}{n} \cdot \sum_{k=1}^{n}\Indicator{y \le \cdf{x_k}}
  \Covergence{} y
\end{align*}

Если это выполняется для любого $y$ из $\left[ 0; 1 \right]$, то справедливо
следующее тождество
\begin{align*}
  \frac{1}{n} \cdot \sum_{k=1}^{n}\Indicator{\cdot \le \cdf{x_k}}
  \Covergence{} \cdot
\end{align*}

Отметим, что центральная предельная теорема работает и в функциональном
пространстве и следующая случайная величина сходится к стандартному гауссовскому
распределению в пространстве непрерывных функций на $\left[ 0; 1 \right]$
\index{центральная предельная теорема!функциональное пространство}
\begin{align*}
  \frac{1}{\sqrt{n}} \cdot \left(
      \sum_{k=1}^{n}\Indicator{\cdot \le \cdf{x_k}} - \cdot \right)
  \Covergence{} N\left( 0, 1 \right)
\end{align*}

Распределение супремума такой случайной величины имеет распределение Смирнова
\begin{align*}
  \frac{1}{\sqrt{n}} \cdot \sup\limits_{y \in \left[ 0; 1 \right]}
      {\left| \sum_{k=1}^{n} \left( \Indicator{y \le \cdf{x_k}} - y  \right)
      \right|}
\end{align*}

Для распределения Смирнова нет аналитической функции, но есть таблицы.

Поскольку следующее тождество очевидно
\begin{align*}
  \cdf{x_{\left( k \right)}} = \cdf{x}_{\left( k \right)}
\end{align*}

Вместо того, чтобы брать супремум по всем точкам, достаточно перебирать лишь
точки, в которых происходит скачок
\begin{align*}
  \frac{1}{\sqrt{n}} \cdot \left(
      \sum_{k=1}^{n}\Indicator{\cdot \le \cdf{x_k}} - \cdot \right)
  = \max{\frac{1}{\sqrt{n}} \cdot \left| \frac{k}{n}
      - \cdf{x_{\left( x_k \right)}} \right|}
  \sim Q
\end{align*}

\index{распределение!Смирнова}
\index{распределение!Колмогорова-Смирнова}
\index{Смирнов!распределение}
\index{Колмогорова-Смирнова!распределение}
Буква $Q$ означает распределение Колмогорова-Смирнова (в литературе встречается
под названием распределение Смирнова).

Обозначим случайную величину, равную максимуму, через $R$
\begin{align*}
  R = \max\limits_{k= \overline{1,n}}{
      \frac{1}{\sqrt{n}} \cdot \left| \frac{k}{n}
      - \cdf{x_{\left( x_k \right)}} \right|}
  \sim Q
\end{align*}

\subsection{Рецепт применения критерия Колмогорова-Смирнова}
\index{критерий согласия!Колмогорова-Смирнова!рецепт}

Систематизируем полученные знания для фиксированного $\alpha$.

\begin{enumerate}
  \item Вычисляем по табличкам значение $r_{\alpha}$
      \begin{align*}
      r_{\alpha}:\; Q\left( \xi \le r_{\alpha} \right) = \alpha
      \end{align*}
  \item Вычисляем $R$
      \begin{align*}
      R = \max\limits_{k= \overline{1,n}}{
          \frac{1}{\sqrt{n}} \cdot \left| \frac{k}{n}
        - \cdf{x_{\left( k \right)}} \right|}
      \end{align*}
  \item Если $R$ не больше, чем $r_{\alpha}$, то ответ --- ``да''.
      Если же $R$ больше, чем $r_{\alpha}$, то ответ --- ``нет''
\end{enumerate}

Но как же быть с дискретными распределениями? Иногда видоизменяют центральную
предельную теорему, но мы же пойдём иным путём.

\subsection{Критерий Пирсона $\chi^2$}
\index{критерий согласия!Пирсона|Пирсона $\chi^2$}
\index{Пирсон!критерий согласия}

Есть выборка \xsample. Имеет ли она распределение $F$?

Известно, что $F$ принадлежит какому-то интервалу $\left[ a; b \right]$.
Разбиваем интервал возможных значений выборки на $m$ полуинтервалов
так же, как для построения гистограммы в подразделе \ref{subsection:histogram}.

Введём несколько обозначений
\begin{enumerate}
  \item $\Delta_i$ --- полуинтервал номер $i$
  \item $\nu_i$ --- количество элементов выборки, которые попали в $\Delta_i$
      \begin{align*}
      \nu_i = \sum_{k=1}^{n} \Indicator{x_k \in \Delta_i}
      \end{align*}
  \item $p_i$ --- вероятность случайных точек попасть в $\Delta_i$
      \begin{align*}
      p_i = \Probability[F]{x_1 \in \Delta_i}
      \end{align*}
\end{enumerate}

Идея следующая: после применения центральной предельной теоремы (с некоторой
правкой) получим распределение Пирсона
\begin{align*}
  \sum_{i=1}^{m}\frac{\left( \nu_i - n \cdot p_i \right)^2}{n \cdot p_i}
  \sim \chi_{m-1}^2
\end{align*}

Почему так происходит?

Предположим, мы угадываем $F$. В таком случае $n \cdot p_i$ --- математическое
ожидание. Смещаем $\nu_i$ для применения центральной предельной теоремы.
Без квадрата нужно было бы делить на $\sqrt{n}$, с квадратом же --- на $n$.

Чем больше линейных связей, тем меньше степеней свободы, поэтому их у нас $m-1$.

\begin{theorem}
  \index{теорема!критерий Пирсона}
  Если \xsample --- выборка из $F$, $\nu_i$ и $p_i$ определены
  следующим образом
  \begin{align*}
      \nu_i &= \sum_{k=1}^{n}\Indicator{x_k \in \Delta_i} \\
      p_i &= \Probability[F]{x_1 \in \Delta_i}
  \end{align*}
  Тогда
  \begin{align*}
      \sum_{i=1}^{m}\frac{\left( \nu_i - n \cdot p_i \right)^2}{n \cdot p_i}
      \Covergence{} \chi_{m-1}^2
  \end{align*}
\end{theorem}
\begin{proof}
  Обозначим вектор $\vec{\eta}_n$ следующим образом
  \begin{align*}
      \vec{\eta}_n = \left( \frac{\nu_1 - n \cdot p_1}{\sqrt{n \cdot p_1}},
      \dots, \frac{\nu_m - n \cdot p_m}{\sqrt{n \cdot p_m}} \right)
  \end{align*}

  Координаты вектора $\vec{\eta}_n$ зависимы между собой, поэтому центральная
  предельная теорема тут не поможет.

  Вспомним, что слабая сходимость --- то же самое, что поточечная сходимость
  характеристических функций. Вычислим характеристическую функцию вектора
  $\vec{\eta}_n$.

  Берём произвольный вектор $\lambda \in \mathbb{R}^m$. Нам нужно найти
  характеристическую функцию
  \begin{align*}
      \varphi_{\vec{\eta}_n}
      = \mean{e^{i \cdot \vec{\lambda} \cdot \vec{\eta}_n}}
  \end{align*}

  Для начала разберём скалярное произведение в терминах исходной выборки
  \begin{align*}
      \left( \vec{\lambda}, \vec{\eta}_n \right)
      = \sum_{k=1}^{m} \lambda_k \cdot \frac{\nu_k - n \cdot p_k}{
          \sqrt{n \cdot p_k}}
      = \sum_{k=1}^{m}\frac{\lambda_k}{\sqrt{n \cdot p_k}}
      \cdot \sum_{j=1}^{n}\left(
          \Indicator{x_j \in \Delta_k} - p_k \right)
  \end{align*}

  Поменяем суммы местами и получим сумму $n$ независимых одинаково
  распределённых случайных величин
  \begin{align*}
      \sum_{k=1}^{m}\frac{\lambda_k}{\sqrt{n \cdot p_k}}
      \cdot \sum_{j=1}^{n}\left(
          \Indicator{x_j \in \Delta_k} - p_k \right)
      = \sum_{j=1}^{n}\sum_{k=1}^{m}\frac{\lambda_k}{\sqrt{n \cdot p_k}}
          \cdot \left( \Indicator{x_j \in \Delta_k} - p_k \right)
  \end{align*}

  Помним, что характеристическая функция суммы независимых случайных величин
  --- произведение характеристических функций, а когда это сумма одной и той
  же случайной величины, то получается характеристическая функция в степени
  \begin{align*}
      \varphi_{\left( \sum_{i=1}^n \xi_i \right)}
      = \prod_{i=1}^n{\varphi_{\xi_i}}
      = \begin{array}{|c|}
      \xi_i = \xi
      \end{array}
      = \varphi_{\xi}^n
  \end{align*}

  Получаем следующее
  \begin{align*}
      \mean{e^{i \cdot \vec{\lambda} \cdot \vec{\eta}_n}}
      = \left\{ \mean{e^{i \cdot \sum_{k=1}^{m}\frac{\lambda_k}{
          \sqrt{n \cdot p_k}} \cdot \left(
          \Indicator{x_1 \in \Delta_k} - p_k \right)}} \right\}^n
  \end{align*}

  Посчитаем, чему равняется математическое ожидание, которое нужно возвести
  в степень $n$
  \begin{equation}\label{eq:criteriaPirson:meanPowerable}
      \mean{\exp{\left\{ i \cdot \sum_{k=1}^{m}\frac{\lambda_k}{
          \sqrt{n \cdot p_k}} \cdot \left(
          \Indicator{x_1 \in \Delta_k} - p_k \right) \right\}}} = ?
  \end{equation}

  Для начала проанализируем индикатор в экспоненте.
  Разобьём сумму на две
  \begin{align*}
      \sum_{k=1}^{m}\frac{\lambda_k}{\sqrt{n \cdot p_k}} \cdot \left(
          \Indicator{x_1 \in \Delta_k} - p_k \right)
      = \sum_{k=1}^{m}\frac{\lambda_k}{\sqrt{n \cdot p_k}}
          \cdot \Indicator{x_1 \in \Delta_k}
      - \sum_{k=1}^{m}\frac{\lambda_k}{\sqrt{n \cdot p_k}} \cdot p_k = \\
      = \sum_{k=1}^{m}\frac{\lambda_k}{\sqrt{n \cdot p_k}}
          \cdot \Indicator{x_1 \in \Delta_k}
      - \sum_{k=1}^{m}\frac{\lambda_k \cdot \sqrt{p_k}}{\sqrt{n}}
  \end{align*}

  То есть в первой сумме остаётся лишь лишь то слагаемое, которое
  соответствует промежутку, в который попал элемент выборки. Имеем право
  переписать всё это дело в следующем виде
  \begin{align*}
      \sum_{k=1}^{m}\frac{\lambda_k}{\sqrt{n \cdot p_k}}
          \cdot \Indicator{x_1 \in \Delta_k}
      - \sum_{k=1}^{m}\frac{\lambda_k}{\sqrt{n \cdot p_k}} \cdot p_k = \\
      = \sum_{k=1}^{m} \Indicator{x_1 \in \Delta_k} \cdot \left(
      \frac{\lambda_k}{\sqrt{n \cdot p_k}} - \sum_{l=1}^{m}
          \frac{\lambda_l \cdot \sqrt{p_l}}{\sqrt{n}} \right)
  \end{align*}

  А это значит, что экспоненту \eqref{eq:criteriaPirson:meanPowerable} можем
  переписать в следующем виде
  \begin{align*}
      \mean{\exp{\left\{ i \cdot \sum_{k=1}^{m}\frac{\lambda_k}{
          \sqrt{n \cdot p_k}} \cdot \left(
          \Indicator{x_1 \in \Delta_k} - p_k \right) \right\}}} = \\
      = \mean{\sum_{k=1}^{m} \Indicator{x_1 \in \Delta_k} \cdot
      \exp{\left\{ i \cdot \left( \frac{\lambda_k}{\sqrt{n \cdot p_k}}
          - \sum_{l=1}^{m} \frac{\lambda_l \cdot \sqrt{p_l}}{\sqrt{n}}
          \right) \right\}}}
  \end{align*}

  Воспользуемся линейностью математического ожидания, а так же тем фактом,
  что математическое ожидание индикатора $\Indicator{x_1 \in \Delta_k}$
  есть ни что иное, как вероятность $p_k$
  \begin{align*}
      \mean{\sum_{k=1}^{m} \Indicator{x_1 \in \Delta_k} \cdot
      \exp{\left\{ i \cdot \left( \frac{\lambda_k}{\sqrt{n \cdot p_k}}
          - \sum_{l=1}^{m} \frac{\lambda_l \cdot \sqrt{p_l}}{\sqrt{n}}
          \right) \right\}}} = \\
      = \sum_{k=1}^{m} p_k \cdot \exp{\left\{ i \cdot \frac{1}{\sqrt{n}}
           \cdot \left(\frac{\lambda_k}{\sqrt{p_k}}
        - \sum_{l=1}^{m}\lambda_l \cdot \sqrt{p_l} \right) \right\}}
  \end{align*}

  Добавим и отнимем единицу, возьмём предел и видим замечательный предел!
  \begin{align*}
      \lim_{n \to \infty} \left( 1 + \sum_{k=1}^{m} p_k \cdot \left[
      \exp{\left\{ i \cdot \frac{1}{\sqrt{n}}
           \cdot \left(\frac{\lambda_k}{\sqrt{p_k}}
        - \sum_{l=1}^{m}\lambda_l \cdot \sqrt{p_l} \right) \right\}}
      - 1 \right] \cdot \frac{n}{n} \right)^n = \\
      \lim_{n \to \infty} \left( 1 + \frac{x}{n} \right)^n = e^x
  \end{align*}

  Вытащим наше ``$x$'' из предела, но не забываем, что он содержит $n$,
  которое стремится к бесконечности
  \begin{align*}
      x = \lim_{n \to \infty} n \cdot \sum_{k=1}^{m} p_k \cdot \left[
      \exp{\left\{ i \cdot \frac{1}{\sqrt{n}}
           \cdot \left(\frac{\lambda_k}{\sqrt{p_k}}
        - \sum_{l=1}^{m}\lambda_l \cdot \sqrt{p_l} \right) \right\}}
      - 1 \right]
  \end{align*}

  Тут видим ещё одну экспоненту. Она хороша тем, что в ней происходит деление
  на бесконечно большое значение $\sqrt{n}$, а это значит, что допускается
  следующее разложение
  \begin{align*}
      e^{\alpha} - 1 \approx \alpha + \frac{\alpha^2}{2},\; \alpha \ll 1
  \end{align*}

  Рассмотрим сумму, которой принадлежит экспонента. Пока что подставим лишь
  $\alpha$, а квадрат уже потом
  \begin{align*}
      \sum_{k=1}^{m} p_k \cdot i
       \cdot \left(\frac{\lambda_k}{\sqrt{p_k}}
          - \sum_{l=1}^{m}\lambda_l \cdot \sqrt{p_l} \right)
      = i \cdot \sum_{k=1}^{m} \lambda_k \cdot \sqrt{p_k}
      - i \cdot \sum_{l=1}^{m} \lambda_l \cdot \sqrt{p_l}
      = 0
  \end{align*}

  Теперь рассмотрим квадрат
  \begin{align*}
      \lim_{n \to \infty} -\frac{n}{n}\sum_{k=1}^{m} p_k
      \cdot \left( \frac{\lambda_k}{\sqrt{p_k}}
      - \sum_{l=1}^{m}\lambda_l \cdot \sqrt{p_l} \right)^2
      = - \sum_{k=1}^{m}\lambda_k^2
      + \left( \sum_{k=1}^{m}\lambda_k \cdot \sqrt{p_k} \right)^2
  \end{align*}

  Получаем следующий результат
  \begin{align*}
      \lim_{n \to \infty} \mean{e^{
      i \cdot \left( \vec{\lambda}, \vec{\eta}_n \right)}}
      = \exp{\left\{ - \frac{1}{2} \cdot \left[ \sum_{k=1}^{n}\lambda_k^2
          - \left( \sum_{k=1}^{m}\lambda_k \cdot \sqrt{p_k}
        \right)^2\right] \right\}}
      = e^{-\frac{1}{2} \cdot \left(
          \operatorname{A} \vec{\lambda}, \vec{\lambda} \right)}
  \end{align*}

  Матрица $\operatorname{A}$ имеет следующий вид
  \begin{align*}
      \operatorname{A}
      = \left\| \delta_{ij} - \sqrt{p_i} \cdot \sqrt{p_j} \right\|_{i=1}^n
  \end{align*}

  Матрица $\operatorname{A}$ должна быть симметричная и неотрицательно
  определённая. Симметричность очевидна, неотрицательную определённость
  нужно проверить. С этой целью введём вектор $\vec{e}$
  \begin{align*}
      \vec{e} &= \left( \sqrt{p_1}, \dots, \sqrt{p_m} \right) \\
      \left\| e \right\| &= 1
  \end{align*}

  Очевидно следующее тождество
  \begin{align*}
      \left( \operatorname{A} \vec{\lambda}, \vec{\lambda} \right)
      = \left\| \vec{\lambda} \right\|^2
      - \left( \vec{\lambda}, \vec{e} \right)^2
  \end{align*}

  Далее используем неравенство Коши
  \begin{align*}
      \left| \left( \vec{\lambda}, \vec{e} \right) \right|
      \le \left\| \vec{\lambda} \right\| \cdot \left\| \vec{e} \right\|
      = \left\| \vec{\lambda} \right\|
  \end{align*}

  Делаем вывод, что матрица $\operatorname{A}$ неотрицательно определённая
  \begin{align*}
      \left( \operatorname{A} \vec{\lambda}, \vec{\lambda} \right) \ge 0
  \end{align*}

  Это значит, что вектор $\vec{\eta}_n$ имеет гауссовское распределение с
  ковариацией $\operatorname{A}$ при достаточно большом значении $n$
  \begin{align*}
      \vec{\eta}_n \Covergence{} N\left( \vec{0}, \operatorname{A} \right)
  \end{align*}

  Но откуда взялось распределение $\chi_{m-1}^2$ в теореме?

  Введём стандартный гауссовский вектор $\vec{\kappa}$
  \begin{align*}
      \vec{\kappa} \sim N\left( \vec{0}, \operatorname{I} \right)
  \end{align*}

  Из линейной алгебры помним, что если у нас есть ортонормированный базис,
  то при воздействии на него ортогональным оператором $\operatorname{U}$
  получим другой ортонормированный базис.

  Возьмём $\vec{f}_1, \dots, \vec{f}_m$ --- стандартный базис в
  $\mathbb{R}^m$
  \begin{align*}
      \vec{\kappa} = \sum_{k=1}^{m} \kappa_k \cdot \vec{f}_k,\;
      \kappa_k \sim N\left( 0, 1 \right)
  \end{align*}

  Введём новый базис $\vec{e} = \vec{e}_1, \vec{e}_2, \dots, \vec{e}_m$
  (обратим внимание на то, что для первого элемент базиса мы дали два имени
  --- с индексом и без него). Также есть $\operatorname{U}$ --- ортогональная
  матрица перехода $\left\{ \vec{f} \right\}$ к $\left\{ \vec{e} \right\}$.
  \begin{align*}
      \operatorname{U} \vec{\kappa}
      = \sum_{k=1}^{m} \hat{\kappa}_k \cdot \vec{e}_k
      \sim N\left( \vec{0}, \operatorname{U} \operatorname{U^*} \right)
      = N\left( \vec{0}, \operatorname{I} \right)
  \end{align*}

  Рассмотрим новый вектор $\vec{\xi}_*$, в который мы возьмём все элементы
  вектора $\left\{ \hat{\kappa}_k \right\}$ кроме первого. Отметим, что
  элементы вектора $\vec{\xi}_*$ независимы между собой
  \begin{align*}
      \vec{\xi}_* = \sum_{k=2}^{m} \hat{\kappa}_k \cdot \vec{e}_k
      \sim N\left( \vec{0}, ? \right)
  \end{align*}

  Посмотрим, чему равна квадратичная форма ковариационной матрицы вектора
  $\vec{\xi}_*$
  \begin{align*}
      \Mean{\left( \vec{\xi}_*, \vec{\lambda} \right) \cdot
      \left( \vec{\xi}_*, \vec{\lambda} \right)}
      = \sum_{k=2}^{m}\left( \vec{\lambda}, \vec{e}_k \right)^2
      = \sum_{k=1}^{m}\left( \vec{\lambda}, \vec{e}_k \right)^2
      - \left( \vec{\lambda}, \vec{e} \right)^2 = \\
      = \left\| \vec{\lambda} \right\|^2
      - \left( \vec{\lambda}, \vec{e} \right)^2
      = \left( \operatorname{A} \vec{\lambda}, \vec{\lambda} \right)
  \end{align*}

  То есть это и есть распределение вектора $\vec{\eta}_n$. Норма вектора
  $\vec{\xi}_*$ имеет распределение Пирсона с $m-1$ степенями свободы, что и
  требовалось доказать
  \begin{align*}
      \left\| \vec{\xi}_* \right\| = \sum_{k=2}^{m} \hat{\kappa}_k^2
      \sim \chi_{m-1}^2
  \end{align*}

  То есть линейной связью убиваем одну степень свободы.
\end{proof}

\subsection{Рецепт применения критерия Пирсона}
\index{критерий согласия!Пирсона!рецепт}
У нас есть $\alpha \in \left( 0; 1 \right)$, выборка \xsample и
предполагаемая функция распределения $F$.

\begin{enumerate}
  \item Разбиваем область значений выборки на $\Delta_1, \dots, \Delta_m$
  \item Вводим определение $p_i$
      \begin{align*}
      p_i = \Probability[F]{x_i \in \Delta_i}
      \end{align*}
  \item Считаем $R$
      \begin{align*}
      R = \sum_{k=1}^{m}\frac{\left( \nu_k - n \cdot p_k \right)^2}{
          n \cdot p_k}
      \end{align*}
  \item Находим $r_{\alpha}$ по таблице
      \begin{align*}
      \Probability{\chi_{m-1}^2 \le r_{\alpha}} = \alpha
      \end{align*}
  \item Если $R \le r_{\alpha}$, то ответ ``да''. Если же $R > r_{\alpha}$,
      то ответ ``нет''.
\end{enumerate}

Если мы не угадали распределение, то сумма будет вести себя как $\sqrt{n}$ ---
будет очень большой.
