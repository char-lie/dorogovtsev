\chapter*{Вступление}
\addcontentsline{toc}{chapter}{Вступление}

В курсе ``Теория вероятностей'' у нас было вероятностное пространство
$\left( \Omega, \mathcal{F}, \mathbb{P} \right)$ и случайные величины
$\left( \xi, \eta, \varkappa, \dots \right)$ с известными распределениями.
Нас интересовало поведение этих случайных величин.
Теперь мы будем заниматься выяснением распределения случайных величин.

$x_1, \dots, x_n$ --- выборка из распределения $F$. То есть, это независимые
одинаково распределённые случайные величины с неизвестной функцией распределения
$F$.

Предмет математической статистики --- выяснение характера вероятностного
эксперимента, отвечающего данной модели.
В настоящем курсе будут рассматриваться такие задачи:
\begin{enumerate}
  \item
    $x_1, \dots, x_n$ --- значения, полученные в результате измерения случайной
    величины.
    Какое распределение имеет эта величина?
  \item
    Наблюдается последовательность $x_1, \dots, x_n$.
    Являются ли случайные величины $x_1, \dots, x_n$ реализацией независимых
    одинаково распределённых случайных величин?
  \item
    Есть две гипотезы относительно типа распределения.
    Какая из гипотез верна?
  \item
    Существует ли связь между параметрами $x$ и $y$ и каков её вид?
\end{enumerate}
