\chapter*{Вступление}
\addcontentsline{toc}{chapter}{Вступление}

В курсе теории вероятностей мы рассматривали вероятностное пространство
$\left( \Omega, \mathcal{F}, \mathbb{P} \right)$ и случайные величины
$\left( \xi, \eta, \varkappa, \dots \right)$ с известными распределениями.
Нас интересовало поведение этих случайных величин.
Математическая статистика исследует характер распределений случайных величин.

Пусть \xsample --- независимые одинаково распределённые случайные величины с
неизвестной функцией распределения $F$.
Такой набор случайных величин называют выборкой из распределения $F$.
В каждом конкретном вероятностном экспериемнте мы наблюдаем некоторую
реализацию выборкки, то есть набор значений \xwsample при некотором
$\omega \in \Omega$.
Математическаая статистика занимается восстановлением характера распределений
из реализаций выборок.
Там, где это не вызовет путаницы, мы будем опускать символ $\omega$ в записи
\xwsample и употреблять термин ``выборка'' в значении ``реализация выборки''.

В настоящем курсе будут рассматриваться такие задачи:
\begin{enumerate}
  \item
    \xsample --- значения, полученные в результате измерения случайной
    величины.
    Какое распределение имеет эта величина?
  \item
    Наблюдается последовательность \xsample.
    Являются ли величины \xsample реализацией независимых
    одинаково распределённых случайных величин?
  \item
    Есть две гипотезы относительно типа распределения \xsample.
    Какая из гипотез верна?
  \item
    Существует ли зависимость наблюдаемых значений от каких-либо других величин?
    Каков вид этой зависимости?
\end{enumerate}
