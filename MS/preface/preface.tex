\chapter*{Вступление}
\addcontentsline{toc}{chapter}{Вступление}

В курсе теории вероятностей у нас было вероятностное пространство
$\left( \Omega, \mathcal{F}, \mathbb{P} \right)$ и случайные величины
$\left( \xi, \eta, \varkappa, \dots \right)$ с известными распределениями.
Нас интересовало поведение этих случайных величин.
Математическая статистика занимается выяснением распределения случайных величин.

\xsample --- выборка из распределения $F$. То есть это независимые
одинаково распределённые случайные величины с неизвестной функцией распределения
$F$.

Предмет математической статистики --- выяснение характера вероятностного
эксперимента, отвечающего данной модели.
В настоящем курсе будут рассматриваться такие задачи:
\begin{enumerate}
  \item
    \xsample --- значения, полученные в результате измерения случайной
    величины.
    Какое распределение имеет эта величина?
  \item
    Наблюдается последовательность \xsample.
    Являются ли величины \xsample реализацией независимых
    одинаково распределённых случайных величин?
  \item
    Есть две гипотезы относительно типа распределения \xsample.
    Какая из гипотез верна?
  \item
    Существует ли связь между параметрами модели $x$ и $y$?
    Каков вид этой зависимости?
\end{enumerate}
