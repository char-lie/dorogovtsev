\documentclass{standalone}
\usepackage{tikz}
\usepackage{verbatim}
\usetikzlibrary{calc}

\usetikzlibrary{arrows}

\begin{document}
\begin{tikzpicture}[scale=5]

    %% axes
    \draw[->] (-0.1,0) -- (1.1,0) node[right]{$X$};
    \draw[->] (0,-0.1) -- (0,1.1) node[above]{$F_n$};

    %% sample to draw
    \def\sample{0.1/1,0.2/2,0.5/1}
    \def\n{4}
    %\pgfmathparse{dim({\sample})}
    %\let\n\pgfmathresult;

    %% set initial variables
    \edef\startheight{0};
    \edef\samplessum{0};
    \coordinate (previous point) at (-.1,0);
    \coordinate (center) at (0,0);

    % loop by every sample element
    \foreach \x/\samples [count=\xi] in \sample{
      %% calculate current 'bar' height
      \pgfmathparse{\samples/\n}
      \let\height\pgfmathresult;
      %% set current line end
      \coordinate (current point) at ($(\x,\startheight+\height)$);
      %% label current step
      \node[above] at (current point) {\x};
      %% draw arrow
      \draw[->] (previous point) -- (current point |- previous point);
      %% draw hipe
      \draw[dashed] (current point |- previous point) -- (current point);
      %% set current point as previous
      \coordinate (previous point) at (current point);
      %% calculate next 'bar' start y coordinate
      \pgfmathparse{\startheight+\height}
      \xdef\startheight{\pgfmathresult};
      \pgfmathparse{int(\samplessum+\samples)}
      \xdef\samplessum{\pgfmathresult};
      %% draw vertical axis label
      \node[left] at (center |- previous point) {$\frac{\samplessum}{\n}$};
      \draw[dotted] (center |- previous point) -- (previous point);
    }
    %% add last lines in the top
    \coordinate (current point) at (1,1);
    \draw[-] (previous point |- current point) -- (current point);

\end{tikzpicture}
\end{document}
