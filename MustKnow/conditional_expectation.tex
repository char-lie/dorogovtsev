\part{Свойства условного математического ожидания}
\begin{enumerate}[label=\Roman*]
    \item Формула полной вероятности
        $$\mean{\Mean{ \eta \mcond \mathfrak{F}_1 }} = \mean{\eta}$$
    \item Условное математическое ожидание неотрицательной случайной величины
        неотрицательно почти наверное
        $$\eta \ge 0
            \Rightarrow \Mean{ \eta \mcond \mathfrak{F}_1 } \ge 0$$
    \item Неравенство Йенсена. Если функция $\varphi$ выпуклая вниз, то
        $$\varphi\left( \Mean{ \eta \mcond \mathfrak{F}_1 } \right)
            \le \Mean{\varphi\left( \eta \right) \mcond \mathfrak{F}_1}$$
    \item Теорема о трёх перпендикулярах
        $$\mathfrak{F}_2 \subset \mathfrak{F}_1 \Rightarrow
            \Mean{ \mean{\left( \eta \mcond \mathfrak{F}_1 }
                \mcond \mathfrak{F}_2 \right)}
            = \Mean{ \eta \mcond \mathfrak{F}_2 }$$
    \item Если случайная величина $\eta$ измерима
        относительно $\sigma$-алгебры $\mathfrak{F}_1$,
        то её условное математическое ожидание равно ей самой
        $$\Mean{ \eta \mcond \mathfrak{F}_1 } = \eta$$
    \item Если случайная величина $\eta$ измерима
        относительно $\mathfrak{F}_1$, то для любой случайной величины $\xi$
        $$\Mean{ \eta \cdot \xi \mcond \mathfrak{F}_1 }
            = \eta \cdot \Mean{ \xi \mcond \mathfrak{F}_1 }$$
    \item Если $\eta$ не зависит от $\mathfrak{F}_1$,
        то её условное математическое ожидание
        равно простому математическому ожиданию
        $$\forall \Delta \in \mathfrak{B}, A \in \mathfrak{F}_1:
            \probability{ \left\{ \eta \in \Delta \right\} \mcond A}
                = \left\{ \eta \in \Delta \right\}
            \Rightarrow \Mean{ \eta \mcond \mathfrak{F}_1 }
                = \mean{\eta}$$
    \item Условное математическое ожидание линейно
        $$\forall a, b \in \mathbb{R}:
            \Mean{a \cdot \xi + b \cdot \eta \mcond \mathfrak{F}_1}
                = \Mean{a \cdot \xi \mcond \mathfrak{F}_1}
                    + \Mean{b \cdot \eta \mcond \mathfrak{F}_1}$$
    \item Сохраняется теорема Лебега о возможности предельного перехода
        под знаком условного математического ожидания
        $$\left|\xi_n\right| \le \eta,
            \;\mean{\eta} < \infty,
            \;\xi_n \acovergence \xi
            \Rightarrow
            \Mean{\xi_n \mcond \mathfrak{F}_1}
                \acovergence \Mean{\xi \mcond \mathfrak{F}_1}$$
\end{enumerate}

Пара полезных частных случаев неравенства Йенсена
$$\begin{array}{crcl}
    \varphi\left( x \right) = \left|x\right|:&
        \left| \Mean{\eta \mcond \mathfrak{F}_1} \right|
            &\le& \Mean{\left| \eta \right| \mcond \mathfrak{F}_1} \\
    \varphi\left( x \right) = x^2:&
        \left( \Mean{\eta \mcond \mathfrak{F}_1} \right)^2
            &\le& \Mean{\eta^2 \mcond \mathfrak{F}_1}
\end{array}$$
