\chapter{Основы}

\section{Предисловие}
Теория вероятностей --- наука о \textbf{повторяющихся случайных} явлениях.

Примеры: квантовая физика, вихри в турбулентных потоках,
изменения молекулы ДНК в жидкостях, ...

Цель: \textbf{прогноз}. Мы, исследуя случайные явления,
будем пытаться что-то предсказать.

Нас интересует то, что случается с нами периодически.

При исследовании повторяющихся случайных явлений мы исследуем
количественные характеристики и закономерности

\section{Вероятностный эксперимент}
\begin{definition}[Вероятностный эксперимент]
    \index{вероятностный эксперимент}
    Вероятностный эксперимент --- явление,
    исход которого для нас не определён
    и который можно повторить любое число раз \textbf{независимым} образом.
\end{definition}
\begin{definition}[Вероятностное пространство]
    \index{вероятностное пространство}
    Вероятностное пространство --- совокупность всех исходов
    вероятностного эксперимента.
\end{definition}
$\Omega$ --- вероятностное пространство

$\Omega \ni \omega$ --- элементарный исход

\begin{example}[Подбрасывание монетки]
Выпадение орла $\left( 0 \right)$
и выпадение решки $\left( 1 \right)$ --- элементарные исходы
$$\Omega = \left\{ 0, 1 \right\}$$
\end{example}
\begin{example}[Подбрасывание игрального кубика]
Количество выпавших очков --- элементарный исход
$$\Omega = \left\{ 1, \dots, 6 \right\}$$
\end{example}
\begin{example}[Задача о встрече]
Два человека в течение часа один раз приходят на одно и то же место.

Пара $\left( x,y \right)$ --- элементарный исход
(время, в которое первый и второй человек появится в указанном месте)
$$\Omega = \left\{ \left(x,y\right) \mid x,y \in [0;1] \right\}$$
\end{example}
\begin{example}[Монета подбрасывается до первого появления орла]
    Элементарный исход --- номер подбрасывания, на котором выпадет орёл
$$\Omega = \left\{ 1,2, \dots \right\} =  \mathbb{N}$$
\end{example}
\section{Случайные события и действия над ними}
\begin{definition}[Случайное событие]
\index{случайные события}
Случайное событие --- подмножество множества всех исходов
вероятностного эксперимента.
\end{definition}
$\Omega \supset A$ --- случайное событие

$\Omega$ --- достоверное событие

$\emptyset$ --- невозможное события

\begin{example}[Подбрасывание монетки]
$A$ -- выпал орёл
$$A=\left\{0\right\}$$
\end{example}
\begin{example}[Подбрасывание игрального кубика]
$A$ --- выпало чётное число
$$A = \left\{ 2,4,6 \right\}$$
\end{example}
\begin{example}[Задава о встрече]
$A$ --- встретились
\end{example}
\begin{example}[Монета подбрасывается до первого появления орла]
$A$ --- не потребовалось 100 подбрасываний
$$A = \left\{ 1, \dots ,99 \right\}$$
\end{example}

Можно говорить о \textbf{пересечении} событий $A$ и $B$,
но лучше говорить, что они \textbf{произошли одновременно}.

$\emptyset$ --- невозможное событие

$\Omega$ --- достоверное события

$\stcomp{A}$ --- событие $A$ не произошло

$A\cap B$ --- $A$ и $B$ произошли

$A\cup B$ --- произошло хотя бы одно из событий $A$, $B$

$A\setminus B$ --- произошло $A$, но не $B$

\begin{definition}[Последовательность случайных событий]
    \index{последовательность!случайных событий}
Последовательность случайных событий
$$\left\{A_n\mid n\geq 1\right\}$$
\end{definition}
\begin{definition}[Верхний предел последовательности случайных величин]
    \index{предел!верхний!последовательности случайных событий}
    \index{последовательность!случайных событий!верхний предел}
    Верхний предел последовательности случайных событий
    $\left\{A_n\mid n\geq 1\right\}$ --- это
    случайное событие, состоящее в том,
    что произошло бесконечно много событий из исходной последовательности.
    $$\varliminf_{n \to \infty} A_n
    = \bigcup_{n=1}^{\infty} \bigcap_{m=n}^{\infty} A_m $$
\end{definition}
\begin{definition}[Нижний предел последовательности случайных величин]
    \index{предел!нижний!последовательности случайных событий}
    \index{последовательность!случайных событий!нижний предел}
    Нижний предел последовательности случайных событий
    $\left\{A_n\mid n\geq 1\right\}$ --- это
    случайное событие, состоящее в том, что произошли все события
    из исходной последовательности.
    $$ \varliminf_{n \to \infty} A_n
    = \bigcap_{n=1}^{\infty} \bigcup_{m=n}^{\infty} A_m $$
\end{definition}

Некоторые свойства:
$$\varliminf_{n \to \infty} A_n \subset \varlimsup_{n \to \infty} A_n $$
$$\stcomp{\varlimsup_{n \to \infty} A_n}
= \varliminf_{n \to \infty} \stcomp{A_n}$$
$$\stcomp{\varliminf_{n \to \infty} A_n}
= \varlimsup_{n \to \infty} \stcomp{A_n} $$

\section{Классический вероятностный эксперимент}
\begin{definition}[Классический вероятностный эксперимент]
    \index{вероятностный эксперимент!классический}
Классический вероятностный эксперимент --- такой, у которого
конечное число исходов,
ни один из которых не является более предпочтительным, чем другой.
\end{definition}

У нас появилось желание приписывать числовые характеристики событиям.

$\left|\Omega\right| < +\infty$,
все элементарные исходы равновероятны,
$\Omega \supset A$ --- случайное событие.

\begin{definition}[Вероятность]\index{вероятность}
    Вероятность того, что случайное событие $A\subset\Omega$ произошло,
    является отношением мощности множества $A$ к мощности множества $\Omega$:
    $$\probability{A}=\frac{\left|A\right|}{\left|\Omega\right|}$$
\end{definition}
\begin{remark}[Очевидные свойства]
    Нетрудно заметить такие очевидные свойства вероятности

    \begin{enumerate}
        \item Вероятность невозможного события (пустого множества) равна нулю
            $$\probability{\emptyset}
                =\frac{\left|\emptyset\right|}{\left|\Omega\right|}=0$$
        \item Вероятность достоверного события (вероятностного пространства)
            равна единице
            $$\probability{\Omega}
                =\frac{\left|\Omega\right|}{\left|\Omega\right|}=1$$
    \end{enumerate}
\end{remark}

Рецепт вычисления вероятности классического эксперимента:
\begin{enumerate}
\item Описать $\Omega$
\item Описать события $A$
    (исходы, благоприятный событию $A$, состоят в том, что ...)
\item Найти $\left|\Omega\right|, \left|A\right|, \probability{A}
    =\frac{\left|A\right|}{\left|\Omega\right|}$
\end{enumerate}

\begin{theorem}[Формула включения-исключения]
    \index{формула!включения-исключения}
Есть $n$ случайных событий: $A_1, A_2, \dots, A_n \subset \Omega$.
Вероятность того, что произошло хотя бы одно событие, считается по формуле:
\begin{align*}
    \probability{\bigcup_{k=1}^n A_k}=
    \sum_{i_1=1}^n \probability{A_{i_1}} -
    \sum_{1\le i_1<i_2}^n \probability{A_{i_1} \cap A_{i_2}}
    + \sum_{1\le i_1<i_2<i_3}^n \probability{\bigcap_{k=1}^3 A_{i_k}} -\\
    - \dots +
    \left(-1\right)^{n+1} \cdot \probability{\bigcap_{k=1}^n A_k} =
     \sum_{k=1}^n
        \left[ \left(-1\right)^{k+1} \cdot
            \sum_{1\le i_1<\dots<i_k}^n \probability{\bigcap_{t=1}^k A_{i_t}}
        \right]
\end{align*}
\end{theorem}
\begin{comment}
\begin{proof}
    Докажем методом математической индукции.
    \begin{enumerate}
        \item Очевидно, что тождество выполняется для одного события ($n=1$)
            $$\probability{A}=\probability{A}$$
        \item Допустим, что тождество выполняется для некоего $n=N$, то есть
            $$\probability{\bigcup_{k=1}^N A_k}=
                 \sum_{k=1}^N
                    \left[ \left(-1\right)^{k+1} \cdot
                        \sum_{1\le i_1<\dots<i_k}^N
                            \probability{\bigcap_{t=1}^k A_{i_t}}
                    \right]$$

            Остаётся проверить, что оно справедливо и для $n=N+1$
            \begin{align*}
                \probability{\bigcup_{k=1}^{N+1} A_k}
                =\Probability{\left(\bigcup_{k=1}^{N} A_k\right)\cup A_{N+1}}
                =\Probability{\bigcup_{k=1}^{N}\left(A_k\cup A_{N+1}\right)}
            \end{align*}

            Воспользуемся формулой включения-исключения для двух событий
            и дистрибутивностью пересечения и объединения множеств
            \begin{align*}
                &\Probability{
                    \left(\bigcup_{k=1}^{N} A_k\right)\cup A_{N+1}}=\\
                    &=\probability{\bigcup_{k=1}^{N} A_k}
                        + \probability{A_{N+1}}
                        - \Probability{\left(\bigcup_{k=1}^{N} A_k\right)
                            \cap A_{N+1}}\\
                    &=\probability{\bigcup_{k=1}^{N} A_k}
                        - \Probability{\bigcup_{k=1}^{N} \left(A_k
                            \cap A_{N+1}\right)}
                        + \probability{A_{N+1}}
            \end{align*}
            
            Далее используем то, что тождество справедливо для $N$
            \begin{equation}\label{eq:inc_exc_sum1}
                \probability{\bigcup_{k=1}^{N} A_k}
                =\sum_{k=1}^N
                    \left[ \left(-1\right)^{k+1} \cdot
                        \sum_{1\le i_1<\dots<i_k}^N
                            \probability{\bigcap_{t=1}^k A_{i_t}}
                    \right]
            \end{equation}

            \begin{equation}\label{eq:inc_exc_sum2}
                \begin{split}
                    &\Probability{\left(\bigcup_{k=1}^{N} A_k\right)
                        \cap A_{N+1}}=\\
                    &=\sum_{k=1}^N
                        \left[ \left(-1\right)^{k+1} \cdot
                            \sum_{1\le i_1<\dots<i_k}^N
                                \probability{
                                    \bigcap_{t=1}^k \left(A_{i_t}
                                        \cap A_{N+1}\right)}
                        \right]
                \end{split}
            \end{equation}

            Теперь посмотрим на разность выражений
            \eqref{eq:inc_exc_sum1} и \eqref{eq:inc_exc_sum2}
            \begin{equation}
                \begin{split}
                    &\probability{\bigcup_{k=1}^{N} A_k}-
                    \Probability{\left(\bigcup_{k=1}^{N} A_k\right)
                        \cap A_{N+1}}=\\
                    &=\sum_{k=1}^N
                    \left[ \left(-1\right)^{k+1} \cdot
                        \sum_{1\le i_1<\dots<i_k}^N
                            T_{k}
                        \right],\\
                    &T_{k}=\probability{\bigcap_{t=1}^k A_{i_t}}
                        -\probability{
                            \bigcap_{t=1}^k \left(A_{i_t}
                                \cap A_{N+1}\right)}
                \end{split}
            \end{equation}

            Посмотрим на величины $T_{k}$ и $T_{k+1}$
            \begin{align*}
                T_{k}&=\probability{\bigcap_{t=1}^k A_{i_t}}
                    -\probability{
                        \bigcap_{t=1}^k \left(A_{i_t}
                            \cap A_{N+1}\right)}\\
                T_{k+1}&=\probability{\bigcap_{t=1}^{k+1} A_{i_t}}
                            -\probability{
                                \bigcap_{t=1}^{k+1} \left(A_{i_t}
                                    \cap A_{N+1}\right)}=\\
                       &=\probability{\bigcap_{t=1}^{k} \left(A_{i_t}
                                \cap A_{k+1}\right)}
                            -\probability{
                                \bigcap_{t=1}^{k+1} \left(A_{i_t}
                                    \cap A_{N+1}\right)}
            \end{align*}
            
            Обратим внимание на элементы
            $$\probability{\bigcap_{t=1}^{k} \left(A_{i_t}
                                \cap A_{k+1}\right)}$$

    \end{enumerate}
\end{proof}
\end{comment}

\begin{example}[Задача о конвертах]
Дано $n$ писем и $n$ конвертов. Письма раскладываются случайным образом.
Найти вероятность того, что хотя бы одно письмо попало по правильному адресу.

Случайно раскладываем $n$ писем по $n$ конвертам --- получаем перестановки.

$\Omega$ --- все перестановки чисел от $1$ до $n$

$A_i$ --- событие, состоящее в том, что
$i$-е письмо попало по правильному адресу

\begin{align*}
    \probability{\bigcup_{i=1}^n A_i } &=
    \sum_{i=1}^n \frac{\left(n-1\right)!}{n!} -
        \sum_{i_1<i_2} \frac{\left(n-2\right)!}{n!} + \dots +
        \left(-1\right)^{k+1}\cdot
        \sum_{i_1<\dots<i_k} \frac{\left(n-k\right)!}{n!}=\\
    &=
    \binom{1}{n}\cdot\frac{1}{n}-
        \binom{2}{n}\cdot\frac{1}{n\cdot\left(n-1\right)}+\dots+
        \left(-1\right)^{k+1}
            \cdot\binom{k}{n}\cdot\frac{\left(n-k\right)!}{n!}+\\
        &+\dots =
    \frac{1}{1!}-\frac{1}{2!}+\frac{1}{3!}-\dots+
        \frac{\left(-1\right)^{n+1}}{n!}
        \xrightarrow[\text{$n\to\infty$}]{} 1-\frac{1}{e}
\end{align*}
\end{example}
\begin{remark}[Чуть менее очевидное свойство вероятности]
    Вероятность того,
    что произошло хотя бы одно из событий,
    которые не могут произойти одновременно,
    равна сумме их вероятностей
    $$A\cap B=\emptyset
        \Rightarrow \probability{A\cup B}
            =\probability{A}+\probability{B}$$

    Частный случай --- случайное событие и его дополнение
    $$A\cap \stcomp{A}=\emptyset
        \Rightarrow\probability{A}=1-\probability{\stcomp{A}}$$
    Так как вероятность того, что случайное событие
    хотя бы произошло или не произошло, равна единице
    (объединение случайного события с его дополнением ---
    вероятностное пространство)
    \begin{align*}
        \begin{cases}
            \probability{A\cup\stcomp{A}}=\probability{\Omega}=1\\
            \probability{A\cup\stcomp{A}}
                =\probability{A}+\probability{\stcomp{A}}
        \end{cases}
        \Rightarrow
            \probability{A}+\probability{\stcomp{A}}=1
        \Rightarrow
            \probability{A}=1-\probability{\stcomp{A}}
    \end{align*}
\end{remark}
\begin{proof}
    Начнём с условия и определения вероятности
    $$A\cap B=\emptyset
        \Rightarrow \probability{A\cup B}
        =\frac{\left|A\cup B\right|}{\left|\Omega\right|}$$
    Воспользуемся формулой включения-исключения
    $$\frac{\left|A\cup B\right|}{\left|\Omega\right|}
        =\frac{\left|A\right|+\left|B\right|-\left|A\cap B\right|}
            {\left|\Omega\right|}
        =\frac{\left|A\right|}{\left|\Omega\right|}
            +\frac{\left|B\right|}{\left|\Omega\right|}$$
    А теперь определением вероятности
    $$\frac{\left|A\right|}{\left|\Omega\right|}
            +\frac{\left|B\right|}{\left|\Omega\right|}
        =\probability{A}+\probability{B}$$
    То есть, получили то, что ожидали
    $$A\cap B=\emptyset
        \Rightarrow \probability{A\cup B}
            =\probability{A}+\probability{B}$$

\end{proof}

\section{Геометрический вероятностный эксперимент}
\index{вероятностный эксперимент!геометрический}
\begin{example}[Простейший геометрический эксперимент]
    $G\subset D\subset \mathbb{R}^n$, наугад выбираем точку $M\in D$.
    Событие $A$ состоит в том, что $M\in A$. Тогда вероятность события $G$:
    $$\probability{A}=\frac{\left|G\right|}{\left|D\right|}$$
    $\left|D\right|$ --- длина, площадь, объём $D$ (в зависимости от $n$)
\end{example}
Вероятность в геометрическом эксперименте обладает теми же свойствами,
что и в классическом, но появились ненулевые события с нулевой вероятностью
$$(\exists A\subset\Omega)(A\neq\emptyset): \probability{A}=0$$
\begin{example}[Непустое событие с нулевой вероятностью]
    Вероятность, события $A$, состоящего в том,
    что точка попадёт на линию (или точку) в трёхмерном пространстве,
    равна нулю, так как объём линии(точки) равен нулю
    (но такое событие возможно)
    $$\probability{A}=\frac{0}{\left|D\right|}=0$$
\end{example}

\begin{definition}
    Для $A\subset\mathbb{R}$
    $$A+x=\left\{y+x\mid y\in A\right\}$$
\end{definition}

Свойства длины. Разбиваем числовую прямую на непересекающиеся отрезки $A_k$:
        $$A_k=\left[a_k;b_k\right], a_k < b_k < a_{k+1}$$
\begin{enumerate}
    \item Аддитивность
        $$\left|\bigcup_{k=1}^n A_k\right|
        =\sum_{k=1}^n\left|A_k\right| , A_i\cap A_j=\emptyset, i\neq j$$
    \item Однородность Эвклидового пространства:
        длина множества, сдвинутого на $x$, равна длине исходного множества
        $$\left|A+x\right|=\left|A\right|$$
\end{enumerate}
\begin{affirmation}
    Нельзя определить функцию множества, обладающую свойствами 1 и 2,
    на всех подмножествах числовой прямой.
\end{affirmation}
\begin{proof}
    Пусть $\left|\cdot\right|$ определена на всех подмножествах $\mathbb{R}$ и
    обладает свойствами 1 и 2.

    Возьмём отрезок $\left[0;1\right]$ и введём
    отношение эквивалентности ``$\sim$''.
    Будем считать эквивалентными числа,
    разность между которыми --- число рациональное.
    $$\forall x,y\in\left[0;1\right]: x\sim y
    \Leftrightarrow x-y\in\mathbb{Q}$$

    Тогда множество классов эквивалентности $\left\{K_{\alpha}\right\}$
    будет разбиением отрезка $\left[0;1\right]$:
    \begin{align*}
        \bigcup_{\alpha} K_{\alpha}&=\left[0;1\right]\\
        K_{\alpha_1} \cap K_{\alpha_2}&=\emptyset,\alpha_1\neq\alpha_2
    \end{align*}

    $A$ --- множество, в котором находится по одному представителю
    из каждого класса эквивалентности. Значит, $\left|A\right|\le 1$.

    Введём $\left\{r_k\right\}$ --- множество рациональный чисел
    на отрезке $\left[-1;1\right]$:
    $$\left\{r_k\right\}=\mathbb{Q} \cap \left[-1;1\right]$$

    Поскольку в $A$ содержится по одному представителю
    из каждого класса эквивалентности $K_{\alpha}$,
    разность между числами в каждом классе --- число рациональное,
    а сами числа взяты из отрезка $\left[0;1\right]$,
    то разность между двумя числами не может быть
    больше, чем $1$ и меньше, чем $-1$, и является числом рациональным.
    Значит, объединив все возможные суммы
    чисел из $A$ с числами из $\left\{r_k\right\}$,
    точно можно получить все числа из каждого класса $K_{\alpha}$,
    а, следовательно, и отрезок $\left[0;1\right]$:
    $$\bigcup_{k\ge 1} \left(A+r_k\right) \supset \left[0;1\right]$$

    Очевидно, что произвольное число $x$ из отрезка $\left[0;1\right]$
    будет принадлежать какому-то классу эквивалентности $K_{\alpha}$ и
    найдётся такое число $y$ из этого класса,
    принадлежащее множеству $A$ (по определению множества $A$):
    \begin{equation*}
        \begin{aligned}
            x\in\left[0;1\right] &\Rightarrow \exists\alpha: \left\{
            \begin{aligned}
                 x&\in K_{\alpha}\\
                 \exists y&\in A\cap K_{\alpha}
            \end{aligned}\right.\\
           %$x-y&=r_k\Rightarrow x\in A+r_k$
        \end{aligned}
    \end{equation*}

    Это в свою очередь означает, что разность $x$ и $y$ --- рациональное число
    из множества $\left\{r_k\right\}$, так как они принадлежат
    одному классу эквивалентности $K_{\alpha}$:
           $$x-y=r_k\Rightarrow x\in A+r_k$$

    \textbf{Примечание} к применению свойства 2:
    множество $\left\{A+r_k\right\}$ необязательно является разбиением,
    поэтому мощность объединений его элементов
    может быть меньше суммы их мощностей
    \begin{equation}\label{eq:A_bigger_then_one}
        \left|\left[0;1\right]\right|=1
        \le\left|\bigcup_{k\ge 1} \left(A+r_k\right)\right|
        \le\sum_{k\ge 1} \left|A+r_k\right|
        =\sum_{k\ge 1} \left|A\right|
    \end{equation}
    
    Поскольку все числа из $A$ попадают на отрезок $\left[0;1\right]$,
    а $r_k\in\left[ -1;1\right]$,
    то ни одно из чисел множества $\left\{A+r_k\right\}$ не попадёт за пределы
    отрезка $\left[-1;2\right]$:
    $$\bigcup_{k\ge 1} \left(A+r_k\right)\subset\left[-1;2\right]
    \Rightarrow \left|\bigcup_{k\ge 1} \left(A+r_k\right)\right|\le 3
    \Rightarrow \left|\bigcup_{k=1}^n \left(A+r_k\right)\right|\le 3,
    \forall n\ge1$$
    
    Если взять два разных рациональных числа $r_{k_1}$ и $r_{k_2}$,
    то $A+r_{k_1}$ и $A+r_{k_2}$ не будут иметь одинаковых элементов,
    так как, если разность между двумя числами --- число рациональное,
    то они принадлежат одному и тому же классу эквивалентности,
    но множество $A$ содержит лишь по одному представителю
    каждого класса эквивалентности $K_{\alpha}$:
    $$k_1\neq k_2\Rightarrow
    r_{k_1}\neq r_{k_2}\Rightarrow
    A+r_{k_1} \cap A+r_{k_2} = \emptyset$$

    А это значит, что множество $\left\{A+r_k\right\}$ является разбиением
    и к нему применимо свойство 2:
    $$
    3\ge\left|\bigcup_{k=1}^n \left(A+r_k\right)\right|
    =\sum_{k=1}^n \left|A+r_k\right|
    =n\cdot\left|A\right|, \forall n\ge 1$$

    Поскольку это действительно для любого натурального $n$, то
    $\left|A\right|$ может равняться лишь нулю,
    чтобы произведение не могло превышать $3$,
    но из \eqref{eq:A_bigger_then_one} имеем,
    что эта сумма (произведение) это значение должно быть не меньше, чем $1$:
    $$\left|A\right|=0
    \Rightarrow \sum_{k=1}^\infty \left|A\right|=0\not\ge 1$$

    Противоречие.
\end{proof}
Пока что не будут разглядываться непустые множества без длины.
