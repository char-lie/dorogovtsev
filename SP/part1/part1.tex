\chapter{Азы}
\section{Определение случайного процесса}
\index{случайный процесс}

\begin{definition}[Случайный процесс]
  \index{случайный процесс}
  \index{параметрическое множество}
  \index{множество!параметрическое}
  Случайные процесс с параметрическим множеством $T$ --- совокупность случайных
  величин $\xi_t$, зафиксированных элементами $t$ множества $T$
\end{definition}

То есть, случайный процесс является отображением из декартового произведения
множества элементарных исходов и параметрического множества на множество
действительных чисел
\begin{equation*}
  \xi: \Omega \times T \rightarrow \mathbb{R}
\end{equation*}
Также можно представить случайный процесс как случайную величину в вероятностном
пространстве
\begin{equation*}
  \left( \Omega \times T,
    \mathfrak{F} \otimes \mathfrak{B}\left( \mathbb{R} \right), \mathbb{P}
    \right),
\end{equation*}
где множество <<случайных событий>> построено следующим образом
\begin{equation*}
  \forall t \in T, \omega \in \Omega:\qquad
  \left\{ \left( t, \omega \right) \mcond \xi_t\left( \omega \right)
      \in \Delta \right\}
    \in \mathfrak{F} \otimes \mathfrak{B}\left( \mathbb{R} \right)
\end{equation*}

\begin{remark}[Случайный процесс с дискретным временем]
  \index{случайный процесс!дискретное время}
  Если $T = \mathbb{N}$ или $T = \mathbb{Z}$, то $\xi$ --- случайный процесс с
  дискретным временем.
\end{remark}

\begin{remark}[Случайный процесс с непрерывным временем]
  \index{случайный процесс!непрерывное время}
  Если же $T = \left[ 0; +\infty \right]$, $T = \left[ a; b \right]$ или
  $T = \mathbb{R}$, то $\xi$ --- случайный процесс с непрерывным временем.
\end{remark}

\begin{definition}[Траектория случайного процесса]
  \index{случайный процесс!реализация}
  \index{случайный процесс!траектория}
  Для фиксированного $\omega_0 \in \Omega$ функция $\xi\left( \omega_0 \right)$
  назыввается реализацией или траекторией случайного процесса, соответствующей
  исходу $\omega_0$
\end{definition}

\begin{definition}[Сечение случайного процесса]
  \index{случайный процесс!сечение}
  Если $t_0 \in T$ фиксировано, то случайная величина $\xi_{t_0}$ называется
  сечением случайного процесса в точке $t_0$
\end{definition}

\begin{example}
  Пусть $\xi_n$ --- последовательность случайных величин.
  Тогда $\xi$ --- случайный процесс с параметрическим множеством $\mathbb{N}$
\end{example}

\begin{example}
  Рассмотрим процесс появления случайного события с параметрическим множеством
  $T = \left[ 0; +\infty \right)$.
  Пусть $\tau$ --- неотрицательная случайная величина, а случайный процесс $\xi$
  определён следующим образом
  \begin{equation*}
    \xi_t\left( \omega \right)
    = \begin{cases}
        1, &t \ge \tau\left( \omega \right) \\
        0, &t < \tau\left( \omega \right)
      \end{cases}
  \end{equation*}
  или же, что то же
  \begin{equation*}
    \xi_t = \Indicator{\tau \le t}
  \end{equation*}

  Проверим, что $\xi_t$ действительно является случайной величиной при любом
  $t \in T$.
  Рассмотрим множество $A$
  \begin{equation*}
    A = \left\{ \omega \mcond \tau\left( \omega \right) \le t \right\}
  \end{equation*}
  Оно является случайным событием по определению, так как $\tau$ является
  случайной величиной.
  Рассмотрим прообразы индикатора случайного события $A$
  \begin{equation*}
    \left\{ \Indicator{\omega \in A} \le x \right\}
    = \begin{cases}
      \emptyset, &x < 0 \\
      \stcomp{A}, &0 \le x < 1 \\
      \Omega, &x \ge 1
    \end{cases}
  \end{equation*}
  Это значит, что
  \begin{equation*}
    \xi_t = \Indicator{\tau \le t}
  \end{equation*}
  действительно является случайной величиной.
\end{example}
