\documentclass[10pt,a4paper,titlepage]{book}
%\documentclass[12pt,b5paper,titlepage]{book}
%\documentclass[10pt,a4paper,titlepage,twocolumn]{book}
\usepackage[utf8]{inputenc}
\usepackage{cmap}
\usepackage[russian]{babel}
\usepackage[OT1]{fontenc}
\usepackage{amsmath}
\usepackage{amsfonts}
\usepackage{amssymb}
\usepackage{amsthm}
\usepackage{graphicx}
\usepackage{indentfirst}
\usepackage{bbm}
\usepackage{mathtools}
%\usepackage{index}
\usepackage{makeidx}
%\usepackage{showidx} % Show indeces on pages
\usepackage[
    unicode=true,
    breaklinks,
    colorlinks=true,
    linkcolor=blue,
    ]{hyperref}
\usepackage{enumitem}
\usepackage{verbatim}
\usepackage{framed}
\usepackage{pstricks}
\usepackage{cancel} % formulas cancel
\author{}

\pdfcompresslevel=9

\everymath{\displaystyle}

\theoremstyle{plain}
\newtheorem{affirmation}{Утверждение}[section]
\newtheorem{theorem}{Теорема}[section]
\newtheorem{lemma}{Лемма}[section]
\newtheorem{remark}{Замечание}

\theoremstyle{definition}
\newtheorem{definition}{Определение}[section]
\newtheorem{example}{Пример}[section]

\theoremstyle{remark}

%\newcommand{\stcomp}[1]{{#1'}}
\newcommand{\stcomp}[1]{\overline{#1}}
\newcommand{\Probability}[1]{\mathbb{P}\left\{ #1 \right\}}
\newcommand{\probability}[1]{\mathbb{P}\left( #1 \right)}
\newcommand{\probabilityn}[1]{\mathbb{P}_n\left( #1 \right)}
\newcommand{\indicator}[1]{\mathbbm{1}\left( #1 \right)}
\newcommand{\Indicator}[1]{\mathbbm{1}\left\{ #1 \right\}}
\newcommand{\indicatorof}[1]{\mathbbm{1}_{#1}}
\newcommand{\meanof}[2]{M_{#1} #2}
\newcommand{\dispersionof}[2]{D_{#1} #2}
\newcommand{\dispersion}[1]{\dispersionof{}{#1}}
\newcommand{\mean}[1]{\meanof{}{#1}}
\newcommand{\covergence}[1]{\xrightarrow[n\to\infty]{#1}}
\newcommand{\Covergence}[1]{\xRightarrow[n\to\infty]{#1}}
\newcommand{\covergencen}[2]{\xrightarrow[#1\to\infty]{#2}}
\def \probabilityCovergenceText {\mathbb{P}}
\newcommand{\pcovergence}{\covergence{\probabilityCovergenceText}}
\newcommand{\pCovergence}{\Covergence{\probabilityCovergenceText}}
\def \almostSureCovergenceText {a.s.}
\newcommand{\acovergence}{\covergence{\almostSureCovergenceText}}
\newcommand{\aCovergence}{\Covergence{\almostSureCovergenceText}}
\def \distributionCovergenceText {d}
\newcommand{\dcovergence}{\covergence{\distributionCovergenceText}}
\newcommand{\dCovergence}{\Covergence{}}
\newcommand{\cdfof}[2]{F_{#1}\left(#2\right)}
\newcommand{\cdf}[1]{\cdfof{}{#1}}
\newcommand{\cdfn}[1]{F_n\left(#1\right)}
\newcommand{\CDFOF}[2]{F_{#1}\left\{#2\right\}}
\newcommand{\CDF}[1]{\CDFOF{}{#1}}
\newcommand{\pdf}[1]{p\left(#1\right)}

\newcommand{\integralc}[4]{\int\limits_{#1}^{#2} #4 #3}
\newcommand{\integral}[4]{\integralc{#1}{#2}{d #3}{#4}}
\newcommand{\integralp}[4]{\integralc{#1}{#2}{\partial #3}{#4}}

\newcommand{\argmaxof}[2]{\underset{#2}{\operatorname{arg\,max}}#1}
\newcommand{\argmax}[1]{\argmaxof{#1}{}}

\renewcommand{\thefootnote}{\fnsymbol{footnote}} 
