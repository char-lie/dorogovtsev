\everymath{\displaystyle}

\mdfsetup{skipbelow=\topskip}

\mdfdefinestyle{theoremstyle}{%
    apptotikzsetting={%
        \tikzset{mdfbackground/.append style ={%
                %top color=white!95!black,
                %bottom color=white
            },
            mdfframetitlebackground/.append style={
                top color=white!90!black,
                bottom color=white
            }
        }%
    },
    roundcorner=10pt,
    linewidth=3pt,
    middlelinewidth=1pt,
    shadow=true,
    %frametitlerule=true,
    %frametitlerulewidth=1pt,
    innertopmargin=10pt,%
}%{theorem}{Теорема}[section]

\mdfdefinestyle{remarkstyle}{%
    rightline=false,
    %bottomline=false,
    %roundcorner=10pt,
    linecolor=gray,
    innertopmargin=5pt,
    frametitlerule=true,
    %frametitlerulewidth=1pt,
    %frametitlerulecolor=gray,
}

\mdfdefinestyle{examplestyle}{%
    apptotikzsetting={%
        \tikzset{
                mdfframetitlebackground/.append style={
                right color=white!90!black,
                left color=white
            }
        }
    },
    leftline=false,
    topline=false,
    linecolor=white!90!black,
    linewidth=2pt,
    innertopmargin=3pt,
}

\newtheoremstyle{mystyle}
  {\topsep}   % ABOVESPACE
  {\topsep}   % BELOWSPACE
  {\itshape}  % BODYFONT
  {0pt}       % INDENT (empty value is the same as 0pt)
  {\bfseries} % HEADFONT
  {\newline\newline}         % HEADPUNCT
  {5pt plus 1pt minus 1pt} % HEADSPACE
  {}          % CUSTOM-HEAD-SPEC

\theoremstyle{plain}

%\begin{comment}
\mdtheorem[style=theoremstyle]{theorem}{Теорема}[section]
\mdtheorem[style=theoremstyle]{lemma}[theorem]{Лемма}
\mdtheorem[style=theoremstyle]{affirmation}[theorem]{Утверждение}
\mdtheorem[style=remarkstyle]{remark}[theorem]{Замечание}
%\end{comment}
\begin{comment}
\newtheorem{theorem}{Теорема}[section]
\newtheorem{lemma}{Лемма}[theorem]
\newtheorem{affirmation}[theorem]{Утверждение}
\newtheorem{remark}[theorem]{Замечание}
\end{comment}

\theoremstyle{definition}
%\begin{comment}
\mdtheorem[style=theoremstyle]{definition}[theorem]{Определение}
\mdtheorem[style=examplestyle]{example}[theorem]{Пример}
%\end{comment}
\begin{comment}
\newtheorem{definition}[theorem]{Определение}
\newtheorem{example}[theorem]{Пример}
\end{comment}

\theoremstyle{remark}

% Breaking inline math when comma's detected
% http://users.ugent.be/~gdschrij/LaTeX/textricks.html#mathcommabreak
\makeatletter
\def\old@comma{,}
\catcode`\,=13
\def,{%
  \ifmmode%
    \old@comma\discretionary{}{}{}%
  \else%
    \old@comma%
  \fi%
}
\makeatother
% End of trick

%\newcommand{\stcomp}[1]{{#1'}}
\newcommand{\stcomp}[1]{\overline{#1}}
\newcommand{\Probability}[1]{\mathbb{P}\left\{ #1 \right\}}
\newcommand{\probability}[1]{\mathbb{P}\left( #1 \right)}
\newcommand{\probabilityn}[1]{\mathbb{P}_n\left( #1 \right)}
\newcommand{\indicator}[1]{\mathbbm{1}\!\left( #1 \right)}
\newcommand{\Indicator}[1]{\mathbbm{1}\!\left\{ #1 \right\}}
\newcommand{\indicatorof}[1]{\mathbbm{1}_{#1}}
\newcommand{\Indicatorof}[1]{\indicatorof{\left\{ #1 \right\}}}
\newcommand{\meanof}[2]{\operatorname{M}_{#1} #2}
\newcommand{\Meanof}[2]{\meanof{#1}{\left[ #2 \right]}}
\newcommand{\dispersionof}[2]{\operatorname{D}_{#1} #2}
\newcommand{\dispersion}[1]{\dispersionof{}{#1}}
\newcommand{\mean}[1]{\meanof{}{#1}}
\newcommand{\Mean}[1]{\Meanof{}{#1}}
\newcommand{\covergence}[1]{\xrightarrow[n\to\infty]{#1}}
\newcommand{\Covergence}[1]{\xRightarrow[n\to\infty]{#1}}
\newcommand{\covergencen}[2]{\xrightarrow[#1\to\infty]{#2}}
\def \probabilityCovergenceText {\mathbb{P}}
\newcommand{\pcovergence}{\covergence{\probabilityCovergenceText}}
\newcommand{\pCovergence}{\Covergence{\probabilityCovergenceText}}
\def \almostSureCovergenceText {a.s.}
\newcommand{\acovergence}{\covergence{\almostSureCovergenceText}}
\newcommand{\aCovergence}{\Covergence{\almostSureCovergenceText}}
\def \distributionCovergenceText {d}
\newcommand{\dcovergence}{\covergence{\distributionCovergenceText}}
\newcommand{\dCovergence}{\Covergence{}}
\newcommand{\cdfof}[2]{F_{#1}\left(#2\right)}
\newcommand{\cdf}[1]{\cdfof{}{#1}}
\newcommand{\cdfn}[1]{F_n\left(#1\right)}
\newcommand{\CDFOF}[2]{F_{#1}\left\{#2\right\}}
\newcommand{\CDF}[1]{\CDFOF{}{#1}}
\newcommand{\pdf}[1]{p\left(#1\right)}
\def \mcond {\;\middle|\;}
\newcommand{\cov}[1]{\operatorname{cov}\!\left( #1 \right)}
\newcommand{\Cov}[2]{\operatorname{Cov}_{#1, #2}\,}
\newcommand{\cCov}[2]{\operatorname{Cov}_{#1, #2}^*\,}
\newcommand{\dCov}[1]{\Cov{#1}{#1}}
\newcommand{\dcCov}[1]{\cCov{#1}{#1}}

% Common integral macro
% #1: lower limit
% #2: upper limit
% #3: differential (measure)
% #4: integrand
\newcommand{\integralc}[4]{\int\limits_{#1}^{#2} #4 \, #3}
% Riemann integral macro
% #3: argument (differential without `d')
\newcommand{\integral}[4]{\integralc{#1}{#2}{d #3}{#4}}
% Integral with partial differential
\newcommand{\integralp}[4]{\integralc{#1}{#2}{\partial #3}{#4}}
% Lebesgue integral
% #1: domain of integration
% #2: measure
% #3: integrand
\newcommand{\integrall}[3]{\integralc{#1}{ }{#2}{#3}}

\newcommand{\argmaxof}[2]{\underset{#2}{\operatorname{arg\,max}}#1}
\newcommand{\argmax}[1]{\argmaxof{#1}{}}

\renewcommand{\thefootnote}{\fnsymbol{footnote}} 
