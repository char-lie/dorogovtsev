\everymath{\displaystyle}

% Breaking inline math when comma's detected
% http://users.ugent.be/~gdschrij/LaTeX/textricks.html#mathcommabreak
% Also it breaks my Tikz pictures...
\begin{comment}
\makeatletter
\catcode\setminus,=13
\def,{%
    \ifmmode%
    \old@comma\discretionary\left\{ }{%
    %    \old@comma$ $%
    %    $\old@comma\ $%
      \else%
        \old@comma%
        \fi%
      }}<++> \right\}<++>}<++>
\makeatother
\end{comment}
% End of trick

%\newcommand{\stcomp}[1]{{#1'}}
\newcommand{\stcomp}[1]{\overline{#1}}
\newcommand{\Probability}[2][{}]{\mathbb{P}_{#1}\left\{ #2 \right\}}
\newcommand{\probability}[1]{\mathbb{P}\left( #1 \right)}
\newcommand{\probabilityn}[1]{\mathbb{P}_n\left( #1 \right)}
\newcommand{\indicator}[1]{\mathbbm{1}\!\left( #1 \right)}
\newcommand{\Indicator}[1]{\mathbbm{1}\!\left\{ #1 \right\}}
\newcommand{\indicatorof}[1]{\mathbbm{1}_{#1}}
\newcommand{\Indicatorof}[1]{\indicatorof{\left\{ #1 \right\}}}
\newcommand{\meanof}[2]{\operatorname{M}_{#1} #2}
\newcommand{\Meanof}[2]{\meanof{#1}{\left[ #2 \right]}}
\newcommand{\dispersionof}[2]{\operatorname{D}_{#1} #2}
\newcommand{\Dispersionof}[2]{\dispersionof{#1}{\left[ #2 \right]}}
\newcommand{\dispersion}[1]{\dispersionof{}{#1}}
\newcommand{\mean}[1]{\meanof{}{#1}}
\newcommand{\Mean}[1]{\Meanof{}{#1}}
\newcommand{\covergence}[1]{\xrightarrow[n\to\infty]{#1}}
\newcommand{\Covergence}[1]{\xRightarrow[n\to\infty]{#1}}
\newcommand{\covergencen}[2]{\xrightarrow[#1\to\infty]{#2}}
\def \probabilityCovergenceText {\mathbb{P}}
\newcommand{\pcovergence}{\covergence{\probabilityCovergenceText}}
\newcommand{\pCovergence}{\Covergence{\probabilityCovergenceText}}
\def \almostSureCovergenceText {a.s.}
\newcommand{\acovergence}{\covergence{\almostSureCovergenceText}}
\newcommand{\aCovergence}{\Covergence{\almostSureCovergenceText}}
\def \distributionCovergenceText {d}
\newcommand{\dcovergence}{\covergence{\distributionCovergenceText}}
\newcommand{\dCovergence}{\Covergence{}}
\newcommand{\cdfof}[2]{F_{#1}\left(#2\right)}
\newcommand{\cdf}[1]{\cdfof{}{#1}}
\newcommand{\cdfn}[1]{F_n\left(#1\right)}
\newcommand{\CDFOF}[2]{F_{#1}\left\{#2\right\}}
\newcommand{\CDF}[1]{\CDFOF{}{#1}}
\newcommand{\pdf}[1]{p\left(#1\right)}
\newcommand{\pdfof}[2]{p_{#1}\left(#2\right)}
\def \mcond {\;\middle|\;}
\newcommand{\cov}[1]{\operatorname{cov}\!\left( #1 \right)}
\newcommand{\Cov}[3][]{\operatorname{Cov^{#1}_{#2, #3}}}
\newcommand{\cCov}[2]{\Cov[*]{#1}{#2}}
\newcommand{\tCov}[2]{\Cov[T]{#1}{#2}}
\newcommand{\dCov}[2][]{\Cov[#1]{#2}{#2}}
\newcommand{\dcCov}[1]{\cCov{#1}{#1}}

% Common integral macro
% #1: lower limit
% #2: upper limit
% #3: differential (measure)
% #4: integrand
\newcommand{\integralc}[4]{\int\limits_{#1}^{#2} #4 \, #3}
% Riemann integral macro
% #3: argument (differential without `d')
\newcommand{\integral}[4]{\integralc{#1}{#2}{d #3}{#4}}
% Integral with partial differential
\newcommand{\integralp}[4]{\integralc{#1}{#2}{\partial #3}{#4}}
% Lebesgue integral
% #1: domain of integration
% #2: measure
% #3: integrand
\newcommand{\integrall}[3]{\integralc{#1}{ }{#2}{#3}}

\newcommand{\argmaxof}[2]{\underset{#2}{\operatorname{arg\,max}}#1}
\newcommand{\argmax}[1]{\argmaxof{#1}{}}

\newcommand{\argminof}[2]{\underset{#2}{\operatorname{arg\,min}}#1}
\newcommand{\argmin}[1]{\argminof{#1}{}}

\newcommand{\rank}[1]{\operatorname{rank}{#1}}

\renewcommand{\thefootnote}{\fnsymbol{footnote}} 

\def\xsample{$x_1$, $\dots$, $x_n$}
